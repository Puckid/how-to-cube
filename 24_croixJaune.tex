\documentclass[0_Main.tex]{subfiles}
\begin{document}

\showto{french}{\section{Croix Jaune}}
\showto{english}{\section{Yellow Cross}}
\begin{wrapfigure}{r}{0.42\columnwidth}
\vspace{-20pt}
\centering
	\def\CAA{cblue}
	\def\CAB{cblue}
	\def\CAC{cblue}
	\def\CAD{cblue}
	\def\CAE{cblue}
	\def\CAF{cblue}
	\def\CAG{cgray}
	\def\CAH{cgray}
	\def\CAI{cgray}
	
	\def\CBA{cred}
	\def\CBB{cred}
	\def\CBC{cred}
	\def\CBD{cred}
	\def\CBE{cred}
	\def\CBF{cred}
	\def\CBG{cgray}
	\def\CBH{cgray}
	\def\CBI{cgray}
	
	\def\CCA{cgray}
	\def\CCB{cgray}
	\def\CCC{cgray}
	\def\CCD{cgray}
	\def\CCE{cyellow}
	\def\CCF{cgray}
	\def\CCG{cgray}
	\def\CCH{cgray}
	\def\CCI{cgray}
	\resizebox{0.34\columnwidth}{!}
	{
		\def\OFFSET{1.4cm}
	\def\OFF{2.5pt}
	\def\RATIO{0.2}
    \def\HXA{2.5cm}
    \def\HYA{-0.5cm}
    \def\VXA{0.0cm}
    \def\VYA{2.55cm}
    \def\HXB{1.5cm}
    \def\HYB{0.85cm}
    \def\VXB{\VXA}
    \def\VYB{\VYA}
    \def\HXC{\HXA}
    \def\HYC{\HYA}
    \def\VXC{\HXB}
    \def\VYC{\HYB}
    \tikzstyle{square}=[line width=3pt, join=round, cap=round]
    \tikzstyle{side}=[line width=5pt, join=round, cap=round]
    \tikzstyle{arrow}=[line width=9pt, join=round, cap=round,->,rounded corners=3cm, cpurple]
    %\tikzstyle{double arrow}=[9pt colored by black and white]
    \tikzstyle{border}=[line width=2pt, join=round, cap=round]
    \begin{tikzpicture}[>=triangle 45]
		\coordinate (vh1) at (\HXA ,\HYA);
		\coordinate (vv1) at (\VXA, \VYA);
		\coordinate (vh2) at (\HXB, \HYB);
		\coordinate (vv2) at (\VXB, \VYB);
		\coordinate (vh3) at (\HXC, \HYC);
		\coordinate (vv3) at (\VXC, \VYC);
		    
    	\coordinate (p100) at (0,0);
    	\coordinate (p101) at ($(p100)+1*(vh1)$);
    	\coordinate (p102) at ($(p100)+2*(vh1)$);
    	\coordinate (p103) at ($(p100)+3*(vh1)$);
    	\coordinate (p110) at ($(p100)+1*(vv1)$);
    	\coordinate (p111) at ($(p110)+1*(vh1)$);
		\coordinate (p112) at ($(p110)+2*(vh1)$);
    	\coordinate (p113) at ($(p110)+3*(vh1)$);
    	\coordinate (p120) at ($(p100)+2*(vv1)$);
    	\coordinate (p121) at ($(p120)+1*(vh1)$);
    	\coordinate (p122) at ($(p120)+2*(vh1)$);
    	\coordinate (p123) at ($(p120)+3*(vh1)$);
    	\coordinate (p130) at ($(p100)+3*(vv1)$);
    	\coordinate (p131) at ($(p130)+1*(vh1)$);
    	\coordinate (p132) at ($(p130)+2*(vh1)$);
    	\coordinate (p133) at ($(p130)+3*(vh1)$);
    	
    	\coordinate (p200) at (p103);
    	\coordinate (p201) at ($(p200)+1*(vh2)$);
    	\coordinate (p202) at ($(p200)+2*(vh2)$);
    	\coordinate (p203) at ($(p200)+3*(vh2)$);
    	\coordinate (p210) at ($(p200)+1*(vv2)$);
    	\coordinate (p211) at ($(p210)+1*(vh2)$);
		\coordinate (p212) at ($(p210)+2*(vh2)$);
    	\coordinate (p213) at ($(p210)+3*(vh2)$);
    	\coordinate (p220) at ($(p200)+2*(vv2)$);
    	\coordinate (p221) at ($(p220)+1*(vh2)$);
    	\coordinate (p222) at ($(p220)+2*(vh2)$);
    	\coordinate (p223) at ($(p220)+3*(vh2)$);
    	\coordinate (p230) at ($(p200)+3*(vv2)$);
    	\coordinate (p231) at ($(p230)+1*(vh2)$);
    	\coordinate (p232) at ($(p230)+2*(vh2)$);
    	\coordinate (p233) at ($(p230)+3*(vh2)$);
    	
		\coordinate (p300) at (p130);
    	\coordinate (p301) at ($(p300)+1*(vh3)$);
    	\coordinate (p302) at ($(p300)+2*(vh3)$);
    	\coordinate (p303) at ($(p300)+3*(vh3)$);
    	\coordinate (p310) at ($(p300)+1*(vv3)$);
    	\coordinate (p311) at ($(p310)+1*(vh3)$);
		\coordinate (p312) at ($(p310)+2*(vh3)$);
    	\coordinate (p313) at ($(p310)+3*(vh3)$);
    	\coordinate (p320) at ($(p300)+2*(vv3)$);
    	\coordinate (p321) at ($(p320)+1*(vh3)$);
    	\coordinate (p322) at ($(p320)+2*(vh3)$);
    	\coordinate (p323) at ($(p320)+3*(vh3)$);
    	\coordinate (p330) at ($(p300)+3*(vv3)$);
    	\coordinate (p331) at ($(p330)+1*(vh3)$);
    	\coordinate (p332) at ($(p330)+2*(vh3)$);
    	\coordinate (p333) at ($(p330)+3*(vh3)$);  	  	
    	
		\filldraw[square, fill=\CAA] (p100) -- (p101) -- (p111) -- (p110);
		\filldraw[square, fill=\CAB] (p101) -- (p102) -- (p112) -- (p111);
		\filldraw[square, fill=\CAC] (p102) -- (p103) -- (p113) -- (p112);
		\filldraw[square, fill=\CAD] (p110) -- (p111) -- (p121) -- (p120);
		\filldraw[square, fill=\CAE] (p111) -- (p112) -- (p122) -- (p121);
		\filldraw[square, fill=\CAF] (p112) -- (p113) -- (p123) -- (p122);
		\filldraw[square, fill=\CAG] (p120) -- (p121) -- (p131) -- (p130);
		\filldraw[square, fill=\CAH] (p121) -- (p122) -- (p132) -- (p131);
		\filldraw[square, fill=\CAI] (p122) -- (p123) -- (p133) -- (p132);
		
		\filldraw[square, fill=\CBA] (p200) -- (p201) -- (p211) -- (p210);
		\filldraw[square, fill=\CBB] (p201) -- (p202) -- (p212) -- (p211);
		\filldraw[square, fill=\CBC] (p202) -- (p203) -- (p213) -- (p212);
		\filldraw[square, fill=\CBD] (p210) -- (p211) -- (p221) -- (p220);
		\filldraw[square, fill=\CBE] (p211) -- (p212) -- (p222) -- (p221);
		\filldraw[square, fill=\CBF] (p212) -- (p213) -- (p223) -- (p222);
		\filldraw[square, fill=\CBG] (p220) -- (p221) -- (p231) -- (p230);
		\filldraw[square, fill=\CBH] (p221) -- (p222) -- (p232) -- (p231);
		\filldraw[square, fill=\CBI] (p222) -- (p223) -- (p233) -- (p232);
		
		\filldraw[square, fill=\CCA] (p300) -- (p301) -- (p311) -- (p310);
		\filldraw[square, fill=\CCB] (p301) -- (p302) -- (p312) -- (p311);
		\filldraw[square, fill=\CCC] (p302) -- (p303) -- (p313) -- (p312);
		\filldraw[square, fill=\CCD] (p310) -- (p311) -- (p321) -- (p320);
		\filldraw[square, fill=\CCE] (p311) -- (p312) -- (p322) -- (p321);
		\filldraw[square, fill=\CCF] (p312) -- (p313) -- (p323) -- (p322);
		\filldraw[square, fill=\CCG] (p320) -- (p321) -- (p331) -- (p330);
		\filldraw[square, fill=\CCH] (p321) -- (p322) -- (p332) -- (p331);
		\filldraw[square, fill=\CCI] (p322) -- (p323) -- (p333) -- (p332);
	
    	\draw[side] (p100) -- (p200) -- (p203) -- (p233) -- (p330) -- (p130) -- (p100) -- (p200);
    	\draw[side] (p130) -- (p133) -- (p233);
    	\draw[side] (p103) -- (p133);
    	
    	\draw[square] (p101) -- (p131) -- (p331);
    	\draw[square] (p102) -- (p132) -- (p332);
    	\draw[square] (p201) -- (p231) -- (p310);
    	\draw[square] (p202) -- (p232) -- (p320);
    	\draw[square] (p110) -- (p113) -- (p213);
    	\draw[square] (p120) -- (p123) -- (p223);
	\end{tikzpicture}
	}    
	%\vspace{-20pt}
	\showto{french}{\caption{\label{fig:croixJaune} Couronne finie.}}
	\showto{english}{\caption{\label{fig:croixJaune} Finished crown.}}
	\vspace{-10pt}
\end{wrapfigure}

\showto{french}{A ce stade, les deux premiers étages devraient être finis et votre cube devrait ressembler à celui ci-contre. Pour les prochaines représentations, une vue de dessus pourra être choisie. Les couleurs sur le côté des cubes sont visibles avec les petits rectangles sur les côtés (voir p. 8). Et aussi, chaque algorithme présenté après ce point commence avec la face jaune sur le dessus, donc s’il faut tourner le cube afin de l’avoir dans la bonne position, faites en sorte de toujours garder la face jaune dessus.

Comme pour le début du cube, la résolution de la dernière face commence en créant une croix, pour commencer avec les côtés ne correspondant par forcément aux centres (contrairement à la première étape).

Nous nous intéressons donc uniquement aux arêtes de la face jaune, donc sans prendre en compte les coins. Par chance, il n’y a que trois cas possibles arrivé à ce stade, donc si votre cube se retrouve dans une autre position que l’un des trois cas représentés ci-dessous, il doit y avoir une erreur dans la couronne, ou si votre cube a déjà été démonté, il se peut qu’il ait été remonté faux.}

\showto{english}{At this point, the first two layers should be finished and your cube should look like the one beside. For the next representation, a view from above can be chosen. The colors on the side of the cubes are visible with the small rectangles on the sides (see page 8). And also, every algorithm presented after this point starts with the yellow face on the top, so if you have to turn the cube to have it in the right position, make sure to always keep the yellow face on top.

As for the beginning of the cube, the resolution of the last face begins by creating a cross, to start with the sides not necessarily corresponding to the centers (unlike the first step).

We are therefore interested only in the edges of the yellow face, so without taking into account the corners. Luckily, there are only three possible cases happening at this point, so if your cube is in a different position than one of the three cases shown below, there must be an error in the crown, or if your cube has already been disassembled, it may have gone wrong.}

\begin{figure}[H]
\def\W{0.24}
\def\S{0.01mm}
\centering
\begin{subfigure}[t]{\W\columnwidth}
\centering
	\def\CAA{cblue}
	\def\CAB{cblue}
	\def\CAC{cblue}
	\def\CAD{cblue}
	\def\CAE{cblue}
	\def\CAF{cblue}
	\def\CAG{cgray}
	\def\CAH{cyellow}
	\def\CAI{cgray}
	
	\def\CBA{cred}
	\def\CBB{cred}
	\def\CBC{cred}
	\def\CBD{cred}
	\def\CBE{cred}
	\def\CBF{cred}
	\def\CBG{cgray}
	\def\CBH{cgray}
	\def\CBI{cgray}
	
	\def\CCA{cgray}
	\def\CCB{cgray}
	\def\CCC{cgray}
	\def\CCD{cyellow}
	\def\CCE{cyellow}
	\def\CCF{cyellow}
	\def\CCG{cgray}
	\def\CCH{cgray}
	\def\CCI{cgray}
	\resizebox{\columnwidth}{!}
	{
		\def\OFFSET{1.4cm}
	\def\OFF{2.5pt}
	\def\RATIO{0.2}
    \def\HXA{2.5cm}
    \def\HYA{-0.5cm}
    \def\VXA{0.0cm}
    \def\VYA{2.55cm}
    \def\HXB{1.5cm}
    \def\HYB{0.85cm}
    \def\VXB{\VXA}
    \def\VYB{\VYA}
    \def\HXC{\HXA}
    \def\HYC{\HYA}
    \def\VXC{\HXB}
    \def\VYC{\HYB}
    \tikzstyle{square}=[line width=3pt, join=round, cap=round]
    \tikzstyle{side}=[line width=5pt, join=round, cap=round]
    \tikzstyle{arrow}=[line width=9pt, join=round, cap=round,->,rounded corners=3cm, cpurple]
    %\tikzstyle{double arrow}=[9pt colored by black and white]
    \tikzstyle{border}=[line width=2pt, join=round, cap=round]
    \begin{tikzpicture}[>=triangle 45]
		\coordinate (vh1) at (\HXA ,\HYA);
		\coordinate (vv1) at (\VXA, \VYA);
		\coordinate (vh2) at (\HXB, \HYB);
		\coordinate (vv2) at (\VXB, \VYB);
		\coordinate (vh3) at (\HXC, \HYC);
		\coordinate (vv3) at (\VXC, \VYC);
		    
    	\coordinate (p100) at (0,0);
    	\coordinate (p101) at ($(p100)+1*(vh1)$);
    	\coordinate (p102) at ($(p100)+2*(vh1)$);
    	\coordinate (p103) at ($(p100)+3*(vh1)$);
    	\coordinate (p110) at ($(p100)+1*(vv1)$);
    	\coordinate (p111) at ($(p110)+1*(vh1)$);
		\coordinate (p112) at ($(p110)+2*(vh1)$);
    	\coordinate (p113) at ($(p110)+3*(vh1)$);
    	\coordinate (p120) at ($(p100)+2*(vv1)$);
    	\coordinate (p121) at ($(p120)+1*(vh1)$);
    	\coordinate (p122) at ($(p120)+2*(vh1)$);
    	\coordinate (p123) at ($(p120)+3*(vh1)$);
    	\coordinate (p130) at ($(p100)+3*(vv1)$);
    	\coordinate (p131) at ($(p130)+1*(vh1)$);
    	\coordinate (p132) at ($(p130)+2*(vh1)$);
    	\coordinate (p133) at ($(p130)+3*(vh1)$);
    	
    	\coordinate (p200) at (p103);
    	\coordinate (p201) at ($(p200)+1*(vh2)$);
    	\coordinate (p202) at ($(p200)+2*(vh2)$);
    	\coordinate (p203) at ($(p200)+3*(vh2)$);
    	\coordinate (p210) at ($(p200)+1*(vv2)$);
    	\coordinate (p211) at ($(p210)+1*(vh2)$);
		\coordinate (p212) at ($(p210)+2*(vh2)$);
    	\coordinate (p213) at ($(p210)+3*(vh2)$);
    	\coordinate (p220) at ($(p200)+2*(vv2)$);
    	\coordinate (p221) at ($(p220)+1*(vh2)$);
    	\coordinate (p222) at ($(p220)+2*(vh2)$);
    	\coordinate (p223) at ($(p220)+3*(vh2)$);
    	\coordinate (p230) at ($(p200)+3*(vv2)$);
    	\coordinate (p231) at ($(p230)+1*(vh2)$);
    	\coordinate (p232) at ($(p230)+2*(vh2)$);
    	\coordinate (p233) at ($(p230)+3*(vh2)$);
    	
		\coordinate (p300) at (p130);
    	\coordinate (p301) at ($(p300)+1*(vh3)$);
    	\coordinate (p302) at ($(p300)+2*(vh3)$);
    	\coordinate (p303) at ($(p300)+3*(vh3)$);
    	\coordinate (p310) at ($(p300)+1*(vv3)$);
    	\coordinate (p311) at ($(p310)+1*(vh3)$);
		\coordinate (p312) at ($(p310)+2*(vh3)$);
    	\coordinate (p313) at ($(p310)+3*(vh3)$);
    	\coordinate (p320) at ($(p300)+2*(vv3)$);
    	\coordinate (p321) at ($(p320)+1*(vh3)$);
    	\coordinate (p322) at ($(p320)+2*(vh3)$);
    	\coordinate (p323) at ($(p320)+3*(vh3)$);
    	\coordinate (p330) at ($(p300)+3*(vv3)$);
    	\coordinate (p331) at ($(p330)+1*(vh3)$);
    	\coordinate (p332) at ($(p330)+2*(vh3)$);
    	\coordinate (p333) at ($(p330)+3*(vh3)$);  	  	
    	
		\filldraw[square, fill=\CAA] (p100) -- (p101) -- (p111) -- (p110);
		\filldraw[square, fill=\CAB] (p101) -- (p102) -- (p112) -- (p111);
		\filldraw[square, fill=\CAC] (p102) -- (p103) -- (p113) -- (p112);
		\filldraw[square, fill=\CAD] (p110) -- (p111) -- (p121) -- (p120);
		\filldraw[square, fill=\CAE] (p111) -- (p112) -- (p122) -- (p121);
		\filldraw[square, fill=\CAF] (p112) -- (p113) -- (p123) -- (p122);
		\filldraw[square, fill=\CAG] (p120) -- (p121) -- (p131) -- (p130);
		\filldraw[square, fill=\CAH] (p121) -- (p122) -- (p132) -- (p131);
		\filldraw[square, fill=\CAI] (p122) -- (p123) -- (p133) -- (p132);
		
		\filldraw[square, fill=\CBA] (p200) -- (p201) -- (p211) -- (p210);
		\filldraw[square, fill=\CBB] (p201) -- (p202) -- (p212) -- (p211);
		\filldraw[square, fill=\CBC] (p202) -- (p203) -- (p213) -- (p212);
		\filldraw[square, fill=\CBD] (p210) -- (p211) -- (p221) -- (p220);
		\filldraw[square, fill=\CBE] (p211) -- (p212) -- (p222) -- (p221);
		\filldraw[square, fill=\CBF] (p212) -- (p213) -- (p223) -- (p222);
		\filldraw[square, fill=\CBG] (p220) -- (p221) -- (p231) -- (p230);
		\filldraw[square, fill=\CBH] (p221) -- (p222) -- (p232) -- (p231);
		\filldraw[square, fill=\CBI] (p222) -- (p223) -- (p233) -- (p232);
		
		\filldraw[square, fill=\CCA] (p300) -- (p301) -- (p311) -- (p310);
		\filldraw[square, fill=\CCB] (p301) -- (p302) -- (p312) -- (p311);
		\filldraw[square, fill=\CCC] (p302) -- (p303) -- (p313) -- (p312);
		\filldraw[square, fill=\CCD] (p310) -- (p311) -- (p321) -- (p320);
		\filldraw[square, fill=\CCE] (p311) -- (p312) -- (p322) -- (p321);
		\filldraw[square, fill=\CCF] (p312) -- (p313) -- (p323) -- (p322);
		\filldraw[square, fill=\CCG] (p320) -- (p321) -- (p331) -- (p330);
		\filldraw[square, fill=\CCH] (p321) -- (p322) -- (p332) -- (p331);
		\filldraw[square, fill=\CCI] (p322) -- (p323) -- (p333) -- (p332);
	
    	\draw[side] (p100) -- (p200) -- (p203) -- (p233) -- (p330) -- (p130) -- (p100) -- (p200);
    	\draw[side] (p130) -- (p133) -- (p233);
    	\draw[side] (p103) -- (p133);
    	
    	\draw[square] (p101) -- (p131) -- (p331);
    	\draw[square] (p102) -- (p132) -- (p332);
    	\draw[square] (p201) -- (p231) -- (p310);
    	\draw[square] (p202) -- (p232) -- (p320);
    	\draw[square] (p110) -- (p113) -- (p213);
    	\draw[square] (p120) -- (p123) -- (p223);
	\end{tikzpicture}
	}   
	\showto{french}{\caption*{\label{fig:croixBarre} \LARGE{Barre}}}
	\showto{english}{\caption*{\label{fig:croixBarre} \LARGE{Bar}}}
\end{subfigure}
\hspace{\S}
\begin{subfigure}[t]{\W\columnwidth}
\centering
	\def\CAA{cblue}
	\def\CAB{cblue}
	\def\CAC{cblue}
	\def\CAD{cblue}
	\def\CAE{cblue}
	\def\CAF{cblue}
	\def\CAG{cgray}
	\def\CAH{cyellow}
	\def\CAI{cgray}
	
	\def\CBA{cred}
	\def\CBB{cred}
	\def\CBC{cred}
	\def\CBD{cred}
	\def\CBE{cred}
	\def\CBF{cred}
	\def\CBG{cgray}
	\def\CBH{cgray}
	\def\CBI{cgray}
	
	\def\CCA{cgray}
	\def\CCB{cgray}
	\def\CCC{cgray}
	\def\CCD{cgray}
	\def\CCE{cyellow}
	\def\CCF{cyellow}
	\def\CCG{cgray}
	\def\CCH{cyellow}
	\def\CCI{cgray}
	\resizebox{\columnwidth}{!}
	{
		\def\OFFSET{1.4cm}
	\def\OFF{2.5pt}
	\def\RATIO{0.2}
    \def\HXA{2.5cm}
    \def\HYA{-0.5cm}
    \def\VXA{0.0cm}
    \def\VYA{2.55cm}
    \def\HXB{1.5cm}
    \def\HYB{0.85cm}
    \def\VXB{\VXA}
    \def\VYB{\VYA}
    \def\HXC{\HXA}
    \def\HYC{\HYA}
    \def\VXC{\HXB}
    \def\VYC{\HYB}
    \tikzstyle{square}=[line width=3pt, join=round, cap=round]
    \tikzstyle{side}=[line width=5pt, join=round, cap=round]
    \tikzstyle{arrow}=[line width=9pt, join=round, cap=round,->,rounded corners=3cm, cpurple]
    %\tikzstyle{double arrow}=[9pt colored by black and white]
    \tikzstyle{border}=[line width=2pt, join=round, cap=round]
    \begin{tikzpicture}[>=triangle 45]
		\coordinate (vh1) at (\HXA ,\HYA);
		\coordinate (vv1) at (\VXA, \VYA);
		\coordinate (vh2) at (\HXB, \HYB);
		\coordinate (vv2) at (\VXB, \VYB);
		\coordinate (vh3) at (\HXC, \HYC);
		\coordinate (vv3) at (\VXC, \VYC);
		    
    	\coordinate (p100) at (0,0);
    	\coordinate (p101) at ($(p100)+1*(vh1)$);
    	\coordinate (p102) at ($(p100)+2*(vh1)$);
    	\coordinate (p103) at ($(p100)+3*(vh1)$);
    	\coordinate (p110) at ($(p100)+1*(vv1)$);
    	\coordinate (p111) at ($(p110)+1*(vh1)$);
		\coordinate (p112) at ($(p110)+2*(vh1)$);
    	\coordinate (p113) at ($(p110)+3*(vh1)$);
    	\coordinate (p120) at ($(p100)+2*(vv1)$);
    	\coordinate (p121) at ($(p120)+1*(vh1)$);
    	\coordinate (p122) at ($(p120)+2*(vh1)$);
    	\coordinate (p123) at ($(p120)+3*(vh1)$);
    	\coordinate (p130) at ($(p100)+3*(vv1)$);
    	\coordinate (p131) at ($(p130)+1*(vh1)$);
    	\coordinate (p132) at ($(p130)+2*(vh1)$);
    	\coordinate (p133) at ($(p130)+3*(vh1)$);
    	
    	\coordinate (p200) at (p103);
    	\coordinate (p201) at ($(p200)+1*(vh2)$);
    	\coordinate (p202) at ($(p200)+2*(vh2)$);
    	\coordinate (p203) at ($(p200)+3*(vh2)$);
    	\coordinate (p210) at ($(p200)+1*(vv2)$);
    	\coordinate (p211) at ($(p210)+1*(vh2)$);
		\coordinate (p212) at ($(p210)+2*(vh2)$);
    	\coordinate (p213) at ($(p210)+3*(vh2)$);
    	\coordinate (p220) at ($(p200)+2*(vv2)$);
    	\coordinate (p221) at ($(p220)+1*(vh2)$);
    	\coordinate (p222) at ($(p220)+2*(vh2)$);
    	\coordinate (p223) at ($(p220)+3*(vh2)$);
    	\coordinate (p230) at ($(p200)+3*(vv2)$);
    	\coordinate (p231) at ($(p230)+1*(vh2)$);
    	\coordinate (p232) at ($(p230)+2*(vh2)$);
    	\coordinate (p233) at ($(p230)+3*(vh2)$);
    	
		\coordinate (p300) at (p130);
    	\coordinate (p301) at ($(p300)+1*(vh3)$);
    	\coordinate (p302) at ($(p300)+2*(vh3)$);
    	\coordinate (p303) at ($(p300)+3*(vh3)$);
    	\coordinate (p310) at ($(p300)+1*(vv3)$);
    	\coordinate (p311) at ($(p310)+1*(vh3)$);
		\coordinate (p312) at ($(p310)+2*(vh3)$);
    	\coordinate (p313) at ($(p310)+3*(vh3)$);
    	\coordinate (p320) at ($(p300)+2*(vv3)$);
    	\coordinate (p321) at ($(p320)+1*(vh3)$);
    	\coordinate (p322) at ($(p320)+2*(vh3)$);
    	\coordinate (p323) at ($(p320)+3*(vh3)$);
    	\coordinate (p330) at ($(p300)+3*(vv3)$);
    	\coordinate (p331) at ($(p330)+1*(vh3)$);
    	\coordinate (p332) at ($(p330)+2*(vh3)$);
    	\coordinate (p333) at ($(p330)+3*(vh3)$);  	  	
    	
		\filldraw[square, fill=\CAA] (p100) -- (p101) -- (p111) -- (p110);
		\filldraw[square, fill=\CAB] (p101) -- (p102) -- (p112) -- (p111);
		\filldraw[square, fill=\CAC] (p102) -- (p103) -- (p113) -- (p112);
		\filldraw[square, fill=\CAD] (p110) -- (p111) -- (p121) -- (p120);
		\filldraw[square, fill=\CAE] (p111) -- (p112) -- (p122) -- (p121);
		\filldraw[square, fill=\CAF] (p112) -- (p113) -- (p123) -- (p122);
		\filldraw[square, fill=\CAG] (p120) -- (p121) -- (p131) -- (p130);
		\filldraw[square, fill=\CAH] (p121) -- (p122) -- (p132) -- (p131);
		\filldraw[square, fill=\CAI] (p122) -- (p123) -- (p133) -- (p132);
		
		\filldraw[square, fill=\CBA] (p200) -- (p201) -- (p211) -- (p210);
		\filldraw[square, fill=\CBB] (p201) -- (p202) -- (p212) -- (p211);
		\filldraw[square, fill=\CBC] (p202) -- (p203) -- (p213) -- (p212);
		\filldraw[square, fill=\CBD] (p210) -- (p211) -- (p221) -- (p220);
		\filldraw[square, fill=\CBE] (p211) -- (p212) -- (p222) -- (p221);
		\filldraw[square, fill=\CBF] (p212) -- (p213) -- (p223) -- (p222);
		\filldraw[square, fill=\CBG] (p220) -- (p221) -- (p231) -- (p230);
		\filldraw[square, fill=\CBH] (p221) -- (p222) -- (p232) -- (p231);
		\filldraw[square, fill=\CBI] (p222) -- (p223) -- (p233) -- (p232);
		
		\filldraw[square, fill=\CCA] (p300) -- (p301) -- (p311) -- (p310);
		\filldraw[square, fill=\CCB] (p301) -- (p302) -- (p312) -- (p311);
		\filldraw[square, fill=\CCC] (p302) -- (p303) -- (p313) -- (p312);
		\filldraw[square, fill=\CCD] (p310) -- (p311) -- (p321) -- (p320);
		\filldraw[square, fill=\CCE] (p311) -- (p312) -- (p322) -- (p321);
		\filldraw[square, fill=\CCF] (p312) -- (p313) -- (p323) -- (p322);
		\filldraw[square, fill=\CCG] (p320) -- (p321) -- (p331) -- (p330);
		\filldraw[square, fill=\CCH] (p321) -- (p322) -- (p332) -- (p331);
		\filldraw[square, fill=\CCI] (p322) -- (p323) -- (p333) -- (p332);
	
    	\draw[side] (p100) -- (p200) -- (p203) -- (p233) -- (p330) -- (p130) -- (p100) -- (p200);
    	\draw[side] (p130) -- (p133) -- (p233);
    	\draw[side] (p103) -- (p133);
    	
    	\draw[square] (p101) -- (p131) -- (p331);
    	\draw[square] (p102) -- (p132) -- (p332);
    	\draw[square] (p201) -- (p231) -- (p310);
    	\draw[square] (p202) -- (p232) -- (p320);
    	\draw[square] (p110) -- (p113) -- (p213);
    	\draw[square] (p120) -- (p123) -- (p223);
	\end{tikzpicture}
	}    
	\caption*{\label{fig:croixL} \LARGE{L}}	
\end{subfigure}
\hspace{\S}
\begin{subfigure}[t]{\W\columnwidth}
\centering
	\def\CAA{cblue}
	\def\CAB{cblue}
	\def\CAC{cblue}
	\def\CAD{cblue}
	\def\CAE{cblue}
	\def\CAF{cblue}
	\def\CAG{cgray}
	\def\CAH{cyellow}
	\def\CAI{cgray}
	
	\def\CBA{cred}
	\def\CBB{cred}
	\def\CBC{cred}
	\def\CBD{cred}
	\def\CBE{cred}
	\def\CBF{cred}
	\def\CBG{cgray}
	\def\CBH{cyellow}
	\def\CBI{cgray}
	
	\def\CCA{cgray}
	\def\CCB{cgray}
	\def\CCC{cgray}
	\def\CCD{cgray}
	\def\CCE{cyellow}
	\def\CCF{cgray}
	\def\CCG{cgray}
	\def\CCH{cgray}
	\def\CCI{cgray}
	\resizebox{\columnwidth}{!}
	{
		\def\OFFSET{1.4cm}
	\def\OFF{2.5pt}
	\def\RATIO{0.2}
    \def\HXA{2.5cm}
    \def\HYA{-0.5cm}
    \def\VXA{0.0cm}
    \def\VYA{2.55cm}
    \def\HXB{1.5cm}
    \def\HYB{0.85cm}
    \def\VXB{\VXA}
    \def\VYB{\VYA}
    \def\HXC{\HXA}
    \def\HYC{\HYA}
    \def\VXC{\HXB}
    \def\VYC{\HYB}
    \tikzstyle{square}=[line width=3pt, join=round, cap=round]
    \tikzstyle{side}=[line width=5pt, join=round, cap=round]
    \tikzstyle{arrow}=[line width=9pt, join=round, cap=round,->,rounded corners=3cm, cpurple]
    %\tikzstyle{double arrow}=[9pt colored by black and white]
    \tikzstyle{border}=[line width=2pt, join=round, cap=round]
    \begin{tikzpicture}[>=triangle 45]
		\coordinate (vh1) at (\HXA ,\HYA);
		\coordinate (vv1) at (\VXA, \VYA);
		\coordinate (vh2) at (\HXB, \HYB);
		\coordinate (vv2) at (\VXB, \VYB);
		\coordinate (vh3) at (\HXC, \HYC);
		\coordinate (vv3) at (\VXC, \VYC);
		    
    	\coordinate (p100) at (0,0);
    	\coordinate (p101) at ($(p100)+1*(vh1)$);
    	\coordinate (p102) at ($(p100)+2*(vh1)$);
    	\coordinate (p103) at ($(p100)+3*(vh1)$);
    	\coordinate (p110) at ($(p100)+1*(vv1)$);
    	\coordinate (p111) at ($(p110)+1*(vh1)$);
		\coordinate (p112) at ($(p110)+2*(vh1)$);
    	\coordinate (p113) at ($(p110)+3*(vh1)$);
    	\coordinate (p120) at ($(p100)+2*(vv1)$);
    	\coordinate (p121) at ($(p120)+1*(vh1)$);
    	\coordinate (p122) at ($(p120)+2*(vh1)$);
    	\coordinate (p123) at ($(p120)+3*(vh1)$);
    	\coordinate (p130) at ($(p100)+3*(vv1)$);
    	\coordinate (p131) at ($(p130)+1*(vh1)$);
    	\coordinate (p132) at ($(p130)+2*(vh1)$);
    	\coordinate (p133) at ($(p130)+3*(vh1)$);
    	
    	\coordinate (p200) at (p103);
    	\coordinate (p201) at ($(p200)+1*(vh2)$);
    	\coordinate (p202) at ($(p200)+2*(vh2)$);
    	\coordinate (p203) at ($(p200)+3*(vh2)$);
    	\coordinate (p210) at ($(p200)+1*(vv2)$);
    	\coordinate (p211) at ($(p210)+1*(vh2)$);
		\coordinate (p212) at ($(p210)+2*(vh2)$);
    	\coordinate (p213) at ($(p210)+3*(vh2)$);
    	\coordinate (p220) at ($(p200)+2*(vv2)$);
    	\coordinate (p221) at ($(p220)+1*(vh2)$);
    	\coordinate (p222) at ($(p220)+2*(vh2)$);
    	\coordinate (p223) at ($(p220)+3*(vh2)$);
    	\coordinate (p230) at ($(p200)+3*(vv2)$);
    	\coordinate (p231) at ($(p230)+1*(vh2)$);
    	\coordinate (p232) at ($(p230)+2*(vh2)$);
    	\coordinate (p233) at ($(p230)+3*(vh2)$);
    	
		\coordinate (p300) at (p130);
    	\coordinate (p301) at ($(p300)+1*(vh3)$);
    	\coordinate (p302) at ($(p300)+2*(vh3)$);
    	\coordinate (p303) at ($(p300)+3*(vh3)$);
    	\coordinate (p310) at ($(p300)+1*(vv3)$);
    	\coordinate (p311) at ($(p310)+1*(vh3)$);
		\coordinate (p312) at ($(p310)+2*(vh3)$);
    	\coordinate (p313) at ($(p310)+3*(vh3)$);
    	\coordinate (p320) at ($(p300)+2*(vv3)$);
    	\coordinate (p321) at ($(p320)+1*(vh3)$);
    	\coordinate (p322) at ($(p320)+2*(vh3)$);
    	\coordinate (p323) at ($(p320)+3*(vh3)$);
    	\coordinate (p330) at ($(p300)+3*(vv3)$);
    	\coordinate (p331) at ($(p330)+1*(vh3)$);
    	\coordinate (p332) at ($(p330)+2*(vh3)$);
    	\coordinate (p333) at ($(p330)+3*(vh3)$);  	  	
    	
		\filldraw[square, fill=\CAA] (p100) -- (p101) -- (p111) -- (p110);
		\filldraw[square, fill=\CAB] (p101) -- (p102) -- (p112) -- (p111);
		\filldraw[square, fill=\CAC] (p102) -- (p103) -- (p113) -- (p112);
		\filldraw[square, fill=\CAD] (p110) -- (p111) -- (p121) -- (p120);
		\filldraw[square, fill=\CAE] (p111) -- (p112) -- (p122) -- (p121);
		\filldraw[square, fill=\CAF] (p112) -- (p113) -- (p123) -- (p122);
		\filldraw[square, fill=\CAG] (p120) -- (p121) -- (p131) -- (p130);
		\filldraw[square, fill=\CAH] (p121) -- (p122) -- (p132) -- (p131);
		\filldraw[square, fill=\CAI] (p122) -- (p123) -- (p133) -- (p132);
		
		\filldraw[square, fill=\CBA] (p200) -- (p201) -- (p211) -- (p210);
		\filldraw[square, fill=\CBB] (p201) -- (p202) -- (p212) -- (p211);
		\filldraw[square, fill=\CBC] (p202) -- (p203) -- (p213) -- (p212);
		\filldraw[square, fill=\CBD] (p210) -- (p211) -- (p221) -- (p220);
		\filldraw[square, fill=\CBE] (p211) -- (p212) -- (p222) -- (p221);
		\filldraw[square, fill=\CBF] (p212) -- (p213) -- (p223) -- (p222);
		\filldraw[square, fill=\CBG] (p220) -- (p221) -- (p231) -- (p230);
		\filldraw[square, fill=\CBH] (p221) -- (p222) -- (p232) -- (p231);
		\filldraw[square, fill=\CBI] (p222) -- (p223) -- (p233) -- (p232);
		
		\filldraw[square, fill=\CCA] (p300) -- (p301) -- (p311) -- (p310);
		\filldraw[square, fill=\CCB] (p301) -- (p302) -- (p312) -- (p311);
		\filldraw[square, fill=\CCC] (p302) -- (p303) -- (p313) -- (p312);
		\filldraw[square, fill=\CCD] (p310) -- (p311) -- (p321) -- (p320);
		\filldraw[square, fill=\CCE] (p311) -- (p312) -- (p322) -- (p321);
		\filldraw[square, fill=\CCF] (p312) -- (p313) -- (p323) -- (p322);
		\filldraw[square, fill=\CCG] (p320) -- (p321) -- (p331) -- (p330);
		\filldraw[square, fill=\CCH] (p321) -- (p322) -- (p332) -- (p331);
		\filldraw[square, fill=\CCI] (p322) -- (p323) -- (p333) -- (p332);
	
    	\draw[side] (p100) -- (p200) -- (p203) -- (p233) -- (p330) -- (p130) -- (p100) -- (p200);
    	\draw[side] (p130) -- (p133) -- (p233);
    	\draw[side] (p103) -- (p133);
    	
    	\draw[square] (p101) -- (p131) -- (p331);
    	\draw[square] (p102) -- (p132) -- (p332);
    	\draw[square] (p201) -- (p231) -- (p310);
    	\draw[square] (p202) -- (p232) -- (p320);
    	\draw[square] (p110) -- (p113) -- (p213);
    	\draw[square] (p120) -- (p123) -- (p223);
	\end{tikzpicture}
	}    
	\caption*{\label{fig:croixPoint} \LARGE{Point}}
\end{subfigure}
\hspace{\S}
\begin{subfigure}[t]{\W\columnwidth}
\centering
	\def\CAA{cblue}
	\def\CAB{cblue}
	\def\CAC{cblue}
	\def\CAD{cblue}
	\def\CAE{cblue}
	\def\CAF{cblue}
	\def\CAG{cgray}
	\def\CAH{cgray}
	\def\CAI{cgray}
	
	\def\CBA{cred}
	\def\CBB{cred}
	\def\CBC{cred}
	\def\CBD{cred}
	\def\CBE{cred}
	\def\CBF{cred}
	\def\CBG{cgray}
	\def\CBH{cgray}
	\def\CBI{cgray}
	
	\def\CCA{cgray}
	\def\CCB{cyellow}
	\def\CCC{cgray}
	\def\CCD{cyellow}
	\def\CCE{cyellow}
	\def\CCF{cyellow}
	\def\CCG{cgray}
	\def\CCH{cyellow}
	\def\CCI{cgray}
	\resizebox{\columnwidth}{!}
	{
		\def\OFFSET{1.4cm}
	\def\OFF{2.5pt}
	\def\RATIO{0.2}
    \def\HXA{2.5cm}
    \def\HYA{-0.5cm}
    \def\VXA{0.0cm}
    \def\VYA{2.55cm}
    \def\HXB{1.5cm}
    \def\HYB{0.85cm}
    \def\VXB{\VXA}
    \def\VYB{\VYA}
    \def\HXC{\HXA}
    \def\HYC{\HYA}
    \def\VXC{\HXB}
    \def\VYC{\HYB}
    \tikzstyle{square}=[line width=3pt, join=round, cap=round]
    \tikzstyle{side}=[line width=5pt, join=round, cap=round]
    \tikzstyle{arrow}=[line width=9pt, join=round, cap=round,->,rounded corners=3cm, cpurple]
    %\tikzstyle{double arrow}=[9pt colored by black and white]
    \tikzstyle{border}=[line width=2pt, join=round, cap=round]
    \begin{tikzpicture}[>=triangle 45]
		\coordinate (vh1) at (\HXA ,\HYA);
		\coordinate (vv1) at (\VXA, \VYA);
		\coordinate (vh2) at (\HXB, \HYB);
		\coordinate (vv2) at (\VXB, \VYB);
		\coordinate (vh3) at (\HXC, \HYC);
		\coordinate (vv3) at (\VXC, \VYC);
		    
    	\coordinate (p100) at (0,0);
    	\coordinate (p101) at ($(p100)+1*(vh1)$);
    	\coordinate (p102) at ($(p100)+2*(vh1)$);
    	\coordinate (p103) at ($(p100)+3*(vh1)$);
    	\coordinate (p110) at ($(p100)+1*(vv1)$);
    	\coordinate (p111) at ($(p110)+1*(vh1)$);
		\coordinate (p112) at ($(p110)+2*(vh1)$);
    	\coordinate (p113) at ($(p110)+3*(vh1)$);
    	\coordinate (p120) at ($(p100)+2*(vv1)$);
    	\coordinate (p121) at ($(p120)+1*(vh1)$);
    	\coordinate (p122) at ($(p120)+2*(vh1)$);
    	\coordinate (p123) at ($(p120)+3*(vh1)$);
    	\coordinate (p130) at ($(p100)+3*(vv1)$);
    	\coordinate (p131) at ($(p130)+1*(vh1)$);
    	\coordinate (p132) at ($(p130)+2*(vh1)$);
    	\coordinate (p133) at ($(p130)+3*(vh1)$);
    	
    	\coordinate (p200) at (p103);
    	\coordinate (p201) at ($(p200)+1*(vh2)$);
    	\coordinate (p202) at ($(p200)+2*(vh2)$);
    	\coordinate (p203) at ($(p200)+3*(vh2)$);
    	\coordinate (p210) at ($(p200)+1*(vv2)$);
    	\coordinate (p211) at ($(p210)+1*(vh2)$);
		\coordinate (p212) at ($(p210)+2*(vh2)$);
    	\coordinate (p213) at ($(p210)+3*(vh2)$);
    	\coordinate (p220) at ($(p200)+2*(vv2)$);
    	\coordinate (p221) at ($(p220)+1*(vh2)$);
    	\coordinate (p222) at ($(p220)+2*(vh2)$);
    	\coordinate (p223) at ($(p220)+3*(vh2)$);
    	\coordinate (p230) at ($(p200)+3*(vv2)$);
    	\coordinate (p231) at ($(p230)+1*(vh2)$);
    	\coordinate (p232) at ($(p230)+2*(vh2)$);
    	\coordinate (p233) at ($(p230)+3*(vh2)$);
    	
		\coordinate (p300) at (p130);
    	\coordinate (p301) at ($(p300)+1*(vh3)$);
    	\coordinate (p302) at ($(p300)+2*(vh3)$);
    	\coordinate (p303) at ($(p300)+3*(vh3)$);
    	\coordinate (p310) at ($(p300)+1*(vv3)$);
    	\coordinate (p311) at ($(p310)+1*(vh3)$);
		\coordinate (p312) at ($(p310)+2*(vh3)$);
    	\coordinate (p313) at ($(p310)+3*(vh3)$);
    	\coordinate (p320) at ($(p300)+2*(vv3)$);
    	\coordinate (p321) at ($(p320)+1*(vh3)$);
    	\coordinate (p322) at ($(p320)+2*(vh3)$);
    	\coordinate (p323) at ($(p320)+3*(vh3)$);
    	\coordinate (p330) at ($(p300)+3*(vv3)$);
    	\coordinate (p331) at ($(p330)+1*(vh3)$);
    	\coordinate (p332) at ($(p330)+2*(vh3)$);
    	\coordinate (p333) at ($(p330)+3*(vh3)$);  	  	
    	
		\filldraw[square, fill=\CAA] (p100) -- (p101) -- (p111) -- (p110);
		\filldraw[square, fill=\CAB] (p101) -- (p102) -- (p112) -- (p111);
		\filldraw[square, fill=\CAC] (p102) -- (p103) -- (p113) -- (p112);
		\filldraw[square, fill=\CAD] (p110) -- (p111) -- (p121) -- (p120);
		\filldraw[square, fill=\CAE] (p111) -- (p112) -- (p122) -- (p121);
		\filldraw[square, fill=\CAF] (p112) -- (p113) -- (p123) -- (p122);
		\filldraw[square, fill=\CAG] (p120) -- (p121) -- (p131) -- (p130);
		\filldraw[square, fill=\CAH] (p121) -- (p122) -- (p132) -- (p131);
		\filldraw[square, fill=\CAI] (p122) -- (p123) -- (p133) -- (p132);
		
		\filldraw[square, fill=\CBA] (p200) -- (p201) -- (p211) -- (p210);
		\filldraw[square, fill=\CBB] (p201) -- (p202) -- (p212) -- (p211);
		\filldraw[square, fill=\CBC] (p202) -- (p203) -- (p213) -- (p212);
		\filldraw[square, fill=\CBD] (p210) -- (p211) -- (p221) -- (p220);
		\filldraw[square, fill=\CBE] (p211) -- (p212) -- (p222) -- (p221);
		\filldraw[square, fill=\CBF] (p212) -- (p213) -- (p223) -- (p222);
		\filldraw[square, fill=\CBG] (p220) -- (p221) -- (p231) -- (p230);
		\filldraw[square, fill=\CBH] (p221) -- (p222) -- (p232) -- (p231);
		\filldraw[square, fill=\CBI] (p222) -- (p223) -- (p233) -- (p232);
		
		\filldraw[square, fill=\CCA] (p300) -- (p301) -- (p311) -- (p310);
		\filldraw[square, fill=\CCB] (p301) -- (p302) -- (p312) -- (p311);
		\filldraw[square, fill=\CCC] (p302) -- (p303) -- (p313) -- (p312);
		\filldraw[square, fill=\CCD] (p310) -- (p311) -- (p321) -- (p320);
		\filldraw[square, fill=\CCE] (p311) -- (p312) -- (p322) -- (p321);
		\filldraw[square, fill=\CCF] (p312) -- (p313) -- (p323) -- (p322);
		\filldraw[square, fill=\CCG] (p320) -- (p321) -- (p331) -- (p330);
		\filldraw[square, fill=\CCH] (p321) -- (p322) -- (p332) -- (p331);
		\filldraw[square, fill=\CCI] (p322) -- (p323) -- (p333) -- (p332);
	
    	\draw[side] (p100) -- (p200) -- (p203) -- (p233) -- (p330) -- (p130) -- (p100) -- (p200);
    	\draw[side] (p130) -- (p133) -- (p233);
    	\draw[side] (p103) -- (p133);
    	
    	\draw[square] (p101) -- (p131) -- (p331);
    	\draw[square] (p102) -- (p132) -- (p332);
    	\draw[square] (p201) -- (p231) -- (p310);
    	\draw[square] (p202) -- (p232) -- (p320);
    	\draw[square] (p110) -- (p113) -- (p213);
    	\draw[square] (p120) -- (p123) -- (p223);
	\end{tikzpicture}
	}    
	\showto{french}{\caption*{\label{fig:croixEnd} \LARGE{Croix}}}
	\showto{english}{\caption*{\label{fig:croixEnd} \LARGE{Cross}}}
\end{subfigure}
\end{figure}

\showto{french}{La première étape est de déterminer dans lequel de ces trois cas le cube se trouve. Ensuite, il faut placer le cube dans la bonne position avant de commencer l’algorithme en trois parties. Cet algorithme est en trois parties, car  seules les premières et troisièmes parties sont différentes et il ne s’agit en fait que d’un mouvement préliminaire et un mouvement final (qui est l’inverse du mouvement préliminaire). La grande partie de l’algorithme est strictement identique pour les trois cas.}

\showto{english}{The first step is to determine in which of these three cases the cube is. Next, place the cube in the correct position before starting the three-part algorithm. This algorithm is in three parts, because only the first and third parts are different and it is in fact only a preliminary movement and a final movement (which is the inverse of the preliminary movement). Much of the algorithm is strictly the same for all three cases.}

\showto{french}{\subsection{Cas Barre et L}}
\showto{english}{\subsection{Bar and L cases}}

\showto{french}{Nous commençons par regarder les cas pour les cas “Barre” et “L”. Le “point“ se résout par la combinaison des deux autres.}

\showto{english}{We start with solution for the "Bar" and "L" cases. The "point" is solved by the combination of these two}

\def\W{0.13}
\begin{tabular}[b]{r | c c p{\W\columnwidth} c c c p{\W\columnwidth}}
	& \showto{french}{Placement} \showto{english}{Starting point} & & \showto{french}{Mouvement préliminaire} \showto{english}{Preliminary move} & &	\showto{french}{Gâchette} \showto{english}{Trigger} & & \showto{french}{Mouvement final} \showto{english}{Final move}	\\ \hline
\showto{french}{\Large{Barre}} \showto{english}{\Large{Bar}} & \resizebox{\W\columnwidth}{!}
	{
	\def\CAA{cblue}
	\def\CAB{cblue}
	\def\CAC{cblue}
	\def\CAD{cblue}
	\def\CAE{cblue}
	\def\CAF{cblue}
	\def\CAG{cgray}
	\def\CAH{cyellow}
	\def\CAI{cgray}
	
	\def\CBA{cred}
	\def\CBB{cred}
	\def\CBC{cred}
	\def\CBD{cred}
	\def\CBE{cred}
	\def\CBF{cred}
	\def\CBG{cgray}
	\def\CBH{cgray}
	\def\CBI{cgray}
	
	\def\CCA{cgray}
	\def\CCB{cgray}
	\def\CCC{cgray}
	\def\CCD{cyellow}
	\def\CCE{cyellow}
	\def\CCF{cyellow}
	\def\CCG{cgray}
	\def\CCH{cgray}
	\def\CCI{cgray}	
		\def\OFFSET{1.4cm}
	\def\OFF{2.5pt}
	\def\RATIO{0.2}
    \def\HXA{2.5cm}
    \def\HYA{-0.5cm}
    \def\VXA{0.0cm}
    \def\VYA{2.55cm}
    \def\HXB{1.5cm}
    \def\HYB{0.85cm}
    \def\VXB{\VXA}
    \def\VYB{\VYA}
    \def\HXC{\HXA}
    \def\HYC{\HYA}
    \def\VXC{\HXB}
    \def\VYC{\HYB}
    \tikzstyle{square}=[line width=3pt, join=round, cap=round]
    \tikzstyle{side}=[line width=5pt, join=round, cap=round]
    \tikzstyle{arrow}=[line width=9pt, join=round, cap=round,->,rounded corners=3cm, cpurple]
    %\tikzstyle{double arrow}=[9pt colored by black and white]
    \tikzstyle{border}=[line width=2pt, join=round, cap=round]
    \begin{tikzpicture}[>=triangle 45]
		\coordinate (vh1) at (\HXA ,\HYA);
		\coordinate (vv1) at (\VXA, \VYA);
		\coordinate (vh2) at (\HXB, \HYB);
		\coordinate (vv2) at (\VXB, \VYB);
		\coordinate (vh3) at (\HXC, \HYC);
		\coordinate (vv3) at (\VXC, \VYC);
		    
    	\coordinate (p100) at (0,0);
    	\coordinate (p101) at ($(p100)+1*(vh1)$);
    	\coordinate (p102) at ($(p100)+2*(vh1)$);
    	\coordinate (p103) at ($(p100)+3*(vh1)$);
    	\coordinate (p110) at ($(p100)+1*(vv1)$);
    	\coordinate (p111) at ($(p110)+1*(vh1)$);
		\coordinate (p112) at ($(p110)+2*(vh1)$);
    	\coordinate (p113) at ($(p110)+3*(vh1)$);
    	\coordinate (p120) at ($(p100)+2*(vv1)$);
    	\coordinate (p121) at ($(p120)+1*(vh1)$);
    	\coordinate (p122) at ($(p120)+2*(vh1)$);
    	\coordinate (p123) at ($(p120)+3*(vh1)$);
    	\coordinate (p130) at ($(p100)+3*(vv1)$);
    	\coordinate (p131) at ($(p130)+1*(vh1)$);
    	\coordinate (p132) at ($(p130)+2*(vh1)$);
    	\coordinate (p133) at ($(p130)+3*(vh1)$);
    	
    	\coordinate (p200) at (p103);
    	\coordinate (p201) at ($(p200)+1*(vh2)$);
    	\coordinate (p202) at ($(p200)+2*(vh2)$);
    	\coordinate (p203) at ($(p200)+3*(vh2)$);
    	\coordinate (p210) at ($(p200)+1*(vv2)$);
    	\coordinate (p211) at ($(p210)+1*(vh2)$);
		\coordinate (p212) at ($(p210)+2*(vh2)$);
    	\coordinate (p213) at ($(p210)+3*(vh2)$);
    	\coordinate (p220) at ($(p200)+2*(vv2)$);
    	\coordinate (p221) at ($(p220)+1*(vh2)$);
    	\coordinate (p222) at ($(p220)+2*(vh2)$);
    	\coordinate (p223) at ($(p220)+3*(vh2)$);
    	\coordinate (p230) at ($(p200)+3*(vv2)$);
    	\coordinate (p231) at ($(p230)+1*(vh2)$);
    	\coordinate (p232) at ($(p230)+2*(vh2)$);
    	\coordinate (p233) at ($(p230)+3*(vh2)$);
    	
		\coordinate (p300) at (p130);
    	\coordinate (p301) at ($(p300)+1*(vh3)$);
    	\coordinate (p302) at ($(p300)+2*(vh3)$);
    	\coordinate (p303) at ($(p300)+3*(vh3)$);
    	\coordinate (p310) at ($(p300)+1*(vv3)$);
    	\coordinate (p311) at ($(p310)+1*(vh3)$);
		\coordinate (p312) at ($(p310)+2*(vh3)$);
    	\coordinate (p313) at ($(p310)+3*(vh3)$);
    	\coordinate (p320) at ($(p300)+2*(vv3)$);
    	\coordinate (p321) at ($(p320)+1*(vh3)$);
    	\coordinate (p322) at ($(p320)+2*(vh3)$);
    	\coordinate (p323) at ($(p320)+3*(vh3)$);
    	\coordinate (p330) at ($(p300)+3*(vv3)$);
    	\coordinate (p331) at ($(p330)+1*(vh3)$);
    	\coordinate (p332) at ($(p330)+2*(vh3)$);
    	\coordinate (p333) at ($(p330)+3*(vh3)$);  	  	
    	
		\filldraw[square, fill=\CAA] (p100) -- (p101) -- (p111) -- (p110);
		\filldraw[square, fill=\CAB] (p101) -- (p102) -- (p112) -- (p111);
		\filldraw[square, fill=\CAC] (p102) -- (p103) -- (p113) -- (p112);
		\filldraw[square, fill=\CAD] (p110) -- (p111) -- (p121) -- (p120);
		\filldraw[square, fill=\CAE] (p111) -- (p112) -- (p122) -- (p121);
		\filldraw[square, fill=\CAF] (p112) -- (p113) -- (p123) -- (p122);
		\filldraw[square, fill=\CAG] (p120) -- (p121) -- (p131) -- (p130);
		\filldraw[square, fill=\CAH] (p121) -- (p122) -- (p132) -- (p131);
		\filldraw[square, fill=\CAI] (p122) -- (p123) -- (p133) -- (p132);
		
		\filldraw[square, fill=\CBA] (p200) -- (p201) -- (p211) -- (p210);
		\filldraw[square, fill=\CBB] (p201) -- (p202) -- (p212) -- (p211);
		\filldraw[square, fill=\CBC] (p202) -- (p203) -- (p213) -- (p212);
		\filldraw[square, fill=\CBD] (p210) -- (p211) -- (p221) -- (p220);
		\filldraw[square, fill=\CBE] (p211) -- (p212) -- (p222) -- (p221);
		\filldraw[square, fill=\CBF] (p212) -- (p213) -- (p223) -- (p222);
		\filldraw[square, fill=\CBG] (p220) -- (p221) -- (p231) -- (p230);
		\filldraw[square, fill=\CBH] (p221) -- (p222) -- (p232) -- (p231);
		\filldraw[square, fill=\CBI] (p222) -- (p223) -- (p233) -- (p232);
		
		\filldraw[square, fill=\CCA] (p300) -- (p301) -- (p311) -- (p310);
		\filldraw[square, fill=\CCB] (p301) -- (p302) -- (p312) -- (p311);
		\filldraw[square, fill=\CCC] (p302) -- (p303) -- (p313) -- (p312);
		\filldraw[square, fill=\CCD] (p310) -- (p311) -- (p321) -- (p320);
		\filldraw[square, fill=\CCE] (p311) -- (p312) -- (p322) -- (p321);
		\filldraw[square, fill=\CCF] (p312) -- (p313) -- (p323) -- (p322);
		\filldraw[square, fill=\CCG] (p320) -- (p321) -- (p331) -- (p330);
		\filldraw[square, fill=\CCH] (p321) -- (p322) -- (p332) -- (p331);
		\filldraw[square, fill=\CCI] (p322) -- (p323) -- (p333) -- (p332);
	
    	\draw[side] (p100) -- (p200) -- (p203) -- (p233) -- (p330) -- (p130) -- (p100) -- (p200);
    	\draw[side] (p130) -- (p133) -- (p233);
    	\draw[side] (p103) -- (p133);
    	
    	\draw[square] (p101) -- (p131) -- (p331);
    	\draw[square] (p102) -- (p132) -- (p332);
    	\draw[square] (p201) -- (p231) -- (p310);
    	\draw[square] (p202) -- (p232) -- (p320);
    	\draw[square] (p110) -- (p113) -- (p213);
    	\draw[square] (p120) -- (p123) -- (p223);
	\end{tikzpicture}
	}
& \Large{$F$} & 	\resizebox{\W\columnwidth}{!}
	{
	\def\CAA{cblue}
	\def\CAB{cblue}
	\def\CAC{cgray}
	\def\CAD{cblue}
	\def\CAE{cblue}
	\def\CAF{cyellow}
	\def\CAG{cblue}
	\def\CAH{cblue}
	\def\CAI{cgray}
	
	\def\CBA{cgray}
	\def\CBB{cred}
	\def\CBC{cred}
	\def\CBD{cgray}
	\def\CBE{cred}
	\def\CBF{cred}
	\def\CBG{cgray}
	\def\CBH{cgray}
	\def\CBI{cgray}
	
	\def\CCA{corange}
	\def\CCB{corange}
	\def\CCC{cgray}
	\def\CCD{cyellow}
	\def\CCE{cyellow}
	\def\CCF{cyellow}
	\def\CCG{cgray}
	\def\CCH{cgray}
	\def\CCI{cgray}
		\def\OFFSET{1.4cm}
	\def\OFF{2.5pt}
	\def\RATIO{0.2}
    \def\HXA{2.5cm}
    \def\HYA{-0.5cm}
    \def\VXA{0.0cm}
    \def\VYA{2.55cm}
    \def\HXB{1.5cm}
    \def\HYB{0.85cm}
    \def\VXB{\VXA}
    \def\VYB{\VYA}
    \def\HXC{\HXA}
    \def\HYC{\HYA}
    \def\VXC{\HXB}
    \def\VYC{\HYB}
    \tikzstyle{square}=[line width=3pt, join=round, cap=round]
    \tikzstyle{side}=[line width=5pt, join=round, cap=round]
    \tikzstyle{arrow}=[line width=9pt, join=round, cap=round,->,rounded corners=3cm, cpurple]
    %\tikzstyle{double arrow}=[9pt colored by black and white]
    \tikzstyle{border}=[line width=2pt, join=round, cap=round]
    \begin{tikzpicture}[>=triangle 45]
		\coordinate (vh1) at (\HXA ,\HYA);
		\coordinate (vv1) at (\VXA, \VYA);
		\coordinate (vh2) at (\HXB, \HYB);
		\coordinate (vv2) at (\VXB, \VYB);
		\coordinate (vh3) at (\HXC, \HYC);
		\coordinate (vv3) at (\VXC, \VYC);
		    
    	\coordinate (p100) at (0,0);
    	\coordinate (p101) at ($(p100)+1*(vh1)$);
    	\coordinate (p102) at ($(p100)+2*(vh1)$);
    	\coordinate (p103) at ($(p100)+3*(vh1)$);
    	\coordinate (p110) at ($(p100)+1*(vv1)$);
    	\coordinate (p111) at ($(p110)+1*(vh1)$);
		\coordinate (p112) at ($(p110)+2*(vh1)$);
    	\coordinate (p113) at ($(p110)+3*(vh1)$);
    	\coordinate (p120) at ($(p100)+2*(vv1)$);
    	\coordinate (p121) at ($(p120)+1*(vh1)$);
    	\coordinate (p122) at ($(p120)+2*(vh1)$);
    	\coordinate (p123) at ($(p120)+3*(vh1)$);
    	\coordinate (p130) at ($(p100)+3*(vv1)$);
    	\coordinate (p131) at ($(p130)+1*(vh1)$);
    	\coordinate (p132) at ($(p130)+2*(vh1)$);
    	\coordinate (p133) at ($(p130)+3*(vh1)$);
    	
    	\coordinate (p200) at (p103);
    	\coordinate (p201) at ($(p200)+1*(vh2)$);
    	\coordinate (p202) at ($(p200)+2*(vh2)$);
    	\coordinate (p203) at ($(p200)+3*(vh2)$);
    	\coordinate (p210) at ($(p200)+1*(vv2)$);
    	\coordinate (p211) at ($(p210)+1*(vh2)$);
		\coordinate (p212) at ($(p210)+2*(vh2)$);
    	\coordinate (p213) at ($(p210)+3*(vh2)$);
    	\coordinate (p220) at ($(p200)+2*(vv2)$);
    	\coordinate (p221) at ($(p220)+1*(vh2)$);
    	\coordinate (p222) at ($(p220)+2*(vh2)$);
    	\coordinate (p223) at ($(p220)+3*(vh2)$);
    	\coordinate (p230) at ($(p200)+3*(vv2)$);
    	\coordinate (p231) at ($(p230)+1*(vh2)$);
    	\coordinate (p232) at ($(p230)+2*(vh2)$);
    	\coordinate (p233) at ($(p230)+3*(vh2)$);
    	
		\coordinate (p300) at (p130);
    	\coordinate (p301) at ($(p300)+1*(vh3)$);
    	\coordinate (p302) at ($(p300)+2*(vh3)$);
    	\coordinate (p303) at ($(p300)+3*(vh3)$);
    	\coordinate (p310) at ($(p300)+1*(vv3)$);
    	\coordinate (p311) at ($(p310)+1*(vh3)$);
		\coordinate (p312) at ($(p310)+2*(vh3)$);
    	\coordinate (p313) at ($(p310)+3*(vh3)$);
    	\coordinate (p320) at ($(p300)+2*(vv3)$);
    	\coordinate (p321) at ($(p320)+1*(vh3)$);
    	\coordinate (p322) at ($(p320)+2*(vh3)$);
    	\coordinate (p323) at ($(p320)+3*(vh3)$);
    	\coordinate (p330) at ($(p300)+3*(vv3)$);
    	\coordinate (p331) at ($(p330)+1*(vh3)$);
    	\coordinate (p332) at ($(p330)+2*(vh3)$);
    	\coordinate (p333) at ($(p330)+3*(vh3)$);  	  	
    	
		\filldraw[square, fill=\CAA] (p100) -- (p101) -- (p111) -- (p110);
		\filldraw[square, fill=\CAB] (p101) -- (p102) -- (p112) -- (p111);
		\filldraw[square, fill=\CAC] (p102) -- (p103) -- (p113) -- (p112);
		\filldraw[square, fill=\CAD] (p110) -- (p111) -- (p121) -- (p120);
		\filldraw[square, fill=\CAE] (p111) -- (p112) -- (p122) -- (p121);
		\filldraw[square, fill=\CAF] (p112) -- (p113) -- (p123) -- (p122);
		\filldraw[square, fill=\CAG] (p120) -- (p121) -- (p131) -- (p130);
		\filldraw[square, fill=\CAH] (p121) -- (p122) -- (p132) -- (p131);
		\filldraw[square, fill=\CAI] (p122) -- (p123) -- (p133) -- (p132);
		
		\filldraw[square, fill=\CBA] (p200) -- (p201) -- (p211) -- (p210);
		\filldraw[square, fill=\CBB] (p201) -- (p202) -- (p212) -- (p211);
		\filldraw[square, fill=\CBC] (p202) -- (p203) -- (p213) -- (p212);
		\filldraw[square, fill=\CBD] (p210) -- (p211) -- (p221) -- (p220);
		\filldraw[square, fill=\CBE] (p211) -- (p212) -- (p222) -- (p221);
		\filldraw[square, fill=\CBF] (p212) -- (p213) -- (p223) -- (p222);
		\filldraw[square, fill=\CBG] (p220) -- (p221) -- (p231) -- (p230);
		\filldraw[square, fill=\CBH] (p221) -- (p222) -- (p232) -- (p231);
		\filldraw[square, fill=\CBI] (p222) -- (p223) -- (p233) -- (p232);
		
		\filldraw[square, fill=\CCA] (p300) -- (p301) -- (p311) -- (p310);
		\filldraw[square, fill=\CCB] (p301) -- (p302) -- (p312) -- (p311);
		\filldraw[square, fill=\CCC] (p302) -- (p303) -- (p313) -- (p312);
		\filldraw[square, fill=\CCD] (p310) -- (p311) -- (p321) -- (p320);
		\filldraw[square, fill=\CCE] (p311) -- (p312) -- (p322) -- (p321);
		\filldraw[square, fill=\CCF] (p312) -- (p313) -- (p323) -- (p322);
		\filldraw[square, fill=\CCG] (p320) -- (p321) -- (p331) -- (p330);
		\filldraw[square, fill=\CCH] (p321) -- (p322) -- (p332) -- (p331);
		\filldraw[square, fill=\CCI] (p322) -- (p323) -- (p333) -- (p332);
	
    	\draw[side] (p100) -- (p200) -- (p203) -- (p233) -- (p330) -- (p130) -- (p100) -- (p200);
    	\draw[side] (p130) -- (p133) -- (p233);
    	\draw[side] (p103) -- (p133);
    	
    	\draw[square] (p101) -- (p131) -- (p331);
    	\draw[square] (p102) -- (p132) -- (p332);
    	\draw[square] (p201) -- (p231) -- (p310);
    	\draw[square] (p202) -- (p232) -- (p320);
    	\draw[square] (p110) -- (p113) -- (p213);
    	\draw[square] (p120) -- (p123) -- (p223);
	\end{tikzpicture}
	}    
& \Large{$R$ $U$ $R'$ $U'$} & 	\resizebox{\W\columnwidth}{!}
	{
	\def\CAA{cblue}
	\def\CAB{cblue}
	\def\CAC{cgray}
	\def\CAD{cblue}
	\def\CAE{cblue}
	\def\CAF{cgray}
	\def\CAG{cblue}
	\def\CAH{cblue}
	\def\CAI{cgray}
	
	\def\CBA{cgray}
	\def\CBB{cred}
	\def\CBC{cred}
	\def\CBD{cyellow}
	\def\CBE{cred}
	\def\CBF{cred}
	\def\CBG{cgray}
	\def\CBH{cgray}
	\def\CBI{cgray}
	
	\def\CCA{corange}
	\def\CCB{corange}
	\def\CCC{cgray}
	\def\CCD{cyellow}
	\def\CCE{cyellow}
	\def\CCF{cyellow}
	\def\CCG{cgray}
	\def\CCH{cyellow}
	\def\CCI{cgray}
		\def\OFFSET{1.4cm}
	\def\OFF{2.5pt}
	\def\RATIO{0.2}
    \def\HXA{2.5cm}
    \def\HYA{-0.5cm}
    \def\VXA{0.0cm}
    \def\VYA{2.55cm}
    \def\HXB{1.5cm}
    \def\HYB{0.85cm}
    \def\VXB{\VXA}
    \def\VYB{\VYA}
    \def\HXC{\HXA}
    \def\HYC{\HYA}
    \def\VXC{\HXB}
    \def\VYC{\HYB}
    \tikzstyle{square}=[line width=3pt, join=round, cap=round]
    \tikzstyle{side}=[line width=5pt, join=round, cap=round]
    \tikzstyle{arrow}=[line width=9pt, join=round, cap=round,->,rounded corners=3cm, cpurple]
    %\tikzstyle{double arrow}=[9pt colored by black and white]
    \tikzstyle{border}=[line width=2pt, join=round, cap=round]
    \begin{tikzpicture}[>=triangle 45]
		\coordinate (vh1) at (\HXA ,\HYA);
		\coordinate (vv1) at (\VXA, \VYA);
		\coordinate (vh2) at (\HXB, \HYB);
		\coordinate (vv2) at (\VXB, \VYB);
		\coordinate (vh3) at (\HXC, \HYC);
		\coordinate (vv3) at (\VXC, \VYC);
		    
    	\coordinate (p100) at (0,0);
    	\coordinate (p101) at ($(p100)+1*(vh1)$);
    	\coordinate (p102) at ($(p100)+2*(vh1)$);
    	\coordinate (p103) at ($(p100)+3*(vh1)$);
    	\coordinate (p110) at ($(p100)+1*(vv1)$);
    	\coordinate (p111) at ($(p110)+1*(vh1)$);
		\coordinate (p112) at ($(p110)+2*(vh1)$);
    	\coordinate (p113) at ($(p110)+3*(vh1)$);
    	\coordinate (p120) at ($(p100)+2*(vv1)$);
    	\coordinate (p121) at ($(p120)+1*(vh1)$);
    	\coordinate (p122) at ($(p120)+2*(vh1)$);
    	\coordinate (p123) at ($(p120)+3*(vh1)$);
    	\coordinate (p130) at ($(p100)+3*(vv1)$);
    	\coordinate (p131) at ($(p130)+1*(vh1)$);
    	\coordinate (p132) at ($(p130)+2*(vh1)$);
    	\coordinate (p133) at ($(p130)+3*(vh1)$);
    	
    	\coordinate (p200) at (p103);
    	\coordinate (p201) at ($(p200)+1*(vh2)$);
    	\coordinate (p202) at ($(p200)+2*(vh2)$);
    	\coordinate (p203) at ($(p200)+3*(vh2)$);
    	\coordinate (p210) at ($(p200)+1*(vv2)$);
    	\coordinate (p211) at ($(p210)+1*(vh2)$);
		\coordinate (p212) at ($(p210)+2*(vh2)$);
    	\coordinate (p213) at ($(p210)+3*(vh2)$);
    	\coordinate (p220) at ($(p200)+2*(vv2)$);
    	\coordinate (p221) at ($(p220)+1*(vh2)$);
    	\coordinate (p222) at ($(p220)+2*(vh2)$);
    	\coordinate (p223) at ($(p220)+3*(vh2)$);
    	\coordinate (p230) at ($(p200)+3*(vv2)$);
    	\coordinate (p231) at ($(p230)+1*(vh2)$);
    	\coordinate (p232) at ($(p230)+2*(vh2)$);
    	\coordinate (p233) at ($(p230)+3*(vh2)$);
    	
		\coordinate (p300) at (p130);
    	\coordinate (p301) at ($(p300)+1*(vh3)$);
    	\coordinate (p302) at ($(p300)+2*(vh3)$);
    	\coordinate (p303) at ($(p300)+3*(vh3)$);
    	\coordinate (p310) at ($(p300)+1*(vv3)$);
    	\coordinate (p311) at ($(p310)+1*(vh3)$);
		\coordinate (p312) at ($(p310)+2*(vh3)$);
    	\coordinate (p313) at ($(p310)+3*(vh3)$);
    	\coordinate (p320) at ($(p300)+2*(vv3)$);
    	\coordinate (p321) at ($(p320)+1*(vh3)$);
    	\coordinate (p322) at ($(p320)+2*(vh3)$);
    	\coordinate (p323) at ($(p320)+3*(vh3)$);
    	\coordinate (p330) at ($(p300)+3*(vv3)$);
    	\coordinate (p331) at ($(p330)+1*(vh3)$);
    	\coordinate (p332) at ($(p330)+2*(vh3)$);
    	\coordinate (p333) at ($(p330)+3*(vh3)$);  	  	
    	
		\filldraw[square, fill=\CAA] (p100) -- (p101) -- (p111) -- (p110);
		\filldraw[square, fill=\CAB] (p101) -- (p102) -- (p112) -- (p111);
		\filldraw[square, fill=\CAC] (p102) -- (p103) -- (p113) -- (p112);
		\filldraw[square, fill=\CAD] (p110) -- (p111) -- (p121) -- (p120);
		\filldraw[square, fill=\CAE] (p111) -- (p112) -- (p122) -- (p121);
		\filldraw[square, fill=\CAF] (p112) -- (p113) -- (p123) -- (p122);
		\filldraw[square, fill=\CAG] (p120) -- (p121) -- (p131) -- (p130);
		\filldraw[square, fill=\CAH] (p121) -- (p122) -- (p132) -- (p131);
		\filldraw[square, fill=\CAI] (p122) -- (p123) -- (p133) -- (p132);
		
		\filldraw[square, fill=\CBA] (p200) -- (p201) -- (p211) -- (p210);
		\filldraw[square, fill=\CBB] (p201) -- (p202) -- (p212) -- (p211);
		\filldraw[square, fill=\CBC] (p202) -- (p203) -- (p213) -- (p212);
		\filldraw[square, fill=\CBD] (p210) -- (p211) -- (p221) -- (p220);
		\filldraw[square, fill=\CBE] (p211) -- (p212) -- (p222) -- (p221);
		\filldraw[square, fill=\CBF] (p212) -- (p213) -- (p223) -- (p222);
		\filldraw[square, fill=\CBG] (p220) -- (p221) -- (p231) -- (p230);
		\filldraw[square, fill=\CBH] (p221) -- (p222) -- (p232) -- (p231);
		\filldraw[square, fill=\CBI] (p222) -- (p223) -- (p233) -- (p232);
		
		\filldraw[square, fill=\CCA] (p300) -- (p301) -- (p311) -- (p310);
		\filldraw[square, fill=\CCB] (p301) -- (p302) -- (p312) -- (p311);
		\filldraw[square, fill=\CCC] (p302) -- (p303) -- (p313) -- (p312);
		\filldraw[square, fill=\CCD] (p310) -- (p311) -- (p321) -- (p320);
		\filldraw[square, fill=\CCE] (p311) -- (p312) -- (p322) -- (p321);
		\filldraw[square, fill=\CCF] (p312) -- (p313) -- (p323) -- (p322);
		\filldraw[square, fill=\CCG] (p320) -- (p321) -- (p331) -- (p330);
		\filldraw[square, fill=\CCH] (p321) -- (p322) -- (p332) -- (p331);
		\filldraw[square, fill=\CCI] (p322) -- (p323) -- (p333) -- (p332);
	
    	\draw[side] (p100) -- (p200) -- (p203) -- (p233) -- (p330) -- (p130) -- (p100) -- (p200);
    	\draw[side] (p130) -- (p133) -- (p233);
    	\draw[side] (p103) -- (p133);
    	
    	\draw[square] (p101) -- (p131) -- (p331);
    	\draw[square] (p102) -- (p132) -- (p332);
    	\draw[square] (p201) -- (p231) -- (p310);
    	\draw[square] (p202) -- (p232) -- (p320);
    	\draw[square] (p110) -- (p113) -- (p213);
    	\draw[square] (p120) -- (p123) -- (p223);
	\end{tikzpicture}
	}
& \Large{$F'$} &	\resizebox{\W\columnwidth}{!}
	{
	\def\CAA{cblue}
	\def\CAB{cblue}
	\def\CAC{cblue}
	\def\CAD{cblue}
	\def\CAE{cblue}
	\def\CAF{cblue}
	\def\CAG{cgray}
	\def\CAH{cgray}
	\def\CAI{cgray}
	
	\def\CBA{cred}
	\def\CBB{cred}
	\def\CBC{cred}
	\def\CBD{cred}
	\def\CBE{cred}
	\def\CBF{cred}
	\def\CBG{cgray}
	\def\CBH{cgray}
	\def\CBI{cgray}
	
	\def\CCA{cgray}
	\def\CCB{cyellow}
	\def\CCC{cgray}
	\def\CCD{cyellow}
	\def\CCE{cyellow}
	\def\CCF{cyellow}
	\def\CCG{cgray}
	\def\CCH{cyellow}
	\def\CCI{cgray}
		\def\OFFSET{1.4cm}
	\def\OFF{2.5pt}
	\def\RATIO{0.2}
    \def\HXA{2.5cm}
    \def\HYA{-0.5cm}
    \def\VXA{0.0cm}
    \def\VYA{2.55cm}
    \def\HXB{1.5cm}
    \def\HYB{0.85cm}
    \def\VXB{\VXA}
    \def\VYB{\VYA}
    \def\HXC{\HXA}
    \def\HYC{\HYA}
    \def\VXC{\HXB}
    \def\VYC{\HYB}
    \tikzstyle{square}=[line width=3pt, join=round, cap=round]
    \tikzstyle{side}=[line width=5pt, join=round, cap=round]
    \tikzstyle{arrow}=[line width=9pt, join=round, cap=round,->,rounded corners=3cm, cpurple]
    %\tikzstyle{double arrow}=[9pt colored by black and white]
    \tikzstyle{border}=[line width=2pt, join=round, cap=round]
    \begin{tikzpicture}[>=triangle 45]
		\coordinate (vh1) at (\HXA ,\HYA);
		\coordinate (vv1) at (\VXA, \VYA);
		\coordinate (vh2) at (\HXB, \HYB);
		\coordinate (vv2) at (\VXB, \VYB);
		\coordinate (vh3) at (\HXC, \HYC);
		\coordinate (vv3) at (\VXC, \VYC);
		    
    	\coordinate (p100) at (0,0);
    	\coordinate (p101) at ($(p100)+1*(vh1)$);
    	\coordinate (p102) at ($(p100)+2*(vh1)$);
    	\coordinate (p103) at ($(p100)+3*(vh1)$);
    	\coordinate (p110) at ($(p100)+1*(vv1)$);
    	\coordinate (p111) at ($(p110)+1*(vh1)$);
		\coordinate (p112) at ($(p110)+2*(vh1)$);
    	\coordinate (p113) at ($(p110)+3*(vh1)$);
    	\coordinate (p120) at ($(p100)+2*(vv1)$);
    	\coordinate (p121) at ($(p120)+1*(vh1)$);
    	\coordinate (p122) at ($(p120)+2*(vh1)$);
    	\coordinate (p123) at ($(p120)+3*(vh1)$);
    	\coordinate (p130) at ($(p100)+3*(vv1)$);
    	\coordinate (p131) at ($(p130)+1*(vh1)$);
    	\coordinate (p132) at ($(p130)+2*(vh1)$);
    	\coordinate (p133) at ($(p130)+3*(vh1)$);
    	
    	\coordinate (p200) at (p103);
    	\coordinate (p201) at ($(p200)+1*(vh2)$);
    	\coordinate (p202) at ($(p200)+2*(vh2)$);
    	\coordinate (p203) at ($(p200)+3*(vh2)$);
    	\coordinate (p210) at ($(p200)+1*(vv2)$);
    	\coordinate (p211) at ($(p210)+1*(vh2)$);
		\coordinate (p212) at ($(p210)+2*(vh2)$);
    	\coordinate (p213) at ($(p210)+3*(vh2)$);
    	\coordinate (p220) at ($(p200)+2*(vv2)$);
    	\coordinate (p221) at ($(p220)+1*(vh2)$);
    	\coordinate (p222) at ($(p220)+2*(vh2)$);
    	\coordinate (p223) at ($(p220)+3*(vh2)$);
    	\coordinate (p230) at ($(p200)+3*(vv2)$);
    	\coordinate (p231) at ($(p230)+1*(vh2)$);
    	\coordinate (p232) at ($(p230)+2*(vh2)$);
    	\coordinate (p233) at ($(p230)+3*(vh2)$);
    	
		\coordinate (p300) at (p130);
    	\coordinate (p301) at ($(p300)+1*(vh3)$);
    	\coordinate (p302) at ($(p300)+2*(vh3)$);
    	\coordinate (p303) at ($(p300)+3*(vh3)$);
    	\coordinate (p310) at ($(p300)+1*(vv3)$);
    	\coordinate (p311) at ($(p310)+1*(vh3)$);
		\coordinate (p312) at ($(p310)+2*(vh3)$);
    	\coordinate (p313) at ($(p310)+3*(vh3)$);
    	\coordinate (p320) at ($(p300)+2*(vv3)$);
    	\coordinate (p321) at ($(p320)+1*(vh3)$);
    	\coordinate (p322) at ($(p320)+2*(vh3)$);
    	\coordinate (p323) at ($(p320)+3*(vh3)$);
    	\coordinate (p330) at ($(p300)+3*(vv3)$);
    	\coordinate (p331) at ($(p330)+1*(vh3)$);
    	\coordinate (p332) at ($(p330)+2*(vh3)$);
    	\coordinate (p333) at ($(p330)+3*(vh3)$);  	  	
    	
		\filldraw[square, fill=\CAA] (p100) -- (p101) -- (p111) -- (p110);
		\filldraw[square, fill=\CAB] (p101) -- (p102) -- (p112) -- (p111);
		\filldraw[square, fill=\CAC] (p102) -- (p103) -- (p113) -- (p112);
		\filldraw[square, fill=\CAD] (p110) -- (p111) -- (p121) -- (p120);
		\filldraw[square, fill=\CAE] (p111) -- (p112) -- (p122) -- (p121);
		\filldraw[square, fill=\CAF] (p112) -- (p113) -- (p123) -- (p122);
		\filldraw[square, fill=\CAG] (p120) -- (p121) -- (p131) -- (p130);
		\filldraw[square, fill=\CAH] (p121) -- (p122) -- (p132) -- (p131);
		\filldraw[square, fill=\CAI] (p122) -- (p123) -- (p133) -- (p132);
		
		\filldraw[square, fill=\CBA] (p200) -- (p201) -- (p211) -- (p210);
		\filldraw[square, fill=\CBB] (p201) -- (p202) -- (p212) -- (p211);
		\filldraw[square, fill=\CBC] (p202) -- (p203) -- (p213) -- (p212);
		\filldraw[square, fill=\CBD] (p210) -- (p211) -- (p221) -- (p220);
		\filldraw[square, fill=\CBE] (p211) -- (p212) -- (p222) -- (p221);
		\filldraw[square, fill=\CBF] (p212) -- (p213) -- (p223) -- (p222);
		\filldraw[square, fill=\CBG] (p220) -- (p221) -- (p231) -- (p230);
		\filldraw[square, fill=\CBH] (p221) -- (p222) -- (p232) -- (p231);
		\filldraw[square, fill=\CBI] (p222) -- (p223) -- (p233) -- (p232);
		
		\filldraw[square, fill=\CCA] (p300) -- (p301) -- (p311) -- (p310);
		\filldraw[square, fill=\CCB] (p301) -- (p302) -- (p312) -- (p311);
		\filldraw[square, fill=\CCC] (p302) -- (p303) -- (p313) -- (p312);
		\filldraw[square, fill=\CCD] (p310) -- (p311) -- (p321) -- (p320);
		\filldraw[square, fill=\CCE] (p311) -- (p312) -- (p322) -- (p321);
		\filldraw[square, fill=\CCF] (p312) -- (p313) -- (p323) -- (p322);
		\filldraw[square, fill=\CCG] (p320) -- (p321) -- (p331) -- (p330);
		\filldraw[square, fill=\CCH] (p321) -- (p322) -- (p332) -- (p331);
		\filldraw[square, fill=\CCI] (p322) -- (p323) -- (p333) -- (p332);
	
    	\draw[side] (p100) -- (p200) -- (p203) -- (p233) -- (p330) -- (p130) -- (p100) -- (p200);
    	\draw[side] (p130) -- (p133) -- (p233);
    	\draw[side] (p103) -- (p133);
    	
    	\draw[square] (p101) -- (p131) -- (p331);
    	\draw[square] (p102) -- (p132) -- (p332);
    	\draw[square] (p201) -- (p231) -- (p310);
    	\draw[square] (p202) -- (p232) -- (p320);
    	\draw[square] (p110) -- (p113) -- (p213);
    	\draw[square] (p120) -- (p123) -- (p223);
	\end{tikzpicture}
	}
\\
\Large{L} &	\resizebox{\W\columnwidth}{!}
	{
	\def\CAA{cblue}
	\def\CAB{cblue}
	\def\CAC{cblue}
	\def\CAD{cblue}
	\def\CAE{cblue}
	\def\CAF{cblue}
	\def\CAG{cgray}
	\def\CAH{cyellow}
	\def\CAI{cgray}
	
	\def\CBA{cred}
	\def\CBB{cred}
	\def\CBC{cred}
	\def\CBD{cred}
	\def\CBE{cred}
	\def\CBF{cred}
	\def\CBG{cgray}
	\def\CBH{cgray}
	\def\CBI{cgray}
	
	\def\CCA{cgray}
	\def\CCB{cyellow}
	\def\CCC{cgray}
	\def\CCD{cgray}
	\def\CCE{cyellow}
	\def\CCF{cyellow}
	\def\CCG{cgray}
	\def\CCH{cgray}
	\def\CCI{cgray}
		\def\OFFSET{1.4cm}
	\def\OFF{2.5pt}
	\def\RATIO{0.2}
    \def\HXA{2.5cm}
    \def\HYA{-0.5cm}
    \def\VXA{0.0cm}
    \def\VYA{2.55cm}
    \def\HXB{1.5cm}
    \def\HYB{0.85cm}
    \def\VXB{\VXA}
    \def\VYB{\VYA}
    \def\HXC{\HXA}
    \def\HYC{\HYA}
    \def\VXC{\HXB}
    \def\VYC{\HYB}
    \tikzstyle{square}=[line width=3pt, join=round, cap=round]
    \tikzstyle{side}=[line width=5pt, join=round, cap=round]
    \tikzstyle{arrow}=[line width=9pt, join=round, cap=round,->,rounded corners=3cm, cpurple]
    %\tikzstyle{double arrow}=[9pt colored by black and white]
    \tikzstyle{border}=[line width=2pt, join=round, cap=round]
    \begin{tikzpicture}[>=triangle 45]
		\coordinate (vh1) at (\HXA ,\HYA);
		\coordinate (vv1) at (\VXA, \VYA);
		\coordinate (vh2) at (\HXB, \HYB);
		\coordinate (vv2) at (\VXB, \VYB);
		\coordinate (vh3) at (\HXC, \HYC);
		\coordinate (vv3) at (\VXC, \VYC);
		    
    	\coordinate (p100) at (0,0);
    	\coordinate (p101) at ($(p100)+1*(vh1)$);
    	\coordinate (p102) at ($(p100)+2*(vh1)$);
    	\coordinate (p103) at ($(p100)+3*(vh1)$);
    	\coordinate (p110) at ($(p100)+1*(vv1)$);
    	\coordinate (p111) at ($(p110)+1*(vh1)$);
		\coordinate (p112) at ($(p110)+2*(vh1)$);
    	\coordinate (p113) at ($(p110)+3*(vh1)$);
    	\coordinate (p120) at ($(p100)+2*(vv1)$);
    	\coordinate (p121) at ($(p120)+1*(vh1)$);
    	\coordinate (p122) at ($(p120)+2*(vh1)$);
    	\coordinate (p123) at ($(p120)+3*(vh1)$);
    	\coordinate (p130) at ($(p100)+3*(vv1)$);
    	\coordinate (p131) at ($(p130)+1*(vh1)$);
    	\coordinate (p132) at ($(p130)+2*(vh1)$);
    	\coordinate (p133) at ($(p130)+3*(vh1)$);
    	
    	\coordinate (p200) at (p103);
    	\coordinate (p201) at ($(p200)+1*(vh2)$);
    	\coordinate (p202) at ($(p200)+2*(vh2)$);
    	\coordinate (p203) at ($(p200)+3*(vh2)$);
    	\coordinate (p210) at ($(p200)+1*(vv2)$);
    	\coordinate (p211) at ($(p210)+1*(vh2)$);
		\coordinate (p212) at ($(p210)+2*(vh2)$);
    	\coordinate (p213) at ($(p210)+3*(vh2)$);
    	\coordinate (p220) at ($(p200)+2*(vv2)$);
    	\coordinate (p221) at ($(p220)+1*(vh2)$);
    	\coordinate (p222) at ($(p220)+2*(vh2)$);
    	\coordinate (p223) at ($(p220)+3*(vh2)$);
    	\coordinate (p230) at ($(p200)+3*(vv2)$);
    	\coordinate (p231) at ($(p230)+1*(vh2)$);
    	\coordinate (p232) at ($(p230)+2*(vh2)$);
    	\coordinate (p233) at ($(p230)+3*(vh2)$);
    	
		\coordinate (p300) at (p130);
    	\coordinate (p301) at ($(p300)+1*(vh3)$);
    	\coordinate (p302) at ($(p300)+2*(vh3)$);
    	\coordinate (p303) at ($(p300)+3*(vh3)$);
    	\coordinate (p310) at ($(p300)+1*(vv3)$);
    	\coordinate (p311) at ($(p310)+1*(vh3)$);
		\coordinate (p312) at ($(p310)+2*(vh3)$);
    	\coordinate (p313) at ($(p310)+3*(vh3)$);
    	\coordinate (p320) at ($(p300)+2*(vv3)$);
    	\coordinate (p321) at ($(p320)+1*(vh3)$);
    	\coordinate (p322) at ($(p320)+2*(vh3)$);
    	\coordinate (p323) at ($(p320)+3*(vh3)$);
    	\coordinate (p330) at ($(p300)+3*(vv3)$);
    	\coordinate (p331) at ($(p330)+1*(vh3)$);
    	\coordinate (p332) at ($(p330)+2*(vh3)$);
    	\coordinate (p333) at ($(p330)+3*(vh3)$);  	  	
    	
		\filldraw[square, fill=\CAA] (p100) -- (p101) -- (p111) -- (p110);
		\filldraw[square, fill=\CAB] (p101) -- (p102) -- (p112) -- (p111);
		\filldraw[square, fill=\CAC] (p102) -- (p103) -- (p113) -- (p112);
		\filldraw[square, fill=\CAD] (p110) -- (p111) -- (p121) -- (p120);
		\filldraw[square, fill=\CAE] (p111) -- (p112) -- (p122) -- (p121);
		\filldraw[square, fill=\CAF] (p112) -- (p113) -- (p123) -- (p122);
		\filldraw[square, fill=\CAG] (p120) -- (p121) -- (p131) -- (p130);
		\filldraw[square, fill=\CAH] (p121) -- (p122) -- (p132) -- (p131);
		\filldraw[square, fill=\CAI] (p122) -- (p123) -- (p133) -- (p132);
		
		\filldraw[square, fill=\CBA] (p200) -- (p201) -- (p211) -- (p210);
		\filldraw[square, fill=\CBB] (p201) -- (p202) -- (p212) -- (p211);
		\filldraw[square, fill=\CBC] (p202) -- (p203) -- (p213) -- (p212);
		\filldraw[square, fill=\CBD] (p210) -- (p211) -- (p221) -- (p220);
		\filldraw[square, fill=\CBE] (p211) -- (p212) -- (p222) -- (p221);
		\filldraw[square, fill=\CBF] (p212) -- (p213) -- (p223) -- (p222);
		\filldraw[square, fill=\CBG] (p220) -- (p221) -- (p231) -- (p230);
		\filldraw[square, fill=\CBH] (p221) -- (p222) -- (p232) -- (p231);
		\filldraw[square, fill=\CBI] (p222) -- (p223) -- (p233) -- (p232);
		
		\filldraw[square, fill=\CCA] (p300) -- (p301) -- (p311) -- (p310);
		\filldraw[square, fill=\CCB] (p301) -- (p302) -- (p312) -- (p311);
		\filldraw[square, fill=\CCC] (p302) -- (p303) -- (p313) -- (p312);
		\filldraw[square, fill=\CCD] (p310) -- (p311) -- (p321) -- (p320);
		\filldraw[square, fill=\CCE] (p311) -- (p312) -- (p322) -- (p321);
		\filldraw[square, fill=\CCF] (p312) -- (p313) -- (p323) -- (p322);
		\filldraw[square, fill=\CCG] (p320) -- (p321) -- (p331) -- (p330);
		\filldraw[square, fill=\CCH] (p321) -- (p322) -- (p332) -- (p331);
		\filldraw[square, fill=\CCI] (p322) -- (p323) -- (p333) -- (p332);
	
    	\draw[side] (p100) -- (p200) -- (p203) -- (p233) -- (p330) -- (p130) -- (p100) -- (p200);
    	\draw[side] (p130) -- (p133) -- (p233);
    	\draw[side] (p103) -- (p133);
    	
    	\draw[square] (p101) -- (p131) -- (p331);
    	\draw[square] (p102) -- (p132) -- (p332);
    	\draw[square] (p201) -- (p231) -- (p310);
    	\draw[square] (p202) -- (p232) -- (p320);
    	\draw[square] (p110) -- (p113) -- (p213);
    	\draw[square] (p120) -- (p123) -- (p223);
	\end{tikzpicture}
	}    	
& \Large{$f$} &	\resizebox{\W\columnwidth}{!}
	{
	\def\CAA{cblue}
	\def\CAB{cblue}
	\def\CAC{cgray}
	\def\CAD{cblue}
	\def\CAE{cblue}
	\def\CAF{cgray}
	\def\CAG{cblue}
	\def\CAH{cblue}
	\def\CAI{cgray}
	
	\def\CBA{cgray}
	\def\CBB{cyellow}
	\def\CBC{cred}
	\def\CBD{cyellow}
	\def\CBE{cyellow}
	\def\CBF{cred}
	\def\CBG{cgray}
	\def\CBH{cgray}
	\def\CBI{cgray}
	
	\def\CCA{corange}
	\def\CCB{corange}
	\def\CCC{cgray}
	\def\CCD{corange}
	\def\CCE{corange}
	\def\CCF{cyellow}
	\def\CCG{cgray}
	\def\CCH{cgray}
	\def\CCI{cgray}
		\def\OFFSET{1.4cm}
	\def\OFF{2.5pt}
	\def\RATIO{0.2}
    \def\HXA{2.5cm}
    \def\HYA{-0.5cm}
    \def\VXA{0.0cm}
    \def\VYA{2.55cm}
    \def\HXB{1.5cm}
    \def\HYB{0.85cm}
    \def\VXB{\VXA}
    \def\VYB{\VYA}
    \def\HXC{\HXA}
    \def\HYC{\HYA}
    \def\VXC{\HXB}
    \def\VYC{\HYB}
    \tikzstyle{square}=[line width=3pt, join=round, cap=round]
    \tikzstyle{side}=[line width=5pt, join=round, cap=round]
    \tikzstyle{arrow}=[line width=9pt, join=round, cap=round,->,rounded corners=3cm, cpurple]
    %\tikzstyle{double arrow}=[9pt colored by black and white]
    \tikzstyle{border}=[line width=2pt, join=round, cap=round]
    \begin{tikzpicture}[>=triangle 45]
		\coordinate (vh1) at (\HXA ,\HYA);
		\coordinate (vv1) at (\VXA, \VYA);
		\coordinate (vh2) at (\HXB, \HYB);
		\coordinate (vv2) at (\VXB, \VYB);
		\coordinate (vh3) at (\HXC, \HYC);
		\coordinate (vv3) at (\VXC, \VYC);
		    
    	\coordinate (p100) at (0,0);
    	\coordinate (p101) at ($(p100)+1*(vh1)$);
    	\coordinate (p102) at ($(p100)+2*(vh1)$);
    	\coordinate (p103) at ($(p100)+3*(vh1)$);
    	\coordinate (p110) at ($(p100)+1*(vv1)$);
    	\coordinate (p111) at ($(p110)+1*(vh1)$);
		\coordinate (p112) at ($(p110)+2*(vh1)$);
    	\coordinate (p113) at ($(p110)+3*(vh1)$);
    	\coordinate (p120) at ($(p100)+2*(vv1)$);
    	\coordinate (p121) at ($(p120)+1*(vh1)$);
    	\coordinate (p122) at ($(p120)+2*(vh1)$);
    	\coordinate (p123) at ($(p120)+3*(vh1)$);
    	\coordinate (p130) at ($(p100)+3*(vv1)$);
    	\coordinate (p131) at ($(p130)+1*(vh1)$);
    	\coordinate (p132) at ($(p130)+2*(vh1)$);
    	\coordinate (p133) at ($(p130)+3*(vh1)$);
    	
    	\coordinate (p200) at (p103);
    	\coordinate (p201) at ($(p200)+1*(vh2)$);
    	\coordinate (p202) at ($(p200)+2*(vh2)$);
    	\coordinate (p203) at ($(p200)+3*(vh2)$);
    	\coordinate (p210) at ($(p200)+1*(vv2)$);
    	\coordinate (p211) at ($(p210)+1*(vh2)$);
		\coordinate (p212) at ($(p210)+2*(vh2)$);
    	\coordinate (p213) at ($(p210)+3*(vh2)$);
    	\coordinate (p220) at ($(p200)+2*(vv2)$);
    	\coordinate (p221) at ($(p220)+1*(vh2)$);
    	\coordinate (p222) at ($(p220)+2*(vh2)$);
    	\coordinate (p223) at ($(p220)+3*(vh2)$);
    	\coordinate (p230) at ($(p200)+3*(vv2)$);
    	\coordinate (p231) at ($(p230)+1*(vh2)$);
    	\coordinate (p232) at ($(p230)+2*(vh2)$);
    	\coordinate (p233) at ($(p230)+3*(vh2)$);
    	
		\coordinate (p300) at (p130);
    	\coordinate (p301) at ($(p300)+1*(vh3)$);
    	\coordinate (p302) at ($(p300)+2*(vh3)$);
    	\coordinate (p303) at ($(p300)+3*(vh3)$);
    	\coordinate (p310) at ($(p300)+1*(vv3)$);
    	\coordinate (p311) at ($(p310)+1*(vh3)$);
		\coordinate (p312) at ($(p310)+2*(vh3)$);
    	\coordinate (p313) at ($(p310)+3*(vh3)$);
    	\coordinate (p320) at ($(p300)+2*(vv3)$);
    	\coordinate (p321) at ($(p320)+1*(vh3)$);
    	\coordinate (p322) at ($(p320)+2*(vh3)$);
    	\coordinate (p323) at ($(p320)+3*(vh3)$);
    	\coordinate (p330) at ($(p300)+3*(vv3)$);
    	\coordinate (p331) at ($(p330)+1*(vh3)$);
    	\coordinate (p332) at ($(p330)+2*(vh3)$);
    	\coordinate (p333) at ($(p330)+3*(vh3)$);  	  	
    	
		\filldraw[square, fill=\CAA] (p100) -- (p101) -- (p111) -- (p110);
		\filldraw[square, fill=\CAB] (p101) -- (p102) -- (p112) -- (p111);
		\filldraw[square, fill=\CAC] (p102) -- (p103) -- (p113) -- (p112);
		\filldraw[square, fill=\CAD] (p110) -- (p111) -- (p121) -- (p120);
		\filldraw[square, fill=\CAE] (p111) -- (p112) -- (p122) -- (p121);
		\filldraw[square, fill=\CAF] (p112) -- (p113) -- (p123) -- (p122);
		\filldraw[square, fill=\CAG] (p120) -- (p121) -- (p131) -- (p130);
		\filldraw[square, fill=\CAH] (p121) -- (p122) -- (p132) -- (p131);
		\filldraw[square, fill=\CAI] (p122) -- (p123) -- (p133) -- (p132);
		
		\filldraw[square, fill=\CBA] (p200) -- (p201) -- (p211) -- (p210);
		\filldraw[square, fill=\CBB] (p201) -- (p202) -- (p212) -- (p211);
		\filldraw[square, fill=\CBC] (p202) -- (p203) -- (p213) -- (p212);
		\filldraw[square, fill=\CBD] (p210) -- (p211) -- (p221) -- (p220);
		\filldraw[square, fill=\CBE] (p211) -- (p212) -- (p222) -- (p221);
		\filldraw[square, fill=\CBF] (p212) -- (p213) -- (p223) -- (p222);
		\filldraw[square, fill=\CBG] (p220) -- (p221) -- (p231) -- (p230);
		\filldraw[square, fill=\CBH] (p221) -- (p222) -- (p232) -- (p231);
		\filldraw[square, fill=\CBI] (p222) -- (p223) -- (p233) -- (p232);
		
		\filldraw[square, fill=\CCA] (p300) -- (p301) -- (p311) -- (p310);
		\filldraw[square, fill=\CCB] (p301) -- (p302) -- (p312) -- (p311);
		\filldraw[square, fill=\CCC] (p302) -- (p303) -- (p313) -- (p312);
		\filldraw[square, fill=\CCD] (p310) -- (p311) -- (p321) -- (p320);
		\filldraw[square, fill=\CCE] (p311) -- (p312) -- (p322) -- (p321);
		\filldraw[square, fill=\CCF] (p312) -- (p313) -- (p323) -- (p322);
		\filldraw[square, fill=\CCG] (p320) -- (p321) -- (p331) -- (p330);
		\filldraw[square, fill=\CCH] (p321) -- (p322) -- (p332) -- (p331);
		\filldraw[square, fill=\CCI] (p322) -- (p323) -- (p333) -- (p332);
	
    	\draw[side] (p100) -- (p200) -- (p203) -- (p233) -- (p330) -- (p130) -- (p100) -- (p200);
    	\draw[side] (p130) -- (p133) -- (p233);
    	\draw[side] (p103) -- (p133);
    	
    	\draw[square] (p101) -- (p131) -- (p331);
    	\draw[square] (p102) -- (p132) -- (p332);
    	\draw[square] (p201) -- (p231) -- (p310);
    	\draw[square] (p202) -- (p232) -- (p320);
    	\draw[square] (p110) -- (p113) -- (p213);
    	\draw[square] (p120) -- (p123) -- (p223);
	\end{tikzpicture}
	}    	
& \Large{$R$ $U$ $R'$ $U'$} &	\resizebox{\W\columnwidth}{!}
	{
	\def\CAA{cblue}
	\def\CAB{cblue}
	\def\CAC{cgray}
	\def\CAD{cblue}
	\def\CAE{cblue}
	\def\CAF{cgray}
	\def\CAG{cblue}
	\def\CAH{cblue}
	\def\CAI{cgray}
	
	\def\CBA{cgray}
	\def\CBB{cyellow}
	\def\CBC{cred}
	\def\CBD{cyellow}
	\def\CBE{cyellow}
	\def\CBF{cred}
	\def\CBG{cgray}
	\def\CBH{cyellow}
	\def\CBI{cgray}
	
	\def\CCA{corange}
	\def\CCB{corange}
	\def\CCC{cgray}
	\def\CCD{corange}
	\def\CCE{corange}
	\def\CCF{cgray}
	\def\CCG{cgray}
	\def\CCH{cyellow}
	\def\CCI{cgray}	
		\def\OFFSET{1.4cm}
	\def\OFF{2.5pt}
	\def\RATIO{0.2}
    \def\HXA{2.5cm}
    \def\HYA{-0.5cm}
    \def\VXA{0.0cm}
    \def\VYA{2.55cm}
    \def\HXB{1.5cm}
    \def\HYB{0.85cm}
    \def\VXB{\VXA}
    \def\VYB{\VYA}
    \def\HXC{\HXA}
    \def\HYC{\HYA}
    \def\VXC{\HXB}
    \def\VYC{\HYB}
    \tikzstyle{square}=[line width=3pt, join=round, cap=round]
    \tikzstyle{side}=[line width=5pt, join=round, cap=round]
    \tikzstyle{arrow}=[line width=9pt, join=round, cap=round,->,rounded corners=3cm, cpurple]
    %\tikzstyle{double arrow}=[9pt colored by black and white]
    \tikzstyle{border}=[line width=2pt, join=round, cap=round]
    \begin{tikzpicture}[>=triangle 45]
		\coordinate (vh1) at (\HXA ,\HYA);
		\coordinate (vv1) at (\VXA, \VYA);
		\coordinate (vh2) at (\HXB, \HYB);
		\coordinate (vv2) at (\VXB, \VYB);
		\coordinate (vh3) at (\HXC, \HYC);
		\coordinate (vv3) at (\VXC, \VYC);
		    
    	\coordinate (p100) at (0,0);
    	\coordinate (p101) at ($(p100)+1*(vh1)$);
    	\coordinate (p102) at ($(p100)+2*(vh1)$);
    	\coordinate (p103) at ($(p100)+3*(vh1)$);
    	\coordinate (p110) at ($(p100)+1*(vv1)$);
    	\coordinate (p111) at ($(p110)+1*(vh1)$);
		\coordinate (p112) at ($(p110)+2*(vh1)$);
    	\coordinate (p113) at ($(p110)+3*(vh1)$);
    	\coordinate (p120) at ($(p100)+2*(vv1)$);
    	\coordinate (p121) at ($(p120)+1*(vh1)$);
    	\coordinate (p122) at ($(p120)+2*(vh1)$);
    	\coordinate (p123) at ($(p120)+3*(vh1)$);
    	\coordinate (p130) at ($(p100)+3*(vv1)$);
    	\coordinate (p131) at ($(p130)+1*(vh1)$);
    	\coordinate (p132) at ($(p130)+2*(vh1)$);
    	\coordinate (p133) at ($(p130)+3*(vh1)$);
    	
    	\coordinate (p200) at (p103);
    	\coordinate (p201) at ($(p200)+1*(vh2)$);
    	\coordinate (p202) at ($(p200)+2*(vh2)$);
    	\coordinate (p203) at ($(p200)+3*(vh2)$);
    	\coordinate (p210) at ($(p200)+1*(vv2)$);
    	\coordinate (p211) at ($(p210)+1*(vh2)$);
		\coordinate (p212) at ($(p210)+2*(vh2)$);
    	\coordinate (p213) at ($(p210)+3*(vh2)$);
    	\coordinate (p220) at ($(p200)+2*(vv2)$);
    	\coordinate (p221) at ($(p220)+1*(vh2)$);
    	\coordinate (p222) at ($(p220)+2*(vh2)$);
    	\coordinate (p223) at ($(p220)+3*(vh2)$);
    	\coordinate (p230) at ($(p200)+3*(vv2)$);
    	\coordinate (p231) at ($(p230)+1*(vh2)$);
    	\coordinate (p232) at ($(p230)+2*(vh2)$);
    	\coordinate (p233) at ($(p230)+3*(vh2)$);
    	
		\coordinate (p300) at (p130);
    	\coordinate (p301) at ($(p300)+1*(vh3)$);
    	\coordinate (p302) at ($(p300)+2*(vh3)$);
    	\coordinate (p303) at ($(p300)+3*(vh3)$);
    	\coordinate (p310) at ($(p300)+1*(vv3)$);
    	\coordinate (p311) at ($(p310)+1*(vh3)$);
		\coordinate (p312) at ($(p310)+2*(vh3)$);
    	\coordinate (p313) at ($(p310)+3*(vh3)$);
    	\coordinate (p320) at ($(p300)+2*(vv3)$);
    	\coordinate (p321) at ($(p320)+1*(vh3)$);
    	\coordinate (p322) at ($(p320)+2*(vh3)$);
    	\coordinate (p323) at ($(p320)+3*(vh3)$);
    	\coordinate (p330) at ($(p300)+3*(vv3)$);
    	\coordinate (p331) at ($(p330)+1*(vh3)$);
    	\coordinate (p332) at ($(p330)+2*(vh3)$);
    	\coordinate (p333) at ($(p330)+3*(vh3)$);  	  	
    	
		\filldraw[square, fill=\CAA] (p100) -- (p101) -- (p111) -- (p110);
		\filldraw[square, fill=\CAB] (p101) -- (p102) -- (p112) -- (p111);
		\filldraw[square, fill=\CAC] (p102) -- (p103) -- (p113) -- (p112);
		\filldraw[square, fill=\CAD] (p110) -- (p111) -- (p121) -- (p120);
		\filldraw[square, fill=\CAE] (p111) -- (p112) -- (p122) -- (p121);
		\filldraw[square, fill=\CAF] (p112) -- (p113) -- (p123) -- (p122);
		\filldraw[square, fill=\CAG] (p120) -- (p121) -- (p131) -- (p130);
		\filldraw[square, fill=\CAH] (p121) -- (p122) -- (p132) -- (p131);
		\filldraw[square, fill=\CAI] (p122) -- (p123) -- (p133) -- (p132);
		
		\filldraw[square, fill=\CBA] (p200) -- (p201) -- (p211) -- (p210);
		\filldraw[square, fill=\CBB] (p201) -- (p202) -- (p212) -- (p211);
		\filldraw[square, fill=\CBC] (p202) -- (p203) -- (p213) -- (p212);
		\filldraw[square, fill=\CBD] (p210) -- (p211) -- (p221) -- (p220);
		\filldraw[square, fill=\CBE] (p211) -- (p212) -- (p222) -- (p221);
		\filldraw[square, fill=\CBF] (p212) -- (p213) -- (p223) -- (p222);
		\filldraw[square, fill=\CBG] (p220) -- (p221) -- (p231) -- (p230);
		\filldraw[square, fill=\CBH] (p221) -- (p222) -- (p232) -- (p231);
		\filldraw[square, fill=\CBI] (p222) -- (p223) -- (p233) -- (p232);
		
		\filldraw[square, fill=\CCA] (p300) -- (p301) -- (p311) -- (p310);
		\filldraw[square, fill=\CCB] (p301) -- (p302) -- (p312) -- (p311);
		\filldraw[square, fill=\CCC] (p302) -- (p303) -- (p313) -- (p312);
		\filldraw[square, fill=\CCD] (p310) -- (p311) -- (p321) -- (p320);
		\filldraw[square, fill=\CCE] (p311) -- (p312) -- (p322) -- (p321);
		\filldraw[square, fill=\CCF] (p312) -- (p313) -- (p323) -- (p322);
		\filldraw[square, fill=\CCG] (p320) -- (p321) -- (p331) -- (p330);
		\filldraw[square, fill=\CCH] (p321) -- (p322) -- (p332) -- (p331);
		\filldraw[square, fill=\CCI] (p322) -- (p323) -- (p333) -- (p332);
	
    	\draw[side] (p100) -- (p200) -- (p203) -- (p233) -- (p330) -- (p130) -- (p100) -- (p200);
    	\draw[side] (p130) -- (p133) -- (p233);
    	\draw[side] (p103) -- (p133);
    	
    	\draw[square] (p101) -- (p131) -- (p331);
    	\draw[square] (p102) -- (p132) -- (p332);
    	\draw[square] (p201) -- (p231) -- (p310);
    	\draw[square] (p202) -- (p232) -- (p320);
    	\draw[square] (p110) -- (p113) -- (p213);
    	\draw[square] (p120) -- (p123) -- (p223);
	\end{tikzpicture}
	}    	
& \Large{$f'$} &	\resizebox{\W\columnwidth}{!}
	{
	\def\CAA{cblue}
	\def\CAB{cblue}
	\def\CAC{cblue}
	\def\CAD{cblue}
	\def\CAE{cblue}
	\def\CAF{cblue}
	\def\CAG{cgray}
	\def\CAH{cgray}
	\def\CAI{cgray}
	
	\def\CBA{cred}
	\def\CBB{cred}
	\def\CBC{cred}
	\def\CBD{cred}
	\def\CBE{cred}
	\def\CBF{cred}
	\def\CBG{cgray}
	\def\CBH{cgray}
	\def\CBI{cgray}
	
	\def\CCA{cgray}
	\def\CCB{cyellow}
	\def\CCC{cgray}
	\def\CCD{cyellow}
	\def\CCE{cyellow}
	\def\CCF{cyellow}
	\def\CCG{cgray}
	\def\CCH{cyellow}
	\def\CCI{cgray}
		\def\OFFSET{1.4cm}
	\def\OFF{2.5pt}
	\def\RATIO{0.2}
    \def\HXA{2.5cm}
    \def\HYA{-0.5cm}
    \def\VXA{0.0cm}
    \def\VYA{2.55cm}
    \def\HXB{1.5cm}
    \def\HYB{0.85cm}
    \def\VXB{\VXA}
    \def\VYB{\VYA}
    \def\HXC{\HXA}
    \def\HYC{\HYA}
    \def\VXC{\HXB}
    \def\VYC{\HYB}
    \tikzstyle{square}=[line width=3pt, join=round, cap=round]
    \tikzstyle{side}=[line width=5pt, join=round, cap=round]
    \tikzstyle{arrow}=[line width=9pt, join=round, cap=round,->,rounded corners=3cm, cpurple]
    %\tikzstyle{double arrow}=[9pt colored by black and white]
    \tikzstyle{border}=[line width=2pt, join=round, cap=round]
    \begin{tikzpicture}[>=triangle 45]
		\coordinate (vh1) at (\HXA ,\HYA);
		\coordinate (vv1) at (\VXA, \VYA);
		\coordinate (vh2) at (\HXB, \HYB);
		\coordinate (vv2) at (\VXB, \VYB);
		\coordinate (vh3) at (\HXC, \HYC);
		\coordinate (vv3) at (\VXC, \VYC);
		    
    	\coordinate (p100) at (0,0);
    	\coordinate (p101) at ($(p100)+1*(vh1)$);
    	\coordinate (p102) at ($(p100)+2*(vh1)$);
    	\coordinate (p103) at ($(p100)+3*(vh1)$);
    	\coordinate (p110) at ($(p100)+1*(vv1)$);
    	\coordinate (p111) at ($(p110)+1*(vh1)$);
		\coordinate (p112) at ($(p110)+2*(vh1)$);
    	\coordinate (p113) at ($(p110)+3*(vh1)$);
    	\coordinate (p120) at ($(p100)+2*(vv1)$);
    	\coordinate (p121) at ($(p120)+1*(vh1)$);
    	\coordinate (p122) at ($(p120)+2*(vh1)$);
    	\coordinate (p123) at ($(p120)+3*(vh1)$);
    	\coordinate (p130) at ($(p100)+3*(vv1)$);
    	\coordinate (p131) at ($(p130)+1*(vh1)$);
    	\coordinate (p132) at ($(p130)+2*(vh1)$);
    	\coordinate (p133) at ($(p130)+3*(vh1)$);
    	
    	\coordinate (p200) at (p103);
    	\coordinate (p201) at ($(p200)+1*(vh2)$);
    	\coordinate (p202) at ($(p200)+2*(vh2)$);
    	\coordinate (p203) at ($(p200)+3*(vh2)$);
    	\coordinate (p210) at ($(p200)+1*(vv2)$);
    	\coordinate (p211) at ($(p210)+1*(vh2)$);
		\coordinate (p212) at ($(p210)+2*(vh2)$);
    	\coordinate (p213) at ($(p210)+3*(vh2)$);
    	\coordinate (p220) at ($(p200)+2*(vv2)$);
    	\coordinate (p221) at ($(p220)+1*(vh2)$);
    	\coordinate (p222) at ($(p220)+2*(vh2)$);
    	\coordinate (p223) at ($(p220)+3*(vh2)$);
    	\coordinate (p230) at ($(p200)+3*(vv2)$);
    	\coordinate (p231) at ($(p230)+1*(vh2)$);
    	\coordinate (p232) at ($(p230)+2*(vh2)$);
    	\coordinate (p233) at ($(p230)+3*(vh2)$);
    	
		\coordinate (p300) at (p130);
    	\coordinate (p301) at ($(p300)+1*(vh3)$);
    	\coordinate (p302) at ($(p300)+2*(vh3)$);
    	\coordinate (p303) at ($(p300)+3*(vh3)$);
    	\coordinate (p310) at ($(p300)+1*(vv3)$);
    	\coordinate (p311) at ($(p310)+1*(vh3)$);
		\coordinate (p312) at ($(p310)+2*(vh3)$);
    	\coordinate (p313) at ($(p310)+3*(vh3)$);
    	\coordinate (p320) at ($(p300)+2*(vv3)$);
    	\coordinate (p321) at ($(p320)+1*(vh3)$);
    	\coordinate (p322) at ($(p320)+2*(vh3)$);
    	\coordinate (p323) at ($(p320)+3*(vh3)$);
    	\coordinate (p330) at ($(p300)+3*(vv3)$);
    	\coordinate (p331) at ($(p330)+1*(vh3)$);
    	\coordinate (p332) at ($(p330)+2*(vh3)$);
    	\coordinate (p333) at ($(p330)+3*(vh3)$);  	  	
    	
		\filldraw[square, fill=\CAA] (p100) -- (p101) -- (p111) -- (p110);
		\filldraw[square, fill=\CAB] (p101) -- (p102) -- (p112) -- (p111);
		\filldraw[square, fill=\CAC] (p102) -- (p103) -- (p113) -- (p112);
		\filldraw[square, fill=\CAD] (p110) -- (p111) -- (p121) -- (p120);
		\filldraw[square, fill=\CAE] (p111) -- (p112) -- (p122) -- (p121);
		\filldraw[square, fill=\CAF] (p112) -- (p113) -- (p123) -- (p122);
		\filldraw[square, fill=\CAG] (p120) -- (p121) -- (p131) -- (p130);
		\filldraw[square, fill=\CAH] (p121) -- (p122) -- (p132) -- (p131);
		\filldraw[square, fill=\CAI] (p122) -- (p123) -- (p133) -- (p132);
		
		\filldraw[square, fill=\CBA] (p200) -- (p201) -- (p211) -- (p210);
		\filldraw[square, fill=\CBB] (p201) -- (p202) -- (p212) -- (p211);
		\filldraw[square, fill=\CBC] (p202) -- (p203) -- (p213) -- (p212);
		\filldraw[square, fill=\CBD] (p210) -- (p211) -- (p221) -- (p220);
		\filldraw[square, fill=\CBE] (p211) -- (p212) -- (p222) -- (p221);
		\filldraw[square, fill=\CBF] (p212) -- (p213) -- (p223) -- (p222);
		\filldraw[square, fill=\CBG] (p220) -- (p221) -- (p231) -- (p230);
		\filldraw[square, fill=\CBH] (p221) -- (p222) -- (p232) -- (p231);
		\filldraw[square, fill=\CBI] (p222) -- (p223) -- (p233) -- (p232);
		
		\filldraw[square, fill=\CCA] (p300) -- (p301) -- (p311) -- (p310);
		\filldraw[square, fill=\CCB] (p301) -- (p302) -- (p312) -- (p311);
		\filldraw[square, fill=\CCC] (p302) -- (p303) -- (p313) -- (p312);
		\filldraw[square, fill=\CCD] (p310) -- (p311) -- (p321) -- (p320);
		\filldraw[square, fill=\CCE] (p311) -- (p312) -- (p322) -- (p321);
		\filldraw[square, fill=\CCF] (p312) -- (p313) -- (p323) -- (p322);
		\filldraw[square, fill=\CCG] (p320) -- (p321) -- (p331) -- (p330);
		\filldraw[square, fill=\CCH] (p321) -- (p322) -- (p332) -- (p331);
		\filldraw[square, fill=\CCI] (p322) -- (p323) -- (p333) -- (p332);
	
    	\draw[side] (p100) -- (p200) -- (p203) -- (p233) -- (p330) -- (p130) -- (p100) -- (p200);
    	\draw[side] (p130) -- (p133) -- (p233);
    	\draw[side] (p103) -- (p133);
    	
    	\draw[square] (p101) -- (p131) -- (p331);
    	\draw[square] (p102) -- (p132) -- (p332);
    	\draw[square] (p201) -- (p231) -- (p310);
    	\draw[square] (p202) -- (p232) -- (p320);
    	\draw[square] (p110) -- (p113) -- (p213);
    	\draw[square] (p120) -- (p123) -- (p223);
	\end{tikzpicture}
	}
\end{tabular}

\showto{french}{\subsection{Cas Point}}
\showto{english}{\subsection{Point case}}

\showto{french}{Pour résoudre le cas "Point", il suffit de faire la même méthode que pour le cas "Barre" (le sens de départ n’est pas important de par la symétrie). Une fois la formule du cas "Barre" exécutée, le "Point" se sera transformé en cas "L". Il faut donc finir la croix avec la formule du cas "L".}

\showto{english}{To solve the ''Point'' case, it is enough to do the same method as for the case "Bar" (the direction of departure is not important by the symmetry). Once the "Bar" case formula is executed, the "Point" will be transformed into the "L" case. You can finish the cross with the formula of the case "L"}

\begin{conseil}
\showto{french}{Voici un petit récapitulatif:}
\showto{english}{Here is a small recap:}
\begin{description}
\showto{french}{
\item[Barre:] Tourner la face de devant + Gâchette
\item[L:] Tourner les deux faces de devant + Gâchette
\item[Point:] Barre + L
}
\showto{english}{
\item[Bar:] Rotate front face + Trigger
\item[L:] Rotate both front face + Trigger
\item[Point:] Bar + L
}
\end{description}
\end{conseil}
\end{document}