\documentclass[0_Main.tex]{subfiles}
\begin{document}

\showto{french}{\section{Coins Blancs}}
\showto{english}{\section{White corners}}
\begin{wrapfigure}{l}{0.42\columnwidth}
\vspace{-20pt}
\centering
	\def\CAA{cgray}
	\def\CAB{cgray}
	\def\CAC{cgray}
	\def\CAD{cgray}
	\def\CAE{cred}
	\def\CAF{cgray}
	\def\CAG{cred}
	\def\CAH{cred}
	\def\CAI{cred}
	
	\def\CBA{cgray}
	\def\CBB{cgray}
	\def\CBC{cgray}
	\def\CBD{cgray}
	\def\CBE{cblue}
	\def\CBF{cgray}
	\def\CBG{cblue}
	\def\CBH{cblue}
	\def\CBI{cblue}
	
	\def\CCA{cwhite}
	\def\CCB{cwhite}
	\def\CCC{cwhite}
	\def\CCD{cwhite}
	\def\CCE{cwhite}
	\def\CCF{cwhite}
	\def\CCG{cwhite}
	\def\CCH{cwhite}
	\def\CCI{cwhite}
	\resizebox{0.35\columnwidth}{!}
	{
		\def\OFFSET{1.4cm}
	\def\OFF{2.5pt}
	\def\RATIO{0.2}
    \def\HXA{2.5cm}
    \def\HYA{-0.5cm}
    \def\VXA{0.0cm}
    \def\VYA{2.55cm}
    \def\HXB{1.5cm}
    \def\HYB{0.85cm}
    \def\VXB{\VXA}
    \def\VYB{\VYA}
    \def\HXC{\HXA}
    \def\HYC{\HYA}
    \def\VXC{\HXB}
    \def\VYC{\HYB}
    \tikzstyle{square}=[line width=3pt, join=round, cap=round]
    \tikzstyle{side}=[line width=5pt, join=round, cap=round]
    \tikzstyle{arrow}=[line width=9pt, join=round, cap=round,->,rounded corners=3cm, cpurple]
    %\tikzstyle{double arrow}=[9pt colored by black and white]
    \tikzstyle{border}=[line width=2pt, join=round, cap=round]
    \begin{tikzpicture}[>=triangle 45]
		\coordinate (vh1) at (\HXA ,\HYA);
		\coordinate (vv1) at (\VXA, \VYA);
		\coordinate (vh2) at (\HXB, \HYB);
		\coordinate (vv2) at (\VXB, \VYB);
		\coordinate (vh3) at (\HXC, \HYC);
		\coordinate (vv3) at (\VXC, \VYC);
		    
    	\coordinate (p100) at (0,0);
    	\coordinate (p101) at ($(p100)+1*(vh1)$);
    	\coordinate (p102) at ($(p100)+2*(vh1)$);
    	\coordinate (p103) at ($(p100)+3*(vh1)$);
    	\coordinate (p110) at ($(p100)+1*(vv1)$);
    	\coordinate (p111) at ($(p110)+1*(vh1)$);
		\coordinate (p112) at ($(p110)+2*(vh1)$);
    	\coordinate (p113) at ($(p110)+3*(vh1)$);
    	\coordinate (p120) at ($(p100)+2*(vv1)$);
    	\coordinate (p121) at ($(p120)+1*(vh1)$);
    	\coordinate (p122) at ($(p120)+2*(vh1)$);
    	\coordinate (p123) at ($(p120)+3*(vh1)$);
    	\coordinate (p130) at ($(p100)+3*(vv1)$);
    	\coordinate (p131) at ($(p130)+1*(vh1)$);
    	\coordinate (p132) at ($(p130)+2*(vh1)$);
    	\coordinate (p133) at ($(p130)+3*(vh1)$);
    	
    	\coordinate (p200) at (p103);
    	\coordinate (p201) at ($(p200)+1*(vh2)$);
    	\coordinate (p202) at ($(p200)+2*(vh2)$);
    	\coordinate (p203) at ($(p200)+3*(vh2)$);
    	\coordinate (p210) at ($(p200)+1*(vv2)$);
    	\coordinate (p211) at ($(p210)+1*(vh2)$);
		\coordinate (p212) at ($(p210)+2*(vh2)$);
    	\coordinate (p213) at ($(p210)+3*(vh2)$);
    	\coordinate (p220) at ($(p200)+2*(vv2)$);
    	\coordinate (p221) at ($(p220)+1*(vh2)$);
    	\coordinate (p222) at ($(p220)+2*(vh2)$);
    	\coordinate (p223) at ($(p220)+3*(vh2)$);
    	\coordinate (p230) at ($(p200)+3*(vv2)$);
    	\coordinate (p231) at ($(p230)+1*(vh2)$);
    	\coordinate (p232) at ($(p230)+2*(vh2)$);
    	\coordinate (p233) at ($(p230)+3*(vh2)$);
    	
		\coordinate (p300) at (p130);
    	\coordinate (p301) at ($(p300)+1*(vh3)$);
    	\coordinate (p302) at ($(p300)+2*(vh3)$);
    	\coordinate (p303) at ($(p300)+3*(vh3)$);
    	\coordinate (p310) at ($(p300)+1*(vv3)$);
    	\coordinate (p311) at ($(p310)+1*(vh3)$);
		\coordinate (p312) at ($(p310)+2*(vh3)$);
    	\coordinate (p313) at ($(p310)+3*(vh3)$);
    	\coordinate (p320) at ($(p300)+2*(vv3)$);
    	\coordinate (p321) at ($(p320)+1*(vh3)$);
    	\coordinate (p322) at ($(p320)+2*(vh3)$);
    	\coordinate (p323) at ($(p320)+3*(vh3)$);
    	\coordinate (p330) at ($(p300)+3*(vv3)$);
    	\coordinate (p331) at ($(p330)+1*(vh3)$);
    	\coordinate (p332) at ($(p330)+2*(vh3)$);
    	\coordinate (p333) at ($(p330)+3*(vh3)$);  	  	
    	
		\filldraw[square, fill=\CAA] (p100) -- (p101) -- (p111) -- (p110);
		\filldraw[square, fill=\CAB] (p101) -- (p102) -- (p112) -- (p111);
		\filldraw[square, fill=\CAC] (p102) -- (p103) -- (p113) -- (p112);
		\filldraw[square, fill=\CAD] (p110) -- (p111) -- (p121) -- (p120);
		\filldraw[square, fill=\CAE] (p111) -- (p112) -- (p122) -- (p121);
		\filldraw[square, fill=\CAF] (p112) -- (p113) -- (p123) -- (p122);
		\filldraw[square, fill=\CAG] (p120) -- (p121) -- (p131) -- (p130);
		\filldraw[square, fill=\CAH] (p121) -- (p122) -- (p132) -- (p131);
		\filldraw[square, fill=\CAI] (p122) -- (p123) -- (p133) -- (p132);
		
		\filldraw[square, fill=\CBA] (p200) -- (p201) -- (p211) -- (p210);
		\filldraw[square, fill=\CBB] (p201) -- (p202) -- (p212) -- (p211);
		\filldraw[square, fill=\CBC] (p202) -- (p203) -- (p213) -- (p212);
		\filldraw[square, fill=\CBD] (p210) -- (p211) -- (p221) -- (p220);
		\filldraw[square, fill=\CBE] (p211) -- (p212) -- (p222) -- (p221);
		\filldraw[square, fill=\CBF] (p212) -- (p213) -- (p223) -- (p222);
		\filldraw[square, fill=\CBG] (p220) -- (p221) -- (p231) -- (p230);
		\filldraw[square, fill=\CBH] (p221) -- (p222) -- (p232) -- (p231);
		\filldraw[square, fill=\CBI] (p222) -- (p223) -- (p233) -- (p232);
		
		\filldraw[square, fill=\CCA] (p300) -- (p301) -- (p311) -- (p310);
		\filldraw[square, fill=\CCB] (p301) -- (p302) -- (p312) -- (p311);
		\filldraw[square, fill=\CCC] (p302) -- (p303) -- (p313) -- (p312);
		\filldraw[square, fill=\CCD] (p310) -- (p311) -- (p321) -- (p320);
		\filldraw[square, fill=\CCE] (p311) -- (p312) -- (p322) -- (p321);
		\filldraw[square, fill=\CCF] (p312) -- (p313) -- (p323) -- (p322);
		\filldraw[square, fill=\CCG] (p320) -- (p321) -- (p331) -- (p330);
		\filldraw[square, fill=\CCH] (p321) -- (p322) -- (p332) -- (p331);
		\filldraw[square, fill=\CCI] (p322) -- (p323) -- (p333) -- (p332);
	
    	\draw[side] (p100) -- (p200) -- (p203) -- (p233) -- (p330) -- (p130) -- (p100) -- (p200);
    	\draw[side] (p130) -- (p133) -- (p233);
    	\draw[side] (p103) -- (p133);
    	
    	\draw[square] (p101) -- (p131) -- (p331);
    	\draw[square] (p102) -- (p132) -- (p332);
    	\draw[square] (p201) -- (p231) -- (p310);
    	\draw[square] (p202) -- (p232) -- (p320);
    	\draw[square] (p110) -- (p113) -- (p213);
    	\draw[square] (p120) -- (p123) -- (p223);
	\end{tikzpicture}
	}    
	%\vspace{-20pt}
	\showto{french}{\caption{\label{fig:faceBlanche} La face blanche finie}}
	\showto{english}{\caption{\label{fig:faceBlanche} The finished white face}}
	\vspace{-10pt}
\end{wrapfigure}

\showto{french}{Ensuite, il faut finir la face blanche en y plaçant les coins. Comme précédemment, il n’y a pas d’algorithme à suivre, il “suffit” d’essayer jusqu’à comprendre le principe, sauf qu’en plus de l’étape de la croix blanche, il faut faire attention de ne pas défaire ce qui a déjà été fait. Cela ne veut pas dire qu’il ne faut plus du tout toucher à la croix, mais il faut toujours faire attention de la remettre en place à chaque fois. Il faut aussi faire attention de mettre les coins au bon emplacement, par exemple, le coin blanc-bleu-rouge doit aller entre les faces bleu et rouge (rappel: la couleur d’une face est donnée par son centre).}
\showto{english}{Then, we must finish the white face by placing the corners. As before, there is no algorithm to follow, you "just" have to try and understand the principle, except that in addition to the step of the white cross, we must be careful not to undo what has already been done. This does not mean that you do not have to touch the cross any more, but you must always be careful to put it back in place every time. You must also be careful to put the corners in the right place, for example, the white-blue-red corner must go between the blue and red faces (reminder: the color of a face is given by its center).}

\clearpage

\begin{conseil}

\showto{french}{Un bon départ pour placer un coin est de le placer en dessous de l’emplacement où il devra se trouver avec son côté blanc vers soit (voir fig.\ref{fig:coin1}: le coin blanc-rouge-bleu se trouve sous l’emplacement représenté ici en noir). L’idée ensuite est de décaler le coin que l’on désire placer (fig.\ref{fig:coin2}) afin de pouvoir descendre l’emplacement (fig.\ref{fig:coin3}) sans perturber l’orientation du coin. Puis on  remet le coin où il était au début (entre fig.\ref{fig:coin1} et fig.\ref{fig:coin4}, il ne fait qu’un allé-retour), sauf que cette fois, comme on a au préalable descendu l’emplacement souhaité, le coin se trouve quasiment à la bonne place. il suffit de remonter le coin pour terminer le placement. Une fois cette étape réussie, vous devriez commencer à être à l’aise avec le cube!}

\showto{english}{A good starting point for placing a corner is to have it underneath where it should be with its white side toward yourself (see fig.\ref{fig:coin1}: the white-red-blue corner is under the location shown here in black). The idea is then to shift the corner you want to place (fig.\ref{fig:coin2}) so you can lower the location (fig.\ref{fig:coin3}) without disturbing the orientation of the corner. Then you put the corner where it was at the beginning (between fig. \ref{fig:coin1} and fig.\ref{fig:coin4}, it makes only a back-and-forth movement), except that this time, as you previously descended the desired location, the corner is almost in the right place. You just have to go back up to finish the placement. Once this step is successful, you should begin to feel comfortable with the cube!}

\end{conseil}

\begin{figure}[H]
\def\W{0.19}
\def\S{0.05mm}
\begin{subfigure}[b]{\W\columnwidth}
\centering
	\def\CAA{cgray}
	\def\CAB{cgray}
	\def\CAC{cwhite}
	\def\CAD{cgray}
	\def\CAE{cred}
	\def\CAF{cgray}
	\def\CAG{cred}
	\def\CAH{cred}
	\def\CAI{cblack}
	
	\def\CBA{cblue}
	\def\CBB{cgray}
	\def\CBC{cgray}
	\def\CBD{cgray}
	\def\CBE{cblue}
	\def\CBF{cgray}
	\def\CBG{cblack}
	\def\CBH{cblue}
	\def\CBI{cblue}
	
	\def\CCA{cwhite}
	\def\CCB{cwhite}
	\def\CCC{cblack}
	\def\CCD{cwhite}
	\def\CCE{cwhite}
	\def\CCF{cwhite}
	\def\CCG{cwhite}
	\def\CCH{cwhite}
	\def\CCI{cwhite}
	\resizebox{\columnwidth}{!}
	{
		\def\OFFSET{1.4cm}
	\def\OFF{2.5pt}
	\def\RATIO{0.2}
    \def\HXA{2.5cm}
    \def\HYA{-0.5cm}
    \def\VXA{0.0cm}
    \def\VYA{2.55cm}
    \def\HXB{1.5cm}
    \def\HYB{0.85cm}
    \def\VXB{\VXA}
    \def\VYB{\VYA}
    \def\HXC{\HXA}
    \def\HYC{\HYA}
    \def\VXC{\HXB}
    \def\VYC{\HYB}
    \tikzstyle{square}=[line width=3pt, join=round, cap=round]
    \tikzstyle{side}=[line width=5pt, join=round, cap=round]
    \tikzstyle{arrow}=[line width=9pt, join=round, cap=round,->,rounded corners=3cm, cpurple]
    %\tikzstyle{double arrow}=[9pt colored by black and white]
    \tikzstyle{border}=[line width=2pt, join=round, cap=round]
    \begin{tikzpicture}[>=triangle 45]
		\coordinate (vh1) at (\HXA ,\HYA);
		\coordinate (vv1) at (\VXA, \VYA);
		\coordinate (vh2) at (\HXB, \HYB);
		\coordinate (vv2) at (\VXB, \VYB);
		\coordinate (vh3) at (\HXC, \HYC);
		\coordinate (vv3) at (\VXC, \VYC);
		    
    	\coordinate (p100) at (0,0);
    	\coordinate (p101) at ($(p100)+1*(vh1)$);
    	\coordinate (p102) at ($(p100)+2*(vh1)$);
    	\coordinate (p103) at ($(p100)+3*(vh1)$);
    	\coordinate (p110) at ($(p100)+1*(vv1)$);
    	\coordinate (p111) at ($(p110)+1*(vh1)$);
		\coordinate (p112) at ($(p110)+2*(vh1)$);
    	\coordinate (p113) at ($(p110)+3*(vh1)$);
    	\coordinate (p120) at ($(p100)+2*(vv1)$);
    	\coordinate (p121) at ($(p120)+1*(vh1)$);
    	\coordinate (p122) at ($(p120)+2*(vh1)$);
    	\coordinate (p123) at ($(p120)+3*(vh1)$);
    	\coordinate (p130) at ($(p100)+3*(vv1)$);
    	\coordinate (p131) at ($(p130)+1*(vh1)$);
    	\coordinate (p132) at ($(p130)+2*(vh1)$);
    	\coordinate (p133) at ($(p130)+3*(vh1)$);
    	
    	\coordinate (p200) at (p103);
    	\coordinate (p201) at ($(p200)+1*(vh2)$);
    	\coordinate (p202) at ($(p200)+2*(vh2)$);
    	\coordinate (p203) at ($(p200)+3*(vh2)$);
    	\coordinate (p210) at ($(p200)+1*(vv2)$);
    	\coordinate (p211) at ($(p210)+1*(vh2)$);
		\coordinate (p212) at ($(p210)+2*(vh2)$);
    	\coordinate (p213) at ($(p210)+3*(vh2)$);
    	\coordinate (p220) at ($(p200)+2*(vv2)$);
    	\coordinate (p221) at ($(p220)+1*(vh2)$);
    	\coordinate (p222) at ($(p220)+2*(vh2)$);
    	\coordinate (p223) at ($(p220)+3*(vh2)$);
    	\coordinate (p230) at ($(p200)+3*(vv2)$);
    	\coordinate (p231) at ($(p230)+1*(vh2)$);
    	\coordinate (p232) at ($(p230)+2*(vh2)$);
    	\coordinate (p233) at ($(p230)+3*(vh2)$);
    	
		\coordinate (p300) at (p130);
    	\coordinate (p301) at ($(p300)+1*(vh3)$);
    	\coordinate (p302) at ($(p300)+2*(vh3)$);
    	\coordinate (p303) at ($(p300)+3*(vh3)$);
    	\coordinate (p310) at ($(p300)+1*(vv3)$);
    	\coordinate (p311) at ($(p310)+1*(vh3)$);
		\coordinate (p312) at ($(p310)+2*(vh3)$);
    	\coordinate (p313) at ($(p310)+3*(vh3)$);
    	\coordinate (p320) at ($(p300)+2*(vv3)$);
    	\coordinate (p321) at ($(p320)+1*(vh3)$);
    	\coordinate (p322) at ($(p320)+2*(vh3)$);
    	\coordinate (p323) at ($(p320)+3*(vh3)$);
    	\coordinate (p330) at ($(p300)+3*(vv3)$);
    	\coordinate (p331) at ($(p330)+1*(vh3)$);
    	\coordinate (p332) at ($(p330)+2*(vh3)$);
    	\coordinate (p333) at ($(p330)+3*(vh3)$);  	  	
    	
		\filldraw[square, fill=\CAA] (p100) -- (p101) -- (p111) -- (p110);
		\filldraw[square, fill=\CAB] (p101) -- (p102) -- (p112) -- (p111);
		\filldraw[square, fill=\CAC] (p102) -- (p103) -- (p113) -- (p112);
		\filldraw[square, fill=\CAD] (p110) -- (p111) -- (p121) -- (p120);
		\filldraw[square, fill=\CAE] (p111) -- (p112) -- (p122) -- (p121);
		\filldraw[square, fill=\CAF] (p112) -- (p113) -- (p123) -- (p122);
		\filldraw[square, fill=\CAG] (p120) -- (p121) -- (p131) -- (p130);
		\filldraw[square, fill=\CAH] (p121) -- (p122) -- (p132) -- (p131);
		\filldraw[square, fill=\CAI] (p122) -- (p123) -- (p133) -- (p132);
		
		\filldraw[square, fill=\CBA] (p200) -- (p201) -- (p211) -- (p210);
		\filldraw[square, fill=\CBB] (p201) -- (p202) -- (p212) -- (p211);
		\filldraw[square, fill=\CBC] (p202) -- (p203) -- (p213) -- (p212);
		\filldraw[square, fill=\CBD] (p210) -- (p211) -- (p221) -- (p220);
		\filldraw[square, fill=\CBE] (p211) -- (p212) -- (p222) -- (p221);
		\filldraw[square, fill=\CBF] (p212) -- (p213) -- (p223) -- (p222);
		\filldraw[square, fill=\CBG] (p220) -- (p221) -- (p231) -- (p230);
		\filldraw[square, fill=\CBH] (p221) -- (p222) -- (p232) -- (p231);
		\filldraw[square, fill=\CBI] (p222) -- (p223) -- (p233) -- (p232);
		
		\filldraw[square, fill=\CCA] (p300) -- (p301) -- (p311) -- (p310);
		\filldraw[square, fill=\CCB] (p301) -- (p302) -- (p312) -- (p311);
		\filldraw[square, fill=\CCC] (p302) -- (p303) -- (p313) -- (p312);
		\filldraw[square, fill=\CCD] (p310) -- (p311) -- (p321) -- (p320);
		\filldraw[square, fill=\CCE] (p311) -- (p312) -- (p322) -- (p321);
		\filldraw[square, fill=\CCF] (p312) -- (p313) -- (p323) -- (p322);
		\filldraw[square, fill=\CCG] (p320) -- (p321) -- (p331) -- (p330);
		\filldraw[square, fill=\CCH] (p321) -- (p322) -- (p332) -- (p331);
		\filldraw[square, fill=\CCI] (p322) -- (p323) -- (p333) -- (p332);
	
    	\draw[side] (p100) -- (p200) -- (p203) -- (p233) -- (p330) -- (p130) -- (p100) -- (p200);
    	\draw[side] (p130) -- (p133) -- (p233);
    	\draw[side] (p103) -- (p133);
    	
    	\draw[square] (p101) -- (p131) -- (p331);
    	\draw[square] (p102) -- (p132) -- (p332);
    	\draw[square] (p201) -- (p231) -- (p310);
    	\draw[square] (p202) -- (p232) -- (p320);
    	\draw[square] (p110) -- (p113) -- (p213);
    	\draw[square] (p120) -- (p123) -- (p223);
	\draw[arrow] ($(p202)+0.5*(vv2)$) -- ($(p103)+0.3*(vv2)$) -- ($(p101)+0.5*(vv2)$);
	\filldraw[border, fill = cred] ($(p102)+(0,-\OFF)$) --++ (vh1) --++ (vh2) --++ ($-\RATIO*(vv2)$) --++ ($-1*(vh2)$) --++ ($-1*(vh1)$) -- ($(p102)+(0,-\OFF)$);
	\end{tikzpicture}
	}    	
	\caption{\label{fig:coin1} \Large{$D'$}}
\end{subfigure}
\hspace{\S}
\begin{subfigure}[b]{\W\columnwidth}
\centering
	\def\CAA{cblue}
	\def\CAB{cgray}
	\def\CAC{cgray}
	\def\CAD{cgray}
	\def\CAE{cred}
	\def\CAF{cgray}
	\def\CAG{cred}
	\def\CAH{cred}
	\def\CAI{cblack}
	
	\def\CBA{cgray}
	\def\CBB{cgray}
	\def\CBC{cgray}
	\def\CBD{cgray}
	\def\CBE{cblue}
	\def\CBF{cgray}
	\def\CBG{cblack}
	\def\CBH{cblue}
	\def\CBI{cblue}
	
	\def\CCA{cwhite}
	\def\CCB{cwhite}
	\def\CCC{cblack}
	\def\CCD{cwhite}
	\def\CCE{cwhite}
	\def\CCF{cwhite}
	\def\CCG{cwhite}
	\def\CCH{cwhite}
	\def\CCI{cwhite}
	\resizebox{\columnwidth}{!}
	{
		\def\OFFSET{1.4cm}
	\def\OFF{2.5pt}
	\def\RATIO{0.2}
    \def\HXA{2.5cm}
    \def\HYA{-0.5cm}
    \def\VXA{0.0cm}
    \def\VYA{2.55cm}
    \def\HXB{1.5cm}
    \def\HYB{0.85cm}
    \def\VXB{\VXA}
    \def\VYB{\VYA}
    \def\HXC{\HXA}
    \def\HYC{\HYA}
    \def\VXC{\HXB}
    \def\VYC{\HYB}
    \tikzstyle{square}=[line width=3pt, join=round, cap=round]
    \tikzstyle{side}=[line width=5pt, join=round, cap=round]
    \tikzstyle{arrow}=[line width=9pt, join=round, cap=round,->,rounded corners=3cm, cpurple]
    %\tikzstyle{double arrow}=[9pt colored by black and white]
    \tikzstyle{border}=[line width=2pt, join=round, cap=round]
    \begin{tikzpicture}[>=triangle 45]
		\coordinate (vh1) at (\HXA ,\HYA);
		\coordinate (vv1) at (\VXA, \VYA);
		\coordinate (vh2) at (\HXB, \HYB);
		\coordinate (vv2) at (\VXB, \VYB);
		\coordinate (vh3) at (\HXC, \HYC);
		\coordinate (vv3) at (\VXC, \VYC);
		    
    	\coordinate (p100) at (0,0);
    	\coordinate (p101) at ($(p100)+1*(vh1)$);
    	\coordinate (p102) at ($(p100)+2*(vh1)$);
    	\coordinate (p103) at ($(p100)+3*(vh1)$);
    	\coordinate (p110) at ($(p100)+1*(vv1)$);
    	\coordinate (p111) at ($(p110)+1*(vh1)$);
		\coordinate (p112) at ($(p110)+2*(vh1)$);
    	\coordinate (p113) at ($(p110)+3*(vh1)$);
    	\coordinate (p120) at ($(p100)+2*(vv1)$);
    	\coordinate (p121) at ($(p120)+1*(vh1)$);
    	\coordinate (p122) at ($(p120)+2*(vh1)$);
    	\coordinate (p123) at ($(p120)+3*(vh1)$);
    	\coordinate (p130) at ($(p100)+3*(vv1)$);
    	\coordinate (p131) at ($(p130)+1*(vh1)$);
    	\coordinate (p132) at ($(p130)+2*(vh1)$);
    	\coordinate (p133) at ($(p130)+3*(vh1)$);
    	
    	\coordinate (p200) at (p103);
    	\coordinate (p201) at ($(p200)+1*(vh2)$);
    	\coordinate (p202) at ($(p200)+2*(vh2)$);
    	\coordinate (p203) at ($(p200)+3*(vh2)$);
    	\coordinate (p210) at ($(p200)+1*(vv2)$);
    	\coordinate (p211) at ($(p210)+1*(vh2)$);
		\coordinate (p212) at ($(p210)+2*(vh2)$);
    	\coordinate (p213) at ($(p210)+3*(vh2)$);
    	\coordinate (p220) at ($(p200)+2*(vv2)$);
    	\coordinate (p221) at ($(p220)+1*(vh2)$);
    	\coordinate (p222) at ($(p220)+2*(vh2)$);
    	\coordinate (p223) at ($(p220)+3*(vh2)$);
    	\coordinate (p230) at ($(p200)+3*(vv2)$);
    	\coordinate (p231) at ($(p230)+1*(vh2)$);
    	\coordinate (p232) at ($(p230)+2*(vh2)$);
    	\coordinate (p233) at ($(p230)+3*(vh2)$);
    	
		\coordinate (p300) at (p130);
    	\coordinate (p301) at ($(p300)+1*(vh3)$);
    	\coordinate (p302) at ($(p300)+2*(vh3)$);
    	\coordinate (p303) at ($(p300)+3*(vh3)$);
    	\coordinate (p310) at ($(p300)+1*(vv3)$);
    	\coordinate (p311) at ($(p310)+1*(vh3)$);
		\coordinate (p312) at ($(p310)+2*(vh3)$);
    	\coordinate (p313) at ($(p310)+3*(vh3)$);
    	\coordinate (p320) at ($(p300)+2*(vv3)$);
    	\coordinate (p321) at ($(p320)+1*(vh3)$);
    	\coordinate (p322) at ($(p320)+2*(vh3)$);
    	\coordinate (p323) at ($(p320)+3*(vh3)$);
    	\coordinate (p330) at ($(p300)+3*(vv3)$);
    	\coordinate (p331) at ($(p330)+1*(vh3)$);
    	\coordinate (p332) at ($(p330)+2*(vh3)$);
    	\coordinate (p333) at ($(p330)+3*(vh3)$);  	  	
    	
		\filldraw[square, fill=\CAA] (p100) -- (p101) -- (p111) -- (p110);
		\filldraw[square, fill=\CAB] (p101) -- (p102) -- (p112) -- (p111);
		\filldraw[square, fill=\CAC] (p102) -- (p103) -- (p113) -- (p112);
		\filldraw[square, fill=\CAD] (p110) -- (p111) -- (p121) -- (p120);
		\filldraw[square, fill=\CAE] (p111) -- (p112) -- (p122) -- (p121);
		\filldraw[square, fill=\CAF] (p112) -- (p113) -- (p123) -- (p122);
		\filldraw[square, fill=\CAG] (p120) -- (p121) -- (p131) -- (p130);
		\filldraw[square, fill=\CAH] (p121) -- (p122) -- (p132) -- (p131);
		\filldraw[square, fill=\CAI] (p122) -- (p123) -- (p133) -- (p132);
		
		\filldraw[square, fill=\CBA] (p200) -- (p201) -- (p211) -- (p210);
		\filldraw[square, fill=\CBB] (p201) -- (p202) -- (p212) -- (p211);
		\filldraw[square, fill=\CBC] (p202) -- (p203) -- (p213) -- (p212);
		\filldraw[square, fill=\CBD] (p210) -- (p211) -- (p221) -- (p220);
		\filldraw[square, fill=\CBE] (p211) -- (p212) -- (p222) -- (p221);
		\filldraw[square, fill=\CBF] (p212) -- (p213) -- (p223) -- (p222);
		\filldraw[square, fill=\CBG] (p220) -- (p221) -- (p231) -- (p230);
		\filldraw[square, fill=\CBH] (p221) -- (p222) -- (p232) -- (p231);
		\filldraw[square, fill=\CBI] (p222) -- (p223) -- (p233) -- (p232);
		
		\filldraw[square, fill=\CCA] (p300) -- (p301) -- (p311) -- (p310);
		\filldraw[square, fill=\CCB] (p301) -- (p302) -- (p312) -- (p311);
		\filldraw[square, fill=\CCC] (p302) -- (p303) -- (p313) -- (p312);
		\filldraw[square, fill=\CCD] (p310) -- (p311) -- (p321) -- (p320);
		\filldraw[square, fill=\CCE] (p311) -- (p312) -- (p322) -- (p321);
		\filldraw[square, fill=\CCF] (p312) -- (p313) -- (p323) -- (p322);
		\filldraw[square, fill=\CCG] (p320) -- (p321) -- (p331) -- (p330);
		\filldraw[square, fill=\CCH] (p321) -- (p322) -- (p332) -- (p331);
		\filldraw[square, fill=\CCI] (p322) -- (p323) -- (p333) -- (p332);
	
    	\draw[side] (p100) -- (p200) -- (p203) -- (p233) -- (p330) -- (p130) -- (p100) -- (p200);
    	\draw[side] (p130) -- (p133) -- (p233);
    	\draw[side] (p103) -- (p133);
    	
    	\draw[square] (p101) -- (p131) -- (p331);
    	\draw[square] (p102) -- (p132) -- (p332);
    	\draw[square] (p201) -- (p231) -- (p310);
    	\draw[square] (p202) -- (p232) -- (p320);
    	\draw[square] (p110) -- (p113) -- (p213);
    	\draw[square] (p120) -- (p123) -- (p223);
	\draw[arrow] ($(p323)-0.5*(vh3)$) -- ($(p303)-0.7*(vh3)$) -- ($(p113)-0.5*(vh1)$);
	\filldraw[border, fill=cwhite] ($(p100)+(-\OFF,0)$) --++ (vv1) --++ ($-\RATIO*(vh1)$) --++ ($-1*(vv1)$) -- ($(p100)+(-\OFF,0)$);
	\filldraw[border, fill=cred] ($(p100)+(0,-\OFF)$) --++ (vh1) --++ ($-\RATIO*(vv1)$) --++ ($-1*(vh1)$) -- ($(p100)+(0,-\OFF)$);
	\end{tikzpicture}
	}    	
	\caption{\label{fig:coin2} \Large{$R'$}}
\end{subfigure}
\hspace{\S}
\begin{subfigure}[b]{\W\columnwidth}
\centering
	\def\CAA{cblue}
	\def\CAB{cgray}
	\def\CAC{cblack}
	\def\CAD{cgray}
	\def\CAE{cred}
	\def\CAF{cwhite}
	\def\CAG{cred}
	\def\CAH{cred}
	\def\CAI{cwhite}
	
	\def\CBA{cblack}
	\def\CBB{cgray}
	\def\CBC{cgray}
	\def\CBD{cblue}
	\def\CBE{cblue}
	\def\CBF{cgray}
	\def\CBG{cblue}
	\def\CBH{cgray}
	\def\CBI{cgray}
	
	\def\CCA{cwhite}
	\def\CCB{cwhite}
	\def\CCC{corange}
	\def\CCD{cwhite}
	\def\CCE{cwhite}
	\def\CCF{cgray}
	\def\CCG{cwhite}
	\def\CCH{cwhite}
	\def\CCI{cgray}
	\resizebox{\columnwidth}{!}
	{
		\def\OFFSET{1.4cm}
	\def\OFF{2.5pt}
	\def\RATIO{0.2}
    \def\HXA{2.5cm}
    \def\HYA{-0.5cm}
    \def\VXA{0.0cm}
    \def\VYA{2.55cm}
    \def\HXB{1.5cm}
    \def\HYB{0.85cm}
    \def\VXB{\VXA}
    \def\VYB{\VYA}
    \def\HXC{\HXA}
    \def\HYC{\HYA}
    \def\VXC{\HXB}
    \def\VYC{\HYB}
    \tikzstyle{square}=[line width=3pt, join=round, cap=round]
    \tikzstyle{side}=[line width=5pt, join=round, cap=round]
    \tikzstyle{arrow}=[line width=9pt, join=round, cap=round,->,rounded corners=3cm, cpurple]
    %\tikzstyle{double arrow}=[9pt colored by black and white]
    \tikzstyle{border}=[line width=2pt, join=round, cap=round]
    \begin{tikzpicture}[>=triangle 45]
		\coordinate (vh1) at (\HXA ,\HYA);
		\coordinate (vv1) at (\VXA, \VYA);
		\coordinate (vh2) at (\HXB, \HYB);
		\coordinate (vv2) at (\VXB, \VYB);
		\coordinate (vh3) at (\HXC, \HYC);
		\coordinate (vv3) at (\VXC, \VYC);
		    
    	\coordinate (p100) at (0,0);
    	\coordinate (p101) at ($(p100)+1*(vh1)$);
    	\coordinate (p102) at ($(p100)+2*(vh1)$);
    	\coordinate (p103) at ($(p100)+3*(vh1)$);
    	\coordinate (p110) at ($(p100)+1*(vv1)$);
    	\coordinate (p111) at ($(p110)+1*(vh1)$);
		\coordinate (p112) at ($(p110)+2*(vh1)$);
    	\coordinate (p113) at ($(p110)+3*(vh1)$);
    	\coordinate (p120) at ($(p100)+2*(vv1)$);
    	\coordinate (p121) at ($(p120)+1*(vh1)$);
    	\coordinate (p122) at ($(p120)+2*(vh1)$);
    	\coordinate (p123) at ($(p120)+3*(vh1)$);
    	\coordinate (p130) at ($(p100)+3*(vv1)$);
    	\coordinate (p131) at ($(p130)+1*(vh1)$);
    	\coordinate (p132) at ($(p130)+2*(vh1)$);
    	\coordinate (p133) at ($(p130)+3*(vh1)$);
    	
    	\coordinate (p200) at (p103);
    	\coordinate (p201) at ($(p200)+1*(vh2)$);
    	\coordinate (p202) at ($(p200)+2*(vh2)$);
    	\coordinate (p203) at ($(p200)+3*(vh2)$);
    	\coordinate (p210) at ($(p200)+1*(vv2)$);
    	\coordinate (p211) at ($(p210)+1*(vh2)$);
		\coordinate (p212) at ($(p210)+2*(vh2)$);
    	\coordinate (p213) at ($(p210)+3*(vh2)$);
    	\coordinate (p220) at ($(p200)+2*(vv2)$);
    	\coordinate (p221) at ($(p220)+1*(vh2)$);
    	\coordinate (p222) at ($(p220)+2*(vh2)$);
    	\coordinate (p223) at ($(p220)+3*(vh2)$);
    	\coordinate (p230) at ($(p200)+3*(vv2)$);
    	\coordinate (p231) at ($(p230)+1*(vh2)$);
    	\coordinate (p232) at ($(p230)+2*(vh2)$);
    	\coordinate (p233) at ($(p230)+3*(vh2)$);
    	
		\coordinate (p300) at (p130);
    	\coordinate (p301) at ($(p300)+1*(vh3)$);
    	\coordinate (p302) at ($(p300)+2*(vh3)$);
    	\coordinate (p303) at ($(p300)+3*(vh3)$);
    	\coordinate (p310) at ($(p300)+1*(vv3)$);
    	\coordinate (p311) at ($(p310)+1*(vh3)$);
		\coordinate (p312) at ($(p310)+2*(vh3)$);
    	\coordinate (p313) at ($(p310)+3*(vh3)$);
    	\coordinate (p320) at ($(p300)+2*(vv3)$);
    	\coordinate (p321) at ($(p320)+1*(vh3)$);
    	\coordinate (p322) at ($(p320)+2*(vh3)$);
    	\coordinate (p323) at ($(p320)+3*(vh3)$);
    	\coordinate (p330) at ($(p300)+3*(vv3)$);
    	\coordinate (p331) at ($(p330)+1*(vh3)$);
    	\coordinate (p332) at ($(p330)+2*(vh3)$);
    	\coordinate (p333) at ($(p330)+3*(vh3)$);  	  	
    	
		\filldraw[square, fill=\CAA] (p100) -- (p101) -- (p111) -- (p110);
		\filldraw[square, fill=\CAB] (p101) -- (p102) -- (p112) -- (p111);
		\filldraw[square, fill=\CAC] (p102) -- (p103) -- (p113) -- (p112);
		\filldraw[square, fill=\CAD] (p110) -- (p111) -- (p121) -- (p120);
		\filldraw[square, fill=\CAE] (p111) -- (p112) -- (p122) -- (p121);
		\filldraw[square, fill=\CAF] (p112) -- (p113) -- (p123) -- (p122);
		\filldraw[square, fill=\CAG] (p120) -- (p121) -- (p131) -- (p130);
		\filldraw[square, fill=\CAH] (p121) -- (p122) -- (p132) -- (p131);
		\filldraw[square, fill=\CAI] (p122) -- (p123) -- (p133) -- (p132);
		
		\filldraw[square, fill=\CBA] (p200) -- (p201) -- (p211) -- (p210);
		\filldraw[square, fill=\CBB] (p201) -- (p202) -- (p212) -- (p211);
		\filldraw[square, fill=\CBC] (p202) -- (p203) -- (p213) -- (p212);
		\filldraw[square, fill=\CBD] (p210) -- (p211) -- (p221) -- (p220);
		\filldraw[square, fill=\CBE] (p211) -- (p212) -- (p222) -- (p221);
		\filldraw[square, fill=\CBF] (p212) -- (p213) -- (p223) -- (p222);
		\filldraw[square, fill=\CBG] (p220) -- (p221) -- (p231) -- (p230);
		\filldraw[square, fill=\CBH] (p221) -- (p222) -- (p232) -- (p231);
		\filldraw[square, fill=\CBI] (p222) -- (p223) -- (p233) -- (p232);
		
		\filldraw[square, fill=\CCA] (p300) -- (p301) -- (p311) -- (p310);
		\filldraw[square, fill=\CCB] (p301) -- (p302) -- (p312) -- (p311);
		\filldraw[square, fill=\CCC] (p302) -- (p303) -- (p313) -- (p312);
		\filldraw[square, fill=\CCD] (p310) -- (p311) -- (p321) -- (p320);
		\filldraw[square, fill=\CCE] (p311) -- (p312) -- (p322) -- (p321);
		\filldraw[square, fill=\CCF] (p312) -- (p313) -- (p323) -- (p322);
		\filldraw[square, fill=\CCG] (p320) -- (p321) -- (p331) -- (p330);
		\filldraw[square, fill=\CCH] (p321) -- (p322) -- (p332) -- (p331);
		\filldraw[square, fill=\CCI] (p322) -- (p323) -- (p333) -- (p332);
	
    	\draw[side] (p100) -- (p200) -- (p203) -- (p233) -- (p330) -- (p130) -- (p100) -- (p200);
    	\draw[side] (p130) -- (p133) -- (p233);
    	\draw[side] (p103) -- (p133);
    	
    	\draw[square] (p101) -- (p131) -- (p331);
    	\draw[square] (p102) -- (p132) -- (p332);
    	\draw[square] (p201) -- (p231) -- (p310);
    	\draw[square] (p202) -- (p232) -- (p320);
    	\draw[square] (p110) -- (p113) -- (p213);
    	\draw[square] (p120) -- (p123) -- (p223);
	\draw[arrow] ($(p101)+0.5*(vv2)$) -- ($(p103)+0.3*(vv2)$) -- ($(p202)+0.5*(vv2)$);
	\filldraw[border, fill=cwhite] ($(p100)+(-\OFF,0)$) --++ (vv1) --++ ($-\RATIO*(vh1)$) --++ ($-1*(vv1)$) -- ($(p100)+(-\OFF,0)$);
	\filldraw[border, fill=cred] ($(p100)+(0,-\OFF)$) --++ (vh1) --++ ($-\RATIO*(vv1)$) --++ ($-1*(vh1)$) -- ($(p100)+(0,-\OFF)$);
	\end{tikzpicture}
	}    
	\caption{\label{fig:coin3} \Large{$D$}}
\end{subfigure}
\hspace{\S}
\begin{subfigure}[b]{\W\columnwidth}
\centering
	\def\CAA{cgray}
	\def\CAB{cgray}
	\def\CAC{cwhite}
	\def\CAD{cgray}
	\def\CAE{cred}
	\def\CAF{cwhite}
	\def\CAG{cred}
	\def\CAH{cred}
	\def\CAI{cwhite}
	
	\def\CBA{cblue}
	\def\CBB{cgray}
	\def\CBC{cgray}
	\def\CBD{cblue}
	\def\CBE{cblue}
	\def\CBF{cgray}
	\def\CBG{cblue}
	\def\CBH{cgray}
	\def\CBI{cgray}
	
	\def\CCA{cwhite}
	\def\CCB{cwhite}
	\def\CCC{corange}
	\def\CCD{cwhite}
	\def\CCE{cwhite}
	\def\CCF{cgray}
	\def\CCG{cwhite}
	\def\CCH{cwhite}
	\def\CCI{cgray}
	\resizebox{\columnwidth}{!}
	{
		\def\OFFSET{1.4cm}
	\def\OFF{2.5pt}
	\def\RATIO{0.2}
    \def\HXA{2.5cm}
    \def\HYA{-0.5cm}
    \def\VXA{0.0cm}
    \def\VYA{2.55cm}
    \def\HXB{1.5cm}
    \def\HYB{0.85cm}
    \def\VXB{\VXA}
    \def\VYB{\VYA}
    \def\HXC{\HXA}
    \def\HYC{\HYA}
    \def\VXC{\HXB}
    \def\VYC{\HYB}
    \tikzstyle{square}=[line width=3pt, join=round, cap=round]
    \tikzstyle{side}=[line width=5pt, join=round, cap=round]
    \tikzstyle{arrow}=[line width=9pt, join=round, cap=round,->,rounded corners=3cm, cpurple]
    %\tikzstyle{double arrow}=[9pt colored by black and white]
    \tikzstyle{border}=[line width=2pt, join=round, cap=round]
    \begin{tikzpicture}[>=triangle 45]
		\coordinate (vh1) at (\HXA ,\HYA);
		\coordinate (vv1) at (\VXA, \VYA);
		\coordinate (vh2) at (\HXB, \HYB);
		\coordinate (vv2) at (\VXB, \VYB);
		\coordinate (vh3) at (\HXC, \HYC);
		\coordinate (vv3) at (\VXC, \VYC);
		    
    	\coordinate (p100) at (0,0);
    	\coordinate (p101) at ($(p100)+1*(vh1)$);
    	\coordinate (p102) at ($(p100)+2*(vh1)$);
    	\coordinate (p103) at ($(p100)+3*(vh1)$);
    	\coordinate (p110) at ($(p100)+1*(vv1)$);
    	\coordinate (p111) at ($(p110)+1*(vh1)$);
		\coordinate (p112) at ($(p110)+2*(vh1)$);
    	\coordinate (p113) at ($(p110)+3*(vh1)$);
    	\coordinate (p120) at ($(p100)+2*(vv1)$);
    	\coordinate (p121) at ($(p120)+1*(vh1)$);
    	\coordinate (p122) at ($(p120)+2*(vh1)$);
    	\coordinate (p123) at ($(p120)+3*(vh1)$);
    	\coordinate (p130) at ($(p100)+3*(vv1)$);
    	\coordinate (p131) at ($(p130)+1*(vh1)$);
    	\coordinate (p132) at ($(p130)+2*(vh1)$);
    	\coordinate (p133) at ($(p130)+3*(vh1)$);
    	
    	\coordinate (p200) at (p103);
    	\coordinate (p201) at ($(p200)+1*(vh2)$);
    	\coordinate (p202) at ($(p200)+2*(vh2)$);
    	\coordinate (p203) at ($(p200)+3*(vh2)$);
    	\coordinate (p210) at ($(p200)+1*(vv2)$);
    	\coordinate (p211) at ($(p210)+1*(vh2)$);
		\coordinate (p212) at ($(p210)+2*(vh2)$);
    	\coordinate (p213) at ($(p210)+3*(vh2)$);
    	\coordinate (p220) at ($(p200)+2*(vv2)$);
    	\coordinate (p221) at ($(p220)+1*(vh2)$);
    	\coordinate (p222) at ($(p220)+2*(vh2)$);
    	\coordinate (p223) at ($(p220)+3*(vh2)$);
    	\coordinate (p230) at ($(p200)+3*(vv2)$);
    	\coordinate (p231) at ($(p230)+1*(vh2)$);
    	\coordinate (p232) at ($(p230)+2*(vh2)$);
    	\coordinate (p233) at ($(p230)+3*(vh2)$);
    	
		\coordinate (p300) at (p130);
    	\coordinate (p301) at ($(p300)+1*(vh3)$);
    	\coordinate (p302) at ($(p300)+2*(vh3)$);
    	\coordinate (p303) at ($(p300)+3*(vh3)$);
    	\coordinate (p310) at ($(p300)+1*(vv3)$);
    	\coordinate (p311) at ($(p310)+1*(vh3)$);
		\coordinate (p312) at ($(p310)+2*(vh3)$);
    	\coordinate (p313) at ($(p310)+3*(vh3)$);
    	\coordinate (p320) at ($(p300)+2*(vv3)$);
    	\coordinate (p321) at ($(p320)+1*(vh3)$);
    	\coordinate (p322) at ($(p320)+2*(vh3)$);
    	\coordinate (p323) at ($(p320)+3*(vh3)$);
    	\coordinate (p330) at ($(p300)+3*(vv3)$);
    	\coordinate (p331) at ($(p330)+1*(vh3)$);
    	\coordinate (p332) at ($(p330)+2*(vh3)$);
    	\coordinate (p333) at ($(p330)+3*(vh3)$);  	  	
    	
		\filldraw[square, fill=\CAA] (p100) -- (p101) -- (p111) -- (p110);
		\filldraw[square, fill=\CAB] (p101) -- (p102) -- (p112) -- (p111);
		\filldraw[square, fill=\CAC] (p102) -- (p103) -- (p113) -- (p112);
		\filldraw[square, fill=\CAD] (p110) -- (p111) -- (p121) -- (p120);
		\filldraw[square, fill=\CAE] (p111) -- (p112) -- (p122) -- (p121);
		\filldraw[square, fill=\CAF] (p112) -- (p113) -- (p123) -- (p122);
		\filldraw[square, fill=\CAG] (p120) -- (p121) -- (p131) -- (p130);
		\filldraw[square, fill=\CAH] (p121) -- (p122) -- (p132) -- (p131);
		\filldraw[square, fill=\CAI] (p122) -- (p123) -- (p133) -- (p132);
		
		\filldraw[square, fill=\CBA] (p200) -- (p201) -- (p211) -- (p210);
		\filldraw[square, fill=\CBB] (p201) -- (p202) -- (p212) -- (p211);
		\filldraw[square, fill=\CBC] (p202) -- (p203) -- (p213) -- (p212);
		\filldraw[square, fill=\CBD] (p210) -- (p211) -- (p221) -- (p220);
		\filldraw[square, fill=\CBE] (p211) -- (p212) -- (p222) -- (p221);
		\filldraw[square, fill=\CBF] (p212) -- (p213) -- (p223) -- (p222);
		\filldraw[square, fill=\CBG] (p220) -- (p221) -- (p231) -- (p230);
		\filldraw[square, fill=\CBH] (p221) -- (p222) -- (p232) -- (p231);
		\filldraw[square, fill=\CBI] (p222) -- (p223) -- (p233) -- (p232);
		
		\filldraw[square, fill=\CCA] (p300) -- (p301) -- (p311) -- (p310);
		\filldraw[square, fill=\CCB] (p301) -- (p302) -- (p312) -- (p311);
		\filldraw[square, fill=\CCC] (p302) -- (p303) -- (p313) -- (p312);
		\filldraw[square, fill=\CCD] (p310) -- (p311) -- (p321) -- (p320);
		\filldraw[square, fill=\CCE] (p311) -- (p312) -- (p322) -- (p321);
		\filldraw[square, fill=\CCF] (p312) -- (p313) -- (p323) -- (p322);
		\filldraw[square, fill=\CCG] (p320) -- (p321) -- (p331) -- (p330);
		\filldraw[square, fill=\CCH] (p321) -- (p322) -- (p332) -- (p331);
		\filldraw[square, fill=\CCI] (p322) -- (p323) -- (p333) -- (p332);
	
    	\draw[side] (p100) -- (p200) -- (p203) -- (p233) -- (p330) -- (p130) -- (p100) -- (p200);
    	\draw[side] (p130) -- (p133) -- (p233);
    	\draw[side] (p103) -- (p133);
    	
    	\draw[square] (p101) -- (p131) -- (p331);
    	\draw[square] (p102) -- (p132) -- (p332);
    	\draw[square] (p201) -- (p231) -- (p310);
    	\draw[square] (p202) -- (p232) -- (p320);
    	\draw[square] (p110) -- (p113) -- (p213);
    	\draw[square] (p120) -- (p123) -- (p223);
	\draw[arrow] ($(p113)-0.5*(vh1)$) -- ($(p303)-0.7*(vh3)$) -- ($(p323)-0.5*(vh3)$);
	\filldraw[border, fill = cred] ($(p102)+(0,-\OFF)$) --++ (vh1) --++ (vh2) --++ ($-\RATIO*(vv2)$) --++ ($-1*(vh2)$) --++ ($-1*(vh1)$) -- ($(p102)+(0,-\OFF)$);
	\end{tikzpicture}
	}    
	\caption{\label{fig:coin4} \Large{$R$}}
\end{subfigure}
\hspace{\S}
\begin{subfigure}[b]{\W\columnwidth}
\centering
	\def\CAA{cgray}
	\def\CAB{cgray}
	\def\CAC{cgray}
	\def\CAD{cgray}
	\def\CAE{cred}
	\def\CAF{cgray}
	\def\CAG{cred}
	\def\CAH{cred}
	\def\CAI{cred}
	
	\def\CBA{cgray}
	\def\CBB{cgray}
	\def\CBC{cgray}
	\def\CBD{cgray}
	\def\CBE{cblue}
	\def\CBF{cgray}
	\def\CBG{cblue}
	\def\CBH{cblue}
	\def\CBI{cblue}
	
	\def\CCA{cwhite}
	\def\CCB{cwhite}
	\def\CCC{cwhite}
	\def\CCD{cwhite}
	\def\CCE{cwhite}
	\def\CCF{cwhite}
	\def\CCG{cwhite}
	\def\CCH{cwhite}
	\def\CCI{cwhite}
	\resizebox{\columnwidth}{!}
	{
		\def\OFFSET{1.4cm}
	\def\OFF{2.5pt}
	\def\RATIO{0.2}
    \def\HXA{2.5cm}
    \def\HYA{-0.5cm}
    \def\VXA{0.0cm}
    \def\VYA{2.55cm}
    \def\HXB{1.5cm}
    \def\HYB{0.85cm}
    \def\VXB{\VXA}
    \def\VYB{\VYA}
    \def\HXC{\HXA}
    \def\HYC{\HYA}
    \def\VXC{\HXB}
    \def\VYC{\HYB}
    \tikzstyle{square}=[line width=3pt, join=round, cap=round]
    \tikzstyle{side}=[line width=5pt, join=round, cap=round]
    \tikzstyle{arrow}=[line width=9pt, join=round, cap=round,->,rounded corners=3cm, cpurple]
    %\tikzstyle{double arrow}=[9pt colored by black and white]
    \tikzstyle{border}=[line width=2pt, join=round, cap=round]
    \begin{tikzpicture}[>=triangle 45]
		\coordinate (vh1) at (\HXA ,\HYA);
		\coordinate (vv1) at (\VXA, \VYA);
		\coordinate (vh2) at (\HXB, \HYB);
		\coordinate (vv2) at (\VXB, \VYB);
		\coordinate (vh3) at (\HXC, \HYC);
		\coordinate (vv3) at (\VXC, \VYC);
		    
    	\coordinate (p100) at (0,0);
    	\coordinate (p101) at ($(p100)+1*(vh1)$);
    	\coordinate (p102) at ($(p100)+2*(vh1)$);
    	\coordinate (p103) at ($(p100)+3*(vh1)$);
    	\coordinate (p110) at ($(p100)+1*(vv1)$);
    	\coordinate (p111) at ($(p110)+1*(vh1)$);
		\coordinate (p112) at ($(p110)+2*(vh1)$);
    	\coordinate (p113) at ($(p110)+3*(vh1)$);
    	\coordinate (p120) at ($(p100)+2*(vv1)$);
    	\coordinate (p121) at ($(p120)+1*(vh1)$);
    	\coordinate (p122) at ($(p120)+2*(vh1)$);
    	\coordinate (p123) at ($(p120)+3*(vh1)$);
    	\coordinate (p130) at ($(p100)+3*(vv1)$);
    	\coordinate (p131) at ($(p130)+1*(vh1)$);
    	\coordinate (p132) at ($(p130)+2*(vh1)$);
    	\coordinate (p133) at ($(p130)+3*(vh1)$);
    	
    	\coordinate (p200) at (p103);
    	\coordinate (p201) at ($(p200)+1*(vh2)$);
    	\coordinate (p202) at ($(p200)+2*(vh2)$);
    	\coordinate (p203) at ($(p200)+3*(vh2)$);
    	\coordinate (p210) at ($(p200)+1*(vv2)$);
    	\coordinate (p211) at ($(p210)+1*(vh2)$);
		\coordinate (p212) at ($(p210)+2*(vh2)$);
    	\coordinate (p213) at ($(p210)+3*(vh2)$);
    	\coordinate (p220) at ($(p200)+2*(vv2)$);
    	\coordinate (p221) at ($(p220)+1*(vh2)$);
    	\coordinate (p222) at ($(p220)+2*(vh2)$);
    	\coordinate (p223) at ($(p220)+3*(vh2)$);
    	\coordinate (p230) at ($(p200)+3*(vv2)$);
    	\coordinate (p231) at ($(p230)+1*(vh2)$);
    	\coordinate (p232) at ($(p230)+2*(vh2)$);
    	\coordinate (p233) at ($(p230)+3*(vh2)$);
    	
		\coordinate (p300) at (p130);
    	\coordinate (p301) at ($(p300)+1*(vh3)$);
    	\coordinate (p302) at ($(p300)+2*(vh3)$);
    	\coordinate (p303) at ($(p300)+3*(vh3)$);
    	\coordinate (p310) at ($(p300)+1*(vv3)$);
    	\coordinate (p311) at ($(p310)+1*(vh3)$);
		\coordinate (p312) at ($(p310)+2*(vh3)$);
    	\coordinate (p313) at ($(p310)+3*(vh3)$);
    	\coordinate (p320) at ($(p300)+2*(vv3)$);
    	\coordinate (p321) at ($(p320)+1*(vh3)$);
    	\coordinate (p322) at ($(p320)+2*(vh3)$);
    	\coordinate (p323) at ($(p320)+3*(vh3)$);
    	\coordinate (p330) at ($(p300)+3*(vv3)$);
    	\coordinate (p331) at ($(p330)+1*(vh3)$);
    	\coordinate (p332) at ($(p330)+2*(vh3)$);
    	\coordinate (p333) at ($(p330)+3*(vh3)$);  	  	
    	
		\filldraw[square, fill=\CAA] (p100) -- (p101) -- (p111) -- (p110);
		\filldraw[square, fill=\CAB] (p101) -- (p102) -- (p112) -- (p111);
		\filldraw[square, fill=\CAC] (p102) -- (p103) -- (p113) -- (p112);
		\filldraw[square, fill=\CAD] (p110) -- (p111) -- (p121) -- (p120);
		\filldraw[square, fill=\CAE] (p111) -- (p112) -- (p122) -- (p121);
		\filldraw[square, fill=\CAF] (p112) -- (p113) -- (p123) -- (p122);
		\filldraw[square, fill=\CAG] (p120) -- (p121) -- (p131) -- (p130);
		\filldraw[square, fill=\CAH] (p121) -- (p122) -- (p132) -- (p131);
		\filldraw[square, fill=\CAI] (p122) -- (p123) -- (p133) -- (p132);
		
		\filldraw[square, fill=\CBA] (p200) -- (p201) -- (p211) -- (p210);
		\filldraw[square, fill=\CBB] (p201) -- (p202) -- (p212) -- (p211);
		\filldraw[square, fill=\CBC] (p202) -- (p203) -- (p213) -- (p212);
		\filldraw[square, fill=\CBD] (p210) -- (p211) -- (p221) -- (p220);
		\filldraw[square, fill=\CBE] (p211) -- (p212) -- (p222) -- (p221);
		\filldraw[square, fill=\CBF] (p212) -- (p213) -- (p223) -- (p222);
		\filldraw[square, fill=\CBG] (p220) -- (p221) -- (p231) -- (p230);
		\filldraw[square, fill=\CBH] (p221) -- (p222) -- (p232) -- (p231);
		\filldraw[square, fill=\CBI] (p222) -- (p223) -- (p233) -- (p232);
		
		\filldraw[square, fill=\CCA] (p300) -- (p301) -- (p311) -- (p310);
		\filldraw[square, fill=\CCB] (p301) -- (p302) -- (p312) -- (p311);
		\filldraw[square, fill=\CCC] (p302) -- (p303) -- (p313) -- (p312);
		\filldraw[square, fill=\CCD] (p310) -- (p311) -- (p321) -- (p320);
		\filldraw[square, fill=\CCE] (p311) -- (p312) -- (p322) -- (p321);
		\filldraw[square, fill=\CCF] (p312) -- (p313) -- (p323) -- (p322);
		\filldraw[square, fill=\CCG] (p320) -- (p321) -- (p331) -- (p330);
		\filldraw[square, fill=\CCH] (p321) -- (p322) -- (p332) -- (p331);
		\filldraw[square, fill=\CCI] (p322) -- (p323) -- (p333) -- (p332);
	
    	\draw[side] (p100) -- (p200) -- (p203) -- (p233) -- (p330) -- (p130) -- (p100) -- (p200);
    	\draw[side] (p130) -- (p133) -- (p233);
    	\draw[side] (p103) -- (p133);
    	
    	\draw[square] (p101) -- (p131) -- (p331);
    	\draw[square] (p102) -- (p132) -- (p332);
    	\draw[square] (p201) -- (p231) -- (p310);
    	\draw[square] (p202) -- (p232) -- (p320);
    	\draw[square] (p110) -- (p113) -- (p213);
    	\draw[square] (p120) -- (p123) -- (p223);
	\end{tikzpicture}
	}    
	\caption{\label{fig:coin5}}
\end{subfigure}
\showto{french}{\caption{\label{fig:coin} Etapes pour placer un coin blanc sans casser la croix blanche.}}
\showto{english}{\caption{\label{fig:coin} Steps to place a white corner without breaking the white cross.}}
\end{figure}

\end{document}