\documentclass[0_Main.tex]{subfiles}
\begin{document}

\showto{french}{\chapter{Introduction}}
\showto{english}{\chapter{Introduction}}
\begin{shownto}{french}
Ce document présente une méthode d'apprentissage complète du Rubik's Cube original (3x3x3). Cette stratégie d'apprentissage part de la méthode la plus simple de résolution du Rubik's Cube expliquée le plus basiquement possible, puis fait évoluer cette méthode progressivement vers une résolution avancée optimisée pour le Speedcubing. Ce document s'adresse donc aussi bien à des débutants qu'à des gens ayant déjà quelques notions et sachant résoudre le cube avec une méthode simplifié.

Une séparation en étape successive est choisie afin de maximiser les chances de mémoriser tout le processus de résolution. Il est conseillé d’attendre de bien maîtriser une partie avant de passer à la suivante.

Avant de commencer l’apprentissage, il semble intéressant de s’arrêter un moment sur des concepts de base du Rubik’s Cube.
\end{shownto}

\begin{shownto}{english}
This document present a full method to learn how to solve an original Rubik's Cube (3x3x3). This learning strategy starts with the simplest method of solving the Rubik's Cube explained as detailed as possible, then this method evolves progressively towards a Speedcubing optimized method. This document is aimed at both complete beginners and more advanced people that already know basics method to solve the cube and wants to improve.

A stepwise separation is chosen to maximize the chances of memorizing the entire resolution process. It is advisable to wait to master one part before moving on to the next.

Before you start learning, it seems interesting to stop for a moment on basic concepts of Rubik's Cube.
\end{shownto}

\showto{french}{\section{C'est quoi un Rubik's Cube}}
\showto{english}{\section{What's a Rubik's Cube}}
\begin{shownto}{french}
Le Rubik’s Cube a été inventé par Ernő Rubik, un professeur d’architecture Hongrois qui voulait faire deviner à ses élèves à quoi ressemblait l’intérieur de ce cube.
Le Rubik’s Cube est composé de 26 petits cubes (qui en réalité ne sont pas vraiment des cubes) et d’un mécanisme central permettant de faire pivoter chacune des faces. Le principal intérêt du Rubik’s Cube provient du nombre de position initiale possible, il y en a 43~252~003~274~489~856~000, soit plus de 43 trillions. Comme c’est un chiffre un peu grand pour se le représenter, voici quelques exemples:

Si on avait à disposition un cube dans chacune des positions possibles et qu’on les rangeait parfaitement sur le sol, il faudrait recouvrir 256 fois la terre entière pour les placer tous.

Si on empilait tous ces Rubik’s Cube, la hauteur du tout serait 250 mille années-lumière, soit le diamètre moyen d’une galaxie.
\end{shownto}

\begin{shownto}{english}
The Rubik's Cube was invented by Ernő Rubik, a Hungarian architecture teacher who wanted to make his students guess what the inside of this cube looked like.
The Rubik's Cube is composed of 26 small cubes (which actually are not really cubes) and a central mechanism to rotate each face. The main interest of Rubik's Cube comes from the number of initial position possible, there are 43~252~003~274~489~856~000, more than 43 quintillions. Since this is a big number to represent it, here are some representations examples:

If we had a cube in each of the possible positions and we fit them perfectly on the ground, we would have to cover the whole earth 256 times to place them all.

If we stack all these Rubik's Cube, the height of the whole would be 250 thousand light-years, the average diameter of a galaxy.
\end{shownto}

\clearpage
\showto{french}{\section{Trucs à savoir}}
\showto{english}{\section{Basics to know}}
\begin{shownto}{french}
Il y a quelques concepts qu’il est bon d’avoir en mémoire avant de commencer l’apprentissage. Premièrement, il faut savoir que les centres des faces (fig.\ref{fig:cubeCentre}) ne bougent pas! Il ne font que tourner sur eux-même. Ils servent donc de référence à la couleur d’une face: la face jaune est la face dont le centre est jaune, etc... 

Il y a 12 arrêtes sur l’ensemble du cube, ces arrêtes possèdent deux couleurs (fig.\ref{fig:cubeArrete}).

Il y a 8 coins qui possèdent eux 3 couleurs (fig.\ref{fig:cubeCoin}).
\end{shownto}

\begin{shownto}{english}
There are some concepts that are good to remember before you start learning. First, you must know that the centers of the faces (fig.\ref{fig:cubeCentre}) do not move! They are just turning on their own. They serve as a reference to the color of a face: the yellow face is the face whose center is yellow, etc ...

There are 12 edges on the entire cube, these edges have two colors (fig.\ref{fig:cubeArrete}).

There are 8 corners that have 3 colors (fig.\ref{fig:cubeCoin}).
\end{shownto}

\begin{figure}[H]
\centering

\begin{subfigure}[b]{0.31\textwidth}
	\def\CAA{cgray}
	\def\CAB{cgray}
	\def\CAC{cgray}
	\def\CAD{cgray}
	\def\CAE{cred}
	\def\CAF{cgray}
	\def\CAG{cgray}
	\def\CAH{cgray}
	\def\CAI{cgray}
	
	\def\CBA{cgray}
	\def\CBB{cgray}
	\def\CBC{cgray}
	\def\CBD{cgray}
	\def\CBE{cblue}
	\def\CBF{cgray}
	\def\CBG{cgray}
	\def\CBH{cgray}
	\def\CBI{cgray}
	
	\def\CCA{cgray}
	\def\CCB{cgray}
	\def\CCC{cgray}
	\def\CCD{cgray}
	\def\CCE{cwhite}
	\def\CCF{cgray}
	\def\CCG{cgray}
	\def\CCH{cgray}
	\def\CCI{cgray}
	\resizebox{\columnwidth}{!}
	{
		\def\OFFSET{1.4cm}
	\def\OFF{2.5pt}
	\def\RATIO{0.2}
    \def\HXA{2.5cm}
    \def\HYA{-0.5cm}
    \def\VXA{0.0cm}
    \def\VYA{2.55cm}
    \def\HXB{1.5cm}
    \def\HYB{0.85cm}
    \def\VXB{\VXA}
    \def\VYB{\VYA}
    \def\HXC{\HXA}
    \def\HYC{\HYA}
    \def\VXC{\HXB}
    \def\VYC{\HYB}
    \tikzstyle{square}=[line width=3pt, join=round, cap=round]
    \tikzstyle{side}=[line width=5pt, join=round, cap=round]
    \tikzstyle{arrow}=[line width=9pt, join=round, cap=round,->,rounded corners=3cm, cpurple]
    %\tikzstyle{double arrow}=[9pt colored by black and white]
    \tikzstyle{border}=[line width=2pt, join=round, cap=round]
    \begin{tikzpicture}[>=triangle 45]
		\coordinate (vh1) at (\HXA ,\HYA);
		\coordinate (vv1) at (\VXA, \VYA);
		\coordinate (vh2) at (\HXB, \HYB);
		\coordinate (vv2) at (\VXB, \VYB);
		\coordinate (vh3) at (\HXC, \HYC);
		\coordinate (vv3) at (\VXC, \VYC);
		    
    	\coordinate (p100) at (0,0);
    	\coordinate (p101) at ($(p100)+1*(vh1)$);
    	\coordinate (p102) at ($(p100)+2*(vh1)$);
    	\coordinate (p103) at ($(p100)+3*(vh1)$);
    	\coordinate (p110) at ($(p100)+1*(vv1)$);
    	\coordinate (p111) at ($(p110)+1*(vh1)$);
		\coordinate (p112) at ($(p110)+2*(vh1)$);
    	\coordinate (p113) at ($(p110)+3*(vh1)$);
    	\coordinate (p120) at ($(p100)+2*(vv1)$);
    	\coordinate (p121) at ($(p120)+1*(vh1)$);
    	\coordinate (p122) at ($(p120)+2*(vh1)$);
    	\coordinate (p123) at ($(p120)+3*(vh1)$);
    	\coordinate (p130) at ($(p100)+3*(vv1)$);
    	\coordinate (p131) at ($(p130)+1*(vh1)$);
    	\coordinate (p132) at ($(p130)+2*(vh1)$);
    	\coordinate (p133) at ($(p130)+3*(vh1)$);
    	
    	\coordinate (p200) at (p103);
    	\coordinate (p201) at ($(p200)+1*(vh2)$);
    	\coordinate (p202) at ($(p200)+2*(vh2)$);
    	\coordinate (p203) at ($(p200)+3*(vh2)$);
    	\coordinate (p210) at ($(p200)+1*(vv2)$);
    	\coordinate (p211) at ($(p210)+1*(vh2)$);
		\coordinate (p212) at ($(p210)+2*(vh2)$);
    	\coordinate (p213) at ($(p210)+3*(vh2)$);
    	\coordinate (p220) at ($(p200)+2*(vv2)$);
    	\coordinate (p221) at ($(p220)+1*(vh2)$);
    	\coordinate (p222) at ($(p220)+2*(vh2)$);
    	\coordinate (p223) at ($(p220)+3*(vh2)$);
    	\coordinate (p230) at ($(p200)+3*(vv2)$);
    	\coordinate (p231) at ($(p230)+1*(vh2)$);
    	\coordinate (p232) at ($(p230)+2*(vh2)$);
    	\coordinate (p233) at ($(p230)+3*(vh2)$);
    	
		\coordinate (p300) at (p130);
    	\coordinate (p301) at ($(p300)+1*(vh3)$);
    	\coordinate (p302) at ($(p300)+2*(vh3)$);
    	\coordinate (p303) at ($(p300)+3*(vh3)$);
    	\coordinate (p310) at ($(p300)+1*(vv3)$);
    	\coordinate (p311) at ($(p310)+1*(vh3)$);
		\coordinate (p312) at ($(p310)+2*(vh3)$);
    	\coordinate (p313) at ($(p310)+3*(vh3)$);
    	\coordinate (p320) at ($(p300)+2*(vv3)$);
    	\coordinate (p321) at ($(p320)+1*(vh3)$);
    	\coordinate (p322) at ($(p320)+2*(vh3)$);
    	\coordinate (p323) at ($(p320)+3*(vh3)$);
    	\coordinate (p330) at ($(p300)+3*(vv3)$);
    	\coordinate (p331) at ($(p330)+1*(vh3)$);
    	\coordinate (p332) at ($(p330)+2*(vh3)$);
    	\coordinate (p333) at ($(p330)+3*(vh3)$);  	  	
    	
		\filldraw[square, fill=\CAA] (p100) -- (p101) -- (p111) -- (p110);
		\filldraw[square, fill=\CAB] (p101) -- (p102) -- (p112) -- (p111);
		\filldraw[square, fill=\CAC] (p102) -- (p103) -- (p113) -- (p112);
		\filldraw[square, fill=\CAD] (p110) -- (p111) -- (p121) -- (p120);
		\filldraw[square, fill=\CAE] (p111) -- (p112) -- (p122) -- (p121);
		\filldraw[square, fill=\CAF] (p112) -- (p113) -- (p123) -- (p122);
		\filldraw[square, fill=\CAG] (p120) -- (p121) -- (p131) -- (p130);
		\filldraw[square, fill=\CAH] (p121) -- (p122) -- (p132) -- (p131);
		\filldraw[square, fill=\CAI] (p122) -- (p123) -- (p133) -- (p132);
		
		\filldraw[square, fill=\CBA] (p200) -- (p201) -- (p211) -- (p210);
		\filldraw[square, fill=\CBB] (p201) -- (p202) -- (p212) -- (p211);
		\filldraw[square, fill=\CBC] (p202) -- (p203) -- (p213) -- (p212);
		\filldraw[square, fill=\CBD] (p210) -- (p211) -- (p221) -- (p220);
		\filldraw[square, fill=\CBE] (p211) -- (p212) -- (p222) -- (p221);
		\filldraw[square, fill=\CBF] (p212) -- (p213) -- (p223) -- (p222);
		\filldraw[square, fill=\CBG] (p220) -- (p221) -- (p231) -- (p230);
		\filldraw[square, fill=\CBH] (p221) -- (p222) -- (p232) -- (p231);
		\filldraw[square, fill=\CBI] (p222) -- (p223) -- (p233) -- (p232);
		
		\filldraw[square, fill=\CCA] (p300) -- (p301) -- (p311) -- (p310);
		\filldraw[square, fill=\CCB] (p301) -- (p302) -- (p312) -- (p311);
		\filldraw[square, fill=\CCC] (p302) -- (p303) -- (p313) -- (p312);
		\filldraw[square, fill=\CCD] (p310) -- (p311) -- (p321) -- (p320);
		\filldraw[square, fill=\CCE] (p311) -- (p312) -- (p322) -- (p321);
		\filldraw[square, fill=\CCF] (p312) -- (p313) -- (p323) -- (p322);
		\filldraw[square, fill=\CCG] (p320) -- (p321) -- (p331) -- (p330);
		\filldraw[square, fill=\CCH] (p321) -- (p322) -- (p332) -- (p331);
		\filldraw[square, fill=\CCI] (p322) -- (p323) -- (p333) -- (p332);
	
    	\draw[side] (p100) -- (p200) -- (p203) -- (p233) -- (p330) -- (p130) -- (p100) -- (p200);
    	\draw[side] (p130) -- (p133) -- (p233);
    	\draw[side] (p103) -- (p133);
    	
    	\draw[square] (p101) -- (p131) -- (p331);
    	\draw[square] (p102) -- (p132) -- (p332);
    	\draw[square] (p201) -- (p231) -- (p310);
    	\draw[square] (p202) -- (p232) -- (p320);
    	\draw[square] (p110) -- (p113) -- (p213);
    	\draw[square] (p120) -- (p123) -- (p223);
	\end{tikzpicture}
	}	
	\caption{\label{fig:cubeCentre} \showto{french}{Les centres.} \showto{english}{Centers}}
\end{subfigure}
~
\begin{subfigure}[b]{0.31\textwidth}
	\def\CAA{cgray}
	\def\CAB{cred}
	\def\CAC{cgray}
	\def\CAD{cred}
	\def\CAE{cgray}
	\def\CAF{cred}
	\def\CAG{cgray}
	\def\CAH{cred}
	\def\CAI{cgray}
	
	\def\CBA{cgray}
	\def\CBB{cblue}
	\def\CBC{cgray}
	\def\CBD{cblue}
	\def\CBE{cgray}
	\def\CBF{cblue}
	\def\CBG{cgray}
	\def\CBH{cblue}
	\def\CBI{cgray}
	
	\def\CCA{cgray}
	\def\CCB{cwhite}
	\def\CCC{cgray}
	\def\CCD{cwhite}
	\def\CCE{cgray}
	\def\CCF{cwhite}
	\def\CCG{cgray}
	\def\CCH{cwhite}
	\def\CCI{cgray}
	\resizebox{\columnwidth}{!}
	{
		\def\OFFSET{1.4cm}
	\def\OFF{2.5pt}
	\def\RATIO{0.2}
    \def\HXA{2.5cm}
    \def\HYA{-0.5cm}
    \def\VXA{0.0cm}
    \def\VYA{2.55cm}
    \def\HXB{1.5cm}
    \def\HYB{0.85cm}
    \def\VXB{\VXA}
    \def\VYB{\VYA}
    \def\HXC{\HXA}
    \def\HYC{\HYA}
    \def\VXC{\HXB}
    \def\VYC{\HYB}
    \tikzstyle{square}=[line width=3pt, join=round, cap=round]
    \tikzstyle{side}=[line width=5pt, join=round, cap=round]
    \tikzstyle{arrow}=[line width=9pt, join=round, cap=round,->,rounded corners=3cm, cpurple]
    %\tikzstyle{double arrow}=[9pt colored by black and white]
    \tikzstyle{border}=[line width=2pt, join=round, cap=round]
    \begin{tikzpicture}[>=triangle 45]
		\coordinate (vh1) at (\HXA ,\HYA);
		\coordinate (vv1) at (\VXA, \VYA);
		\coordinate (vh2) at (\HXB, \HYB);
		\coordinate (vv2) at (\VXB, \VYB);
		\coordinate (vh3) at (\HXC, \HYC);
		\coordinate (vv3) at (\VXC, \VYC);
		    
    	\coordinate (p100) at (0,0);
    	\coordinate (p101) at ($(p100)+1*(vh1)$);
    	\coordinate (p102) at ($(p100)+2*(vh1)$);
    	\coordinate (p103) at ($(p100)+3*(vh1)$);
    	\coordinate (p110) at ($(p100)+1*(vv1)$);
    	\coordinate (p111) at ($(p110)+1*(vh1)$);
		\coordinate (p112) at ($(p110)+2*(vh1)$);
    	\coordinate (p113) at ($(p110)+3*(vh1)$);
    	\coordinate (p120) at ($(p100)+2*(vv1)$);
    	\coordinate (p121) at ($(p120)+1*(vh1)$);
    	\coordinate (p122) at ($(p120)+2*(vh1)$);
    	\coordinate (p123) at ($(p120)+3*(vh1)$);
    	\coordinate (p130) at ($(p100)+3*(vv1)$);
    	\coordinate (p131) at ($(p130)+1*(vh1)$);
    	\coordinate (p132) at ($(p130)+2*(vh1)$);
    	\coordinate (p133) at ($(p130)+3*(vh1)$);
    	
    	\coordinate (p200) at (p103);
    	\coordinate (p201) at ($(p200)+1*(vh2)$);
    	\coordinate (p202) at ($(p200)+2*(vh2)$);
    	\coordinate (p203) at ($(p200)+3*(vh2)$);
    	\coordinate (p210) at ($(p200)+1*(vv2)$);
    	\coordinate (p211) at ($(p210)+1*(vh2)$);
		\coordinate (p212) at ($(p210)+2*(vh2)$);
    	\coordinate (p213) at ($(p210)+3*(vh2)$);
    	\coordinate (p220) at ($(p200)+2*(vv2)$);
    	\coordinate (p221) at ($(p220)+1*(vh2)$);
    	\coordinate (p222) at ($(p220)+2*(vh2)$);
    	\coordinate (p223) at ($(p220)+3*(vh2)$);
    	\coordinate (p230) at ($(p200)+3*(vv2)$);
    	\coordinate (p231) at ($(p230)+1*(vh2)$);
    	\coordinate (p232) at ($(p230)+2*(vh2)$);
    	\coordinate (p233) at ($(p230)+3*(vh2)$);
    	
		\coordinate (p300) at (p130);
    	\coordinate (p301) at ($(p300)+1*(vh3)$);
    	\coordinate (p302) at ($(p300)+2*(vh3)$);
    	\coordinate (p303) at ($(p300)+3*(vh3)$);
    	\coordinate (p310) at ($(p300)+1*(vv3)$);
    	\coordinate (p311) at ($(p310)+1*(vh3)$);
		\coordinate (p312) at ($(p310)+2*(vh3)$);
    	\coordinate (p313) at ($(p310)+3*(vh3)$);
    	\coordinate (p320) at ($(p300)+2*(vv3)$);
    	\coordinate (p321) at ($(p320)+1*(vh3)$);
    	\coordinate (p322) at ($(p320)+2*(vh3)$);
    	\coordinate (p323) at ($(p320)+3*(vh3)$);
    	\coordinate (p330) at ($(p300)+3*(vv3)$);
    	\coordinate (p331) at ($(p330)+1*(vh3)$);
    	\coordinate (p332) at ($(p330)+2*(vh3)$);
    	\coordinate (p333) at ($(p330)+3*(vh3)$);  	  	
    	
		\filldraw[square, fill=\CAA] (p100) -- (p101) -- (p111) -- (p110);
		\filldraw[square, fill=\CAB] (p101) -- (p102) -- (p112) -- (p111);
		\filldraw[square, fill=\CAC] (p102) -- (p103) -- (p113) -- (p112);
		\filldraw[square, fill=\CAD] (p110) -- (p111) -- (p121) -- (p120);
		\filldraw[square, fill=\CAE] (p111) -- (p112) -- (p122) -- (p121);
		\filldraw[square, fill=\CAF] (p112) -- (p113) -- (p123) -- (p122);
		\filldraw[square, fill=\CAG] (p120) -- (p121) -- (p131) -- (p130);
		\filldraw[square, fill=\CAH] (p121) -- (p122) -- (p132) -- (p131);
		\filldraw[square, fill=\CAI] (p122) -- (p123) -- (p133) -- (p132);
		
		\filldraw[square, fill=\CBA] (p200) -- (p201) -- (p211) -- (p210);
		\filldraw[square, fill=\CBB] (p201) -- (p202) -- (p212) -- (p211);
		\filldraw[square, fill=\CBC] (p202) -- (p203) -- (p213) -- (p212);
		\filldraw[square, fill=\CBD] (p210) -- (p211) -- (p221) -- (p220);
		\filldraw[square, fill=\CBE] (p211) -- (p212) -- (p222) -- (p221);
		\filldraw[square, fill=\CBF] (p212) -- (p213) -- (p223) -- (p222);
		\filldraw[square, fill=\CBG] (p220) -- (p221) -- (p231) -- (p230);
		\filldraw[square, fill=\CBH] (p221) -- (p222) -- (p232) -- (p231);
		\filldraw[square, fill=\CBI] (p222) -- (p223) -- (p233) -- (p232);
		
		\filldraw[square, fill=\CCA] (p300) -- (p301) -- (p311) -- (p310);
		\filldraw[square, fill=\CCB] (p301) -- (p302) -- (p312) -- (p311);
		\filldraw[square, fill=\CCC] (p302) -- (p303) -- (p313) -- (p312);
		\filldraw[square, fill=\CCD] (p310) -- (p311) -- (p321) -- (p320);
		\filldraw[square, fill=\CCE] (p311) -- (p312) -- (p322) -- (p321);
		\filldraw[square, fill=\CCF] (p312) -- (p313) -- (p323) -- (p322);
		\filldraw[square, fill=\CCG] (p320) -- (p321) -- (p331) -- (p330);
		\filldraw[square, fill=\CCH] (p321) -- (p322) -- (p332) -- (p331);
		\filldraw[square, fill=\CCI] (p322) -- (p323) -- (p333) -- (p332);
	
    	\draw[side] (p100) -- (p200) -- (p203) -- (p233) -- (p330) -- (p130) -- (p100) -- (p200);
    	\draw[side] (p130) -- (p133) -- (p233);
    	\draw[side] (p103) -- (p133);
    	
    	\draw[square] (p101) -- (p131) -- (p331);
    	\draw[square] (p102) -- (p132) -- (p332);
    	\draw[square] (p201) -- (p231) -- (p310);
    	\draw[square] (p202) -- (p232) -- (p320);
    	\draw[square] (p110) -- (p113) -- (p213);
    	\draw[square] (p120) -- (p123) -- (p223);
	\end{tikzpicture}
	}
	\caption{\label{fig:cubeArrete} \showto{french}{Les arrêtes.} \showto{english}{Edges}}
\end{subfigure}
~
\begin{subfigure}[b]{0.31\textwidth}
	\def\CAA{cred}
	\def\CAB{cgray}
	\def\CAC{cred}
	\def\CAD{cgray}
	\def\CAE{cgray}
	\def\CAF{cgray}
	\def\CAG{cred}
	\def\CAH{cgray}
	\def\CAI{cred}
	
	\def\CBA{cblue}
	\def\CBB{cgray}
	\def\CBC{cblue}
	\def\CBD{cgray}
	\def\CBE{cgray}
	\def\CBF{cgray}
	\def\CBG{cblue}
	\def\CBH{cgray}
	\def\CBI{cblue}
	
	\def\CCA{cwhite}
	\def\CCB{cgray}
	\def\CCC{cwhite}
	\def\CCD{cgray}
	\def\CCE{cgray}
	\def\CCF{cgray}
	\def\CCG{cwhite}
	\def\CCH{cgray}
	\def\CCI{cwhite}
	\resizebox{\columnwidth}{!}
	{
		\def\OFFSET{1.4cm}
	\def\OFF{2.5pt}
	\def\RATIO{0.2}
    \def\HXA{2.5cm}
    \def\HYA{-0.5cm}
    \def\VXA{0.0cm}
    \def\VYA{2.55cm}
    \def\HXB{1.5cm}
    \def\HYB{0.85cm}
    \def\VXB{\VXA}
    \def\VYB{\VYA}
    \def\HXC{\HXA}
    \def\HYC{\HYA}
    \def\VXC{\HXB}
    \def\VYC{\HYB}
    \tikzstyle{square}=[line width=3pt, join=round, cap=round]
    \tikzstyle{side}=[line width=5pt, join=round, cap=round]
    \tikzstyle{arrow}=[line width=9pt, join=round, cap=round,->,rounded corners=3cm, cpurple]
    %\tikzstyle{double arrow}=[9pt colored by black and white]
    \tikzstyle{border}=[line width=2pt, join=round, cap=round]
    \begin{tikzpicture}[>=triangle 45]
		\coordinate (vh1) at (\HXA ,\HYA);
		\coordinate (vv1) at (\VXA, \VYA);
		\coordinate (vh2) at (\HXB, \HYB);
		\coordinate (vv2) at (\VXB, \VYB);
		\coordinate (vh3) at (\HXC, \HYC);
		\coordinate (vv3) at (\VXC, \VYC);
		    
    	\coordinate (p100) at (0,0);
    	\coordinate (p101) at ($(p100)+1*(vh1)$);
    	\coordinate (p102) at ($(p100)+2*(vh1)$);
    	\coordinate (p103) at ($(p100)+3*(vh1)$);
    	\coordinate (p110) at ($(p100)+1*(vv1)$);
    	\coordinate (p111) at ($(p110)+1*(vh1)$);
		\coordinate (p112) at ($(p110)+2*(vh1)$);
    	\coordinate (p113) at ($(p110)+3*(vh1)$);
    	\coordinate (p120) at ($(p100)+2*(vv1)$);
    	\coordinate (p121) at ($(p120)+1*(vh1)$);
    	\coordinate (p122) at ($(p120)+2*(vh1)$);
    	\coordinate (p123) at ($(p120)+3*(vh1)$);
    	\coordinate (p130) at ($(p100)+3*(vv1)$);
    	\coordinate (p131) at ($(p130)+1*(vh1)$);
    	\coordinate (p132) at ($(p130)+2*(vh1)$);
    	\coordinate (p133) at ($(p130)+3*(vh1)$);
    	
    	\coordinate (p200) at (p103);
    	\coordinate (p201) at ($(p200)+1*(vh2)$);
    	\coordinate (p202) at ($(p200)+2*(vh2)$);
    	\coordinate (p203) at ($(p200)+3*(vh2)$);
    	\coordinate (p210) at ($(p200)+1*(vv2)$);
    	\coordinate (p211) at ($(p210)+1*(vh2)$);
		\coordinate (p212) at ($(p210)+2*(vh2)$);
    	\coordinate (p213) at ($(p210)+3*(vh2)$);
    	\coordinate (p220) at ($(p200)+2*(vv2)$);
    	\coordinate (p221) at ($(p220)+1*(vh2)$);
    	\coordinate (p222) at ($(p220)+2*(vh2)$);
    	\coordinate (p223) at ($(p220)+3*(vh2)$);
    	\coordinate (p230) at ($(p200)+3*(vv2)$);
    	\coordinate (p231) at ($(p230)+1*(vh2)$);
    	\coordinate (p232) at ($(p230)+2*(vh2)$);
    	\coordinate (p233) at ($(p230)+3*(vh2)$);
    	
		\coordinate (p300) at (p130);
    	\coordinate (p301) at ($(p300)+1*(vh3)$);
    	\coordinate (p302) at ($(p300)+2*(vh3)$);
    	\coordinate (p303) at ($(p300)+3*(vh3)$);
    	\coordinate (p310) at ($(p300)+1*(vv3)$);
    	\coordinate (p311) at ($(p310)+1*(vh3)$);
		\coordinate (p312) at ($(p310)+2*(vh3)$);
    	\coordinate (p313) at ($(p310)+3*(vh3)$);
    	\coordinate (p320) at ($(p300)+2*(vv3)$);
    	\coordinate (p321) at ($(p320)+1*(vh3)$);
    	\coordinate (p322) at ($(p320)+2*(vh3)$);
    	\coordinate (p323) at ($(p320)+3*(vh3)$);
    	\coordinate (p330) at ($(p300)+3*(vv3)$);
    	\coordinate (p331) at ($(p330)+1*(vh3)$);
    	\coordinate (p332) at ($(p330)+2*(vh3)$);
    	\coordinate (p333) at ($(p330)+3*(vh3)$);  	  	
    	
		\filldraw[square, fill=\CAA] (p100) -- (p101) -- (p111) -- (p110);
		\filldraw[square, fill=\CAB] (p101) -- (p102) -- (p112) -- (p111);
		\filldraw[square, fill=\CAC] (p102) -- (p103) -- (p113) -- (p112);
		\filldraw[square, fill=\CAD] (p110) -- (p111) -- (p121) -- (p120);
		\filldraw[square, fill=\CAE] (p111) -- (p112) -- (p122) -- (p121);
		\filldraw[square, fill=\CAF] (p112) -- (p113) -- (p123) -- (p122);
		\filldraw[square, fill=\CAG] (p120) -- (p121) -- (p131) -- (p130);
		\filldraw[square, fill=\CAH] (p121) -- (p122) -- (p132) -- (p131);
		\filldraw[square, fill=\CAI] (p122) -- (p123) -- (p133) -- (p132);
		
		\filldraw[square, fill=\CBA] (p200) -- (p201) -- (p211) -- (p210);
		\filldraw[square, fill=\CBB] (p201) -- (p202) -- (p212) -- (p211);
		\filldraw[square, fill=\CBC] (p202) -- (p203) -- (p213) -- (p212);
		\filldraw[square, fill=\CBD] (p210) -- (p211) -- (p221) -- (p220);
		\filldraw[square, fill=\CBE] (p211) -- (p212) -- (p222) -- (p221);
		\filldraw[square, fill=\CBF] (p212) -- (p213) -- (p223) -- (p222);
		\filldraw[square, fill=\CBG] (p220) -- (p221) -- (p231) -- (p230);
		\filldraw[square, fill=\CBH] (p221) -- (p222) -- (p232) -- (p231);
		\filldraw[square, fill=\CBI] (p222) -- (p223) -- (p233) -- (p232);
		
		\filldraw[square, fill=\CCA] (p300) -- (p301) -- (p311) -- (p310);
		\filldraw[square, fill=\CCB] (p301) -- (p302) -- (p312) -- (p311);
		\filldraw[square, fill=\CCC] (p302) -- (p303) -- (p313) -- (p312);
		\filldraw[square, fill=\CCD] (p310) -- (p311) -- (p321) -- (p320);
		\filldraw[square, fill=\CCE] (p311) -- (p312) -- (p322) -- (p321);
		\filldraw[square, fill=\CCF] (p312) -- (p313) -- (p323) -- (p322);
		\filldraw[square, fill=\CCG] (p320) -- (p321) -- (p331) -- (p330);
		\filldraw[square, fill=\CCH] (p321) -- (p322) -- (p332) -- (p331);
		\filldraw[square, fill=\CCI] (p322) -- (p323) -- (p333) -- (p332);
	
    	\draw[side] (p100) -- (p200) -- (p203) -- (p233) -- (p330) -- (p130) -- (p100) -- (p200);
    	\draw[side] (p130) -- (p133) -- (p233);
    	\draw[side] (p103) -- (p133);
    	
    	\draw[square] (p101) -- (p131) -- (p331);
    	\draw[square] (p102) -- (p132) -- (p332);
    	\draw[square] (p201) -- (p231) -- (p310);
    	\draw[square] (p202) -- (p232) -- (p320);
    	\draw[square] (p110) -- (p113) -- (p213);
    	\draw[square] (p120) -- (p123) -- (p223);
	\end{tikzpicture}
	}
	\caption{\label{fig:cubeCoin} \showto{french}{Les coins.} \showto{english}{Corners}}
\end{subfigure}
\showto{french}{\caption{Les différents type de pièces qui compose le Rubik's Cube.}}
\showto{english}{\caption{Different types of parts composing a Rubik's Cube.}}
\end{figure}

\begin{shownto}{french}
Un Rubik's cube comporte six couleurs, sur la majorité des cubes ces couleurs sont: \colorbox{cblack}{\textcolor{cwhite}{blanc}}, \textcolor{cred}{rouge}, \textcolor{cblue}{bleu}, \textcolor{cyellow}{jaune}, \textcolor{cgreen}{vert} et \textcolor{corange}{orange}. Ces couleurs sont en général rangées de sorte qu'il soit facile de mémoriser les couleur qui sont opposées: le \colorbox{cblack}{\textcolor{cwhite}{blanc}} est en face du \textcolor{cyellow}{jaune}; le \textcolor{cred}{rouge} est en face du \textcolor{corange}{orange}; et le \textcolor{cblue}{bleu} est en face du \textcolor{cgreen}{vert}.

Pour toutes les illustrations des cubes, deux couleurs supplémentaires sont utilisés: \textcolor{cgray}{gris} pour les cubes à ignorer et \textcolor{cblack}{noir} pour montrer où une pièce doit aller.
\end{shownto}

\begin{shownto}{english}
A Rubik's cube has six colors, on most cubes these colors are: \colorbox{cblack}{\textcolor{cwhite}{white}}, \textcolor{cred}{red}, \textcolor{cblue}{blue}, \textcolor{cyellow}{yellow}, \textcolor{cgreen}{green} and \textcolor{corange}{orange}. These colors are usually arranged so that it is easy to memorize the colors that are opposite: \colorbox{cblack}{\textcolor{cwhite}{white}} is opposite to \textcolor{cyellow}{yellow}; \textcolor{cred}{red} is opposite to \textcolor{corange}{orange}; and \textcolor{cblue}{blue} is opposite to \textcolor{cgreen}{green}.

For all cubes illustrations, two additional colors are used: \textcolor{cgray}{gray} for the cubes to ignore and \textcolor{cblack}{black} to show where a piece should go.
\end{shownto}

\clearpage

\subfile{11_notations.tex}

\end{document}