\documentclass[0_Main.tex]{subfiles}
\begin{document}

\showto{french}{\section{Notations de mouvement}}
\showto{english}{\section{Movement notations}}
\begin{wrapfigure}{r}{0.4\columnwidth}
\vspace{-35pt}
\centering
	\def\CAA{cred}
	\def\CAB{cred}
	\def\CAC{cred}
	\def\CAD{cred}
	\def\CAE{cred}
	\def\CAF{cred}
	\def\CAG{cred}
	\def\CAH{cred}
	\def\CAI{cred}
	
	\def\CBA{cblue}
	\def\CBB{cblue}
	\def\CBC{cblue}
	\def\CBD{cblue}
	\def\CBE{cblue}
	\def\CBF{cblue}
	\def\CBG{cblue}
	\def\CBH{cblue}
	\def\CBI{cblue}
	
	\def\CCA{cwhite}
	\def\CCB{cwhite}
	\def\CCC{cwhite}
	\def\CCD{cwhite}
	\def\CCE{cwhite}
	\def\CCF{cwhite}
	\def\CCG{cwhite}
	\def\CCH{cwhite}
	\def\CCI{cwhite}
	\resizebox{0.45\columnwidth}{!}
	{
		\def\OFFSET{1.4cm}
	\def\OFF{2.5pt}
	\def\RATIO{0.2}
    \def\HXA{2.5cm}
    \def\HYA{-0.5cm}
    \def\VXA{0.0cm}
    \def\VYA{2.55cm}
    \def\HXB{1.5cm}
    \def\HYB{0.85cm}
    \def\VXB{\VXA}
    \def\VYB{\VYA}
    \def\HXC{\HXA}
    \def\HYC{\HYA}
    \def\VXC{\HXB}
    \def\VYC{\HYB}
    \tikzstyle{square}=[line width=3pt, join=round, cap=round]
    \tikzstyle{side}=[line width=5pt, join=round, cap=round]
    \tikzstyle{arrow}=[line width=9pt, join=round, cap=round,->,rounded corners=3cm, cpurple]
    %\tikzstyle{double arrow}=[9pt colored by black and white]
    \tikzstyle{border}=[line width=2pt, join=round, cap=round]
    \begin{tikzpicture}[>=triangle 45]
		\coordinate (vh1) at (\HXA ,\HYA);
		\coordinate (vv1) at (\VXA, \VYA);
		\coordinate (vh2) at (\HXB, \HYB);
		\coordinate (vv2) at (\VXB, \VYB);
		\coordinate (vh3) at (\HXC, \HYC);
		\coordinate (vv3) at (\VXC, \VYC);
		    
    	\coordinate (p100) at (0,0);
    	\coordinate (p101) at ($(p100)+1*(vh1)$);
    	\coordinate (p102) at ($(p100)+2*(vh1)$);
    	\coordinate (p103) at ($(p100)+3*(vh1)$);
    	\coordinate (p110) at ($(p100)+1*(vv1)$);
    	\coordinate (p111) at ($(p110)+1*(vh1)$);
		\coordinate (p112) at ($(p110)+2*(vh1)$);
    	\coordinate (p113) at ($(p110)+3*(vh1)$);
    	\coordinate (p120) at ($(p100)+2*(vv1)$);
    	\coordinate (p121) at ($(p120)+1*(vh1)$);
    	\coordinate (p122) at ($(p120)+2*(vh1)$);
    	\coordinate (p123) at ($(p120)+3*(vh1)$);
    	\coordinate (p130) at ($(p100)+3*(vv1)$);
    	\coordinate (p131) at ($(p130)+1*(vh1)$);
    	\coordinate (p132) at ($(p130)+2*(vh1)$);
    	\coordinate (p133) at ($(p130)+3*(vh1)$);
    	
    	\coordinate (p200) at (p103);
    	\coordinate (p201) at ($(p200)+1*(vh2)$);
    	\coordinate (p202) at ($(p200)+2*(vh2)$);
    	\coordinate (p203) at ($(p200)+3*(vh2)$);
    	\coordinate (p210) at ($(p200)+1*(vv2)$);
    	\coordinate (p211) at ($(p210)+1*(vh2)$);
		\coordinate (p212) at ($(p210)+2*(vh2)$);
    	\coordinate (p213) at ($(p210)+3*(vh2)$);
    	\coordinate (p220) at ($(p200)+2*(vv2)$);
    	\coordinate (p221) at ($(p220)+1*(vh2)$);
    	\coordinate (p222) at ($(p220)+2*(vh2)$);
    	\coordinate (p223) at ($(p220)+3*(vh2)$);
    	\coordinate (p230) at ($(p200)+3*(vv2)$);
    	\coordinate (p231) at ($(p230)+1*(vh2)$);
    	\coordinate (p232) at ($(p230)+2*(vh2)$);
    	\coordinate (p233) at ($(p230)+3*(vh2)$);
    	
		\coordinate (p300) at (p130);
    	\coordinate (p301) at ($(p300)+1*(vh3)$);
    	\coordinate (p302) at ($(p300)+2*(vh3)$);
    	\coordinate (p303) at ($(p300)+3*(vh3)$);
    	\coordinate (p310) at ($(p300)+1*(vv3)$);
    	\coordinate (p311) at ($(p310)+1*(vh3)$);
		\coordinate (p312) at ($(p310)+2*(vh3)$);
    	\coordinate (p313) at ($(p310)+3*(vh3)$);
    	\coordinate (p320) at ($(p300)+2*(vv3)$);
    	\coordinate (p321) at ($(p320)+1*(vh3)$);
    	\coordinate (p322) at ($(p320)+2*(vh3)$);
    	\coordinate (p323) at ($(p320)+3*(vh3)$);
    	\coordinate (p330) at ($(p300)+3*(vv3)$);
    	\coordinate (p331) at ($(p330)+1*(vh3)$);
    	\coordinate (p332) at ($(p330)+2*(vh3)$);
    	\coordinate (p333) at ($(p330)+3*(vh3)$);  	  	
    	
		\filldraw[square, fill=\CAA] (p100) -- (p101) -- (p111) -- (p110);
		\filldraw[square, fill=\CAB] (p101) -- (p102) -- (p112) -- (p111);
		\filldraw[square, fill=\CAC] (p102) -- (p103) -- (p113) -- (p112);
		\filldraw[square, fill=\CAD] (p110) -- (p111) -- (p121) -- (p120);
		\filldraw[square, fill=\CAE] (p111) -- (p112) -- (p122) -- (p121);
		\filldraw[square, fill=\CAF] (p112) -- (p113) -- (p123) -- (p122);
		\filldraw[square, fill=\CAG] (p120) -- (p121) -- (p131) -- (p130);
		\filldraw[square, fill=\CAH] (p121) -- (p122) -- (p132) -- (p131);
		\filldraw[square, fill=\CAI] (p122) -- (p123) -- (p133) -- (p132);
		
		\filldraw[square, fill=\CBA] (p200) -- (p201) -- (p211) -- (p210);
		\filldraw[square, fill=\CBB] (p201) -- (p202) -- (p212) -- (p211);
		\filldraw[square, fill=\CBC] (p202) -- (p203) -- (p213) -- (p212);
		\filldraw[square, fill=\CBD] (p210) -- (p211) -- (p221) -- (p220);
		\filldraw[square, fill=\CBE] (p211) -- (p212) -- (p222) -- (p221);
		\filldraw[square, fill=\CBF] (p212) -- (p213) -- (p223) -- (p222);
		\filldraw[square, fill=\CBG] (p220) -- (p221) -- (p231) -- (p230);
		\filldraw[square, fill=\CBH] (p221) -- (p222) -- (p232) -- (p231);
		\filldraw[square, fill=\CBI] (p222) -- (p223) -- (p233) -- (p232);
		
		\filldraw[square, fill=\CCA] (p300) -- (p301) -- (p311) -- (p310);
		\filldraw[square, fill=\CCB] (p301) -- (p302) -- (p312) -- (p311);
		\filldraw[square, fill=\CCC] (p302) -- (p303) -- (p313) -- (p312);
		\filldraw[square, fill=\CCD] (p310) -- (p311) -- (p321) -- (p320);
		\filldraw[square, fill=\CCE] (p311) -- (p312) -- (p322) -- (p321);
		\filldraw[square, fill=\CCF] (p312) -- (p313) -- (p323) -- (p322);
		\filldraw[square, fill=\CCG] (p320) -- (p321) -- (p331) -- (p330);
		\filldraw[square, fill=\CCH] (p321) -- (p322) -- (p332) -- (p331);
		\filldraw[square, fill=\CCI] (p322) -- (p323) -- (p333) -- (p332);
	
    	\draw[side] (p100) -- (p200) -- (p203) -- (p233) -- (p330) -- (p130) -- (p100) -- (p200);
    	\draw[side] (p130) -- (p133) -- (p233);
    	\draw[side] (p103) -- (p133);
    	
    	\draw[square] (p101) -- (p131) -- (p331);
    	\draw[square] (p102) -- (p132) -- (p332);
    	\draw[square] (p201) -- (p231) -- (p310);
    	\draw[square] (p202) -- (p232) -- (p320);
    	\draw[square] (p110) -- (p113) -- (p213);
    	\draw[square] (p120) -- (p123) -- (p223);
	\draw[doubled] ($(p311)+0.5*(vh3)+0.5*(vv3)+1*(vv1)$) node[above](up){\resizebox{\HXA}{!}{\textcolor{black}{Up}}} --++($-1*(vv1)$);
	\draw[doubled] ($(p332)+1.5*(vv3)$) node[right](back){\resizebox{3.8cm}{!}{\textcolor{black}{Back}}} --++($-1.5*(vv3)$);
	\draw[doubled] ($(p310)-1.5*(vh3)$) node[left](left){\resizebox{3.8cm}{!}{\textcolor{black}{Left}}} --++($1.5*(vh3)$);
	\draw[doubled] ($(p211)+0.5*(vh2)+0.5*(vv2)+2*(vh2)$) node[right](right){\resizebox{4.2cm}{!}{\textcolor{black}{Right}}} --++($-2*(vh2)$);
	\draw[doubled] ($(p111)+0.5*(vh1)+0.5*(vv1)-2*(vh1)$) node[left](front){\resizebox{4.2cm}{!}{\textcolor{black}{Front}}} --++($2*(vh1)$);
	\draw[doubled] ($(p102)-1*(vv1)$) node[below](below) {\resizebox{4.6cm}{!}{\textcolor{black}{Down}}} --++($1*(vv1)$);
	\end{tikzpicture}
	}    
	\vspace{-10pt}
	\showto{french}{\caption{\label{fig:face}Définition des faces.}}
	\showto{english}{\caption{\label{fig:face}Faces definition}}
\end{wrapfigure}

\showto{french}{Afin de faciliter l'explication des séquences de mouvement, une système de notations des mouvements standardisé est utilisé. Bien que le système ne soit pas forcément simple à maîtriser, il reste clair et facile à comprendre: étant donné un cube dans une certaine position, chacune de ses faces sont représentées par la première lettre du nom anglais (fig.\ref{fig:face}). Les séquences de mouvements sont des suites de ces lettres où une lettre simple représente une rotation d’un quart de tour dans le sens des aiguilles d’une montre et une lettre apostrophée représente une rotation d’un quart de tour dans le sens inverse des aiguilles d’une montre. Le début de l'apprentissage n'utilise pas ces notations, puis elles sont introduite progressivement, il n'est donc pas indispensable qu'elles soient déjà maîtrisées à ce stade.}

\showto{english}{In order to facilitate the explanation of movement sequences, a standardized movement notation system is used. Although the system is not necessarily easy to master, it remains clear and easy to understand: given a cube in a certain position, each of its faces is represented by the first letter of the position's name (fig. \ref{fig:face}). The sequences of movements are defined by sequences of these letters where a single letter represents a clockwise quarter turn rotation and an apostrophed letter represents counterclockwise quarter turn rotation. The beginning of learning does not use these notations, then they are introduced gradually, so it is not essential that they are already mastered at this stage.}

\begin{figure}[H]
\def\W{0.15}
\def\S{0.05mm}

\centering
\begin{subfigure}[b]{\W\columnwidth}
\centering
	\def\CAA{cred}
	\def\CAB{cred}
	\def\CAC{cyellow}
	\def\CAD{cred}
	\def\CAE{cred}
	\def\CAF{cyellow}
	\def\CAG{cred}
	\def\CAH{cred}
	\def\CAI{cyellow}
	
	\def\CBA{cblue}
	\def\CBB{cblue}
	\def\CBC{cblue}
	\def\CBD{cblue}
	\def\CBE{cblue}
	\def\CBF{cblue}
	\def\CBG{cblue}
	\def\CBH{cblue}
	\def\CBI{cblue}
	
	\def\CCA{cwhite}
	\def\CCB{cwhite}
	\def\CCC{cred}
	\def\CCD{cwhite}
	\def\CCE{cwhite}
	\def\CCF{cred}
	\def\CCG{cwhite}
	\def\CCH{cwhite}
	\def\CCI{cred}
	\caption*{\label{fig:mouvR} \LARGE{$R$}}
	\resizebox{\columnwidth}{!}
	{
		\def\OFFSET{1.4cm}
	\def\OFF{2.5pt}
	\def\RATIO{0.2}
    \def\HXA{2.5cm}
    \def\HYA{-0.5cm}
    \def\VXA{0.0cm}
    \def\VYA{2.55cm}
    \def\HXB{1.5cm}
    \def\HYB{0.85cm}
    \def\VXB{\VXA}
    \def\VYB{\VYA}
    \def\HXC{\HXA}
    \def\HYC{\HYA}
    \def\VXC{\HXB}
    \def\VYC{\HYB}
    \tikzstyle{square}=[line width=3pt, join=round, cap=round]
    \tikzstyle{side}=[line width=5pt, join=round, cap=round]
    \tikzstyle{arrow}=[line width=9pt, join=round, cap=round,->,rounded corners=3cm, cpurple]
    %\tikzstyle{double arrow}=[9pt colored by black and white]
    \tikzstyle{border}=[line width=2pt, join=round, cap=round]
    \begin{tikzpicture}[>=triangle 45]
		\coordinate (vh1) at (\HXA ,\HYA);
		\coordinate (vv1) at (\VXA, \VYA);
		\coordinate (vh2) at (\HXB, \HYB);
		\coordinate (vv2) at (\VXB, \VYB);
		\coordinate (vh3) at (\HXC, \HYC);
		\coordinate (vv3) at (\VXC, \VYC);
		    
    	\coordinate (p100) at (0,0);
    	\coordinate (p101) at ($(p100)+1*(vh1)$);
    	\coordinate (p102) at ($(p100)+2*(vh1)$);
    	\coordinate (p103) at ($(p100)+3*(vh1)$);
    	\coordinate (p110) at ($(p100)+1*(vv1)$);
    	\coordinate (p111) at ($(p110)+1*(vh1)$);
		\coordinate (p112) at ($(p110)+2*(vh1)$);
    	\coordinate (p113) at ($(p110)+3*(vh1)$);
    	\coordinate (p120) at ($(p100)+2*(vv1)$);
    	\coordinate (p121) at ($(p120)+1*(vh1)$);
    	\coordinate (p122) at ($(p120)+2*(vh1)$);
    	\coordinate (p123) at ($(p120)+3*(vh1)$);
    	\coordinate (p130) at ($(p100)+3*(vv1)$);
    	\coordinate (p131) at ($(p130)+1*(vh1)$);
    	\coordinate (p132) at ($(p130)+2*(vh1)$);
    	\coordinate (p133) at ($(p130)+3*(vh1)$);
    	
    	\coordinate (p200) at (p103);
    	\coordinate (p201) at ($(p200)+1*(vh2)$);
    	\coordinate (p202) at ($(p200)+2*(vh2)$);
    	\coordinate (p203) at ($(p200)+3*(vh2)$);
    	\coordinate (p210) at ($(p200)+1*(vv2)$);
    	\coordinate (p211) at ($(p210)+1*(vh2)$);
		\coordinate (p212) at ($(p210)+2*(vh2)$);
    	\coordinate (p213) at ($(p210)+3*(vh2)$);
    	\coordinate (p220) at ($(p200)+2*(vv2)$);
    	\coordinate (p221) at ($(p220)+1*(vh2)$);
    	\coordinate (p222) at ($(p220)+2*(vh2)$);
    	\coordinate (p223) at ($(p220)+3*(vh2)$);
    	\coordinate (p230) at ($(p200)+3*(vv2)$);
    	\coordinate (p231) at ($(p230)+1*(vh2)$);
    	\coordinate (p232) at ($(p230)+2*(vh2)$);
    	\coordinate (p233) at ($(p230)+3*(vh2)$);
    	
		\coordinate (p300) at (p130);
    	\coordinate (p301) at ($(p300)+1*(vh3)$);
    	\coordinate (p302) at ($(p300)+2*(vh3)$);
    	\coordinate (p303) at ($(p300)+3*(vh3)$);
    	\coordinate (p310) at ($(p300)+1*(vv3)$);
    	\coordinate (p311) at ($(p310)+1*(vh3)$);
		\coordinate (p312) at ($(p310)+2*(vh3)$);
    	\coordinate (p313) at ($(p310)+3*(vh3)$);
    	\coordinate (p320) at ($(p300)+2*(vv3)$);
    	\coordinate (p321) at ($(p320)+1*(vh3)$);
    	\coordinate (p322) at ($(p320)+2*(vh3)$);
    	\coordinate (p323) at ($(p320)+3*(vh3)$);
    	\coordinate (p330) at ($(p300)+3*(vv3)$);
    	\coordinate (p331) at ($(p330)+1*(vh3)$);
    	\coordinate (p332) at ($(p330)+2*(vh3)$);
    	\coordinate (p333) at ($(p330)+3*(vh3)$);  	  	
    	
		\filldraw[square, fill=\CAA] (p100) -- (p101) -- (p111) -- (p110);
		\filldraw[square, fill=\CAB] (p101) -- (p102) -- (p112) -- (p111);
		\filldraw[square, fill=\CAC] (p102) -- (p103) -- (p113) -- (p112);
		\filldraw[square, fill=\CAD] (p110) -- (p111) -- (p121) -- (p120);
		\filldraw[square, fill=\CAE] (p111) -- (p112) -- (p122) -- (p121);
		\filldraw[square, fill=\CAF] (p112) -- (p113) -- (p123) -- (p122);
		\filldraw[square, fill=\CAG] (p120) -- (p121) -- (p131) -- (p130);
		\filldraw[square, fill=\CAH] (p121) -- (p122) -- (p132) -- (p131);
		\filldraw[square, fill=\CAI] (p122) -- (p123) -- (p133) -- (p132);
		
		\filldraw[square, fill=\CBA] (p200) -- (p201) -- (p211) -- (p210);
		\filldraw[square, fill=\CBB] (p201) -- (p202) -- (p212) -- (p211);
		\filldraw[square, fill=\CBC] (p202) -- (p203) -- (p213) -- (p212);
		\filldraw[square, fill=\CBD] (p210) -- (p211) -- (p221) -- (p220);
		\filldraw[square, fill=\CBE] (p211) -- (p212) -- (p222) -- (p221);
		\filldraw[square, fill=\CBF] (p212) -- (p213) -- (p223) -- (p222);
		\filldraw[square, fill=\CBG] (p220) -- (p221) -- (p231) -- (p230);
		\filldraw[square, fill=\CBH] (p221) -- (p222) -- (p232) -- (p231);
		\filldraw[square, fill=\CBI] (p222) -- (p223) -- (p233) -- (p232);
		
		\filldraw[square, fill=\CCA] (p300) -- (p301) -- (p311) -- (p310);
		\filldraw[square, fill=\CCB] (p301) -- (p302) -- (p312) -- (p311);
		\filldraw[square, fill=\CCC] (p302) -- (p303) -- (p313) -- (p312);
		\filldraw[square, fill=\CCD] (p310) -- (p311) -- (p321) -- (p320);
		\filldraw[square, fill=\CCE] (p311) -- (p312) -- (p322) -- (p321);
		\filldraw[square, fill=\CCF] (p312) -- (p313) -- (p323) -- (p322);
		\filldraw[square, fill=\CCG] (p320) -- (p321) -- (p331) -- (p330);
		\filldraw[square, fill=\CCH] (p321) -- (p322) -- (p332) -- (p331);
		\filldraw[square, fill=\CCI] (p322) -- (p323) -- (p333) -- (p332);
	
    	\draw[side] (p100) -- (p200) -- (p203) -- (p233) -- (p330) -- (p130) -- (p100) -- (p200);
    	\draw[side] (p130) -- (p133) -- (p233);
    	\draw[side] (p103) -- (p133);
    	
    	\draw[square] (p101) -- (p131) -- (p331);
    	\draw[square] (p102) -- (p132) -- (p332);
    	\draw[square] (p201) -- (p231) -- (p310);
    	\draw[square] (p202) -- (p232) -- (p320);
    	\draw[square] (p110) -- (p113) -- (p213);
    	\draw[square] (p120) -- (p123) -- (p223);
	\draw[doubled]($(p112)+0.5*(vh1)$) -- ($(p132)+0.5*(vh1)$) -- ($(p322)+0.5*(vh3)$);
	\end{tikzpicture}
	}    	
\end{subfigure}
\hspace{\S}
\begin{subfigure}[b]{\W\columnwidth}
\centering
	\def\CAA{cwhite}
	\def\CAB{cred}
	\def\CAC{cred}
	\def\CAD{cwhite}
	\def\CAE{cred}
	\def\CAF{cred}
	\def\CAG{cwhite}
	\def\CAH{cred}
	\def\CAI{cred}
	
	\def\CBA{cblue}
	\def\CBB{cblue}
	\def\CBC{cblue}
	\def\CBD{cblue}
	\def\CBE{cblue}
	\def\CBF{cblue}
	\def\CBG{cblue}
	\def\CBH{cblue}
	\def\CBI{cblue}
	
	\def\CCA{corange}
	\def\CCB{cwhite}
	\def\CCC{cwhite}
	\def\CCD{corange}
	\def\CCE{cwhite}
	\def\CCF{cwhite}
	\def\CCG{corange}
	\def\CCH{cwhite}
	\def\CCI{cwhite}
	\caption*{\label{fig:mouvL} \LARGE{$L$}}
	\resizebox{\columnwidth}{!}
	{
		\def\OFFSET{1.4cm}
	\def\OFF{2.5pt}
	\def\RATIO{0.2}
    \def\HXA{2.5cm}
    \def\HYA{-0.5cm}
    \def\VXA{0.0cm}
    \def\VYA{2.55cm}
    \def\HXB{1.5cm}
    \def\HYB{0.85cm}
    \def\VXB{\VXA}
    \def\VYB{\VYA}
    \def\HXC{\HXA}
    \def\HYC{\HYA}
    \def\VXC{\HXB}
    \def\VYC{\HYB}
    \tikzstyle{square}=[line width=3pt, join=round, cap=round]
    \tikzstyle{side}=[line width=5pt, join=round, cap=round]
    \tikzstyle{arrow}=[line width=9pt, join=round, cap=round,->,rounded corners=3cm, cpurple]
    %\tikzstyle{double arrow}=[9pt colored by black and white]
    \tikzstyle{border}=[line width=2pt, join=round, cap=round]
    \begin{tikzpicture}[>=triangle 45]
		\coordinate (vh1) at (\HXA ,\HYA);
		\coordinate (vv1) at (\VXA, \VYA);
		\coordinate (vh2) at (\HXB, \HYB);
		\coordinate (vv2) at (\VXB, \VYB);
		\coordinate (vh3) at (\HXC, \HYC);
		\coordinate (vv3) at (\VXC, \VYC);
		    
    	\coordinate (p100) at (0,0);
    	\coordinate (p101) at ($(p100)+1*(vh1)$);
    	\coordinate (p102) at ($(p100)+2*(vh1)$);
    	\coordinate (p103) at ($(p100)+3*(vh1)$);
    	\coordinate (p110) at ($(p100)+1*(vv1)$);
    	\coordinate (p111) at ($(p110)+1*(vh1)$);
		\coordinate (p112) at ($(p110)+2*(vh1)$);
    	\coordinate (p113) at ($(p110)+3*(vh1)$);
    	\coordinate (p120) at ($(p100)+2*(vv1)$);
    	\coordinate (p121) at ($(p120)+1*(vh1)$);
    	\coordinate (p122) at ($(p120)+2*(vh1)$);
    	\coordinate (p123) at ($(p120)+3*(vh1)$);
    	\coordinate (p130) at ($(p100)+3*(vv1)$);
    	\coordinate (p131) at ($(p130)+1*(vh1)$);
    	\coordinate (p132) at ($(p130)+2*(vh1)$);
    	\coordinate (p133) at ($(p130)+3*(vh1)$);
    	
    	\coordinate (p200) at (p103);
    	\coordinate (p201) at ($(p200)+1*(vh2)$);
    	\coordinate (p202) at ($(p200)+2*(vh2)$);
    	\coordinate (p203) at ($(p200)+3*(vh2)$);
    	\coordinate (p210) at ($(p200)+1*(vv2)$);
    	\coordinate (p211) at ($(p210)+1*(vh2)$);
		\coordinate (p212) at ($(p210)+2*(vh2)$);
    	\coordinate (p213) at ($(p210)+3*(vh2)$);
    	\coordinate (p220) at ($(p200)+2*(vv2)$);
    	\coordinate (p221) at ($(p220)+1*(vh2)$);
    	\coordinate (p222) at ($(p220)+2*(vh2)$);
    	\coordinate (p223) at ($(p220)+3*(vh2)$);
    	\coordinate (p230) at ($(p200)+3*(vv2)$);
    	\coordinate (p231) at ($(p230)+1*(vh2)$);
    	\coordinate (p232) at ($(p230)+2*(vh2)$);
    	\coordinate (p233) at ($(p230)+3*(vh2)$);
    	
		\coordinate (p300) at (p130);
    	\coordinate (p301) at ($(p300)+1*(vh3)$);
    	\coordinate (p302) at ($(p300)+2*(vh3)$);
    	\coordinate (p303) at ($(p300)+3*(vh3)$);
    	\coordinate (p310) at ($(p300)+1*(vv3)$);
    	\coordinate (p311) at ($(p310)+1*(vh3)$);
		\coordinate (p312) at ($(p310)+2*(vh3)$);
    	\coordinate (p313) at ($(p310)+3*(vh3)$);
    	\coordinate (p320) at ($(p300)+2*(vv3)$);
    	\coordinate (p321) at ($(p320)+1*(vh3)$);
    	\coordinate (p322) at ($(p320)+2*(vh3)$);
    	\coordinate (p323) at ($(p320)+3*(vh3)$);
    	\coordinate (p330) at ($(p300)+3*(vv3)$);
    	\coordinate (p331) at ($(p330)+1*(vh3)$);
    	\coordinate (p332) at ($(p330)+2*(vh3)$);
    	\coordinate (p333) at ($(p330)+3*(vh3)$);  	  	
    	
		\filldraw[square, fill=\CAA] (p100) -- (p101) -- (p111) -- (p110);
		\filldraw[square, fill=\CAB] (p101) -- (p102) -- (p112) -- (p111);
		\filldraw[square, fill=\CAC] (p102) -- (p103) -- (p113) -- (p112);
		\filldraw[square, fill=\CAD] (p110) -- (p111) -- (p121) -- (p120);
		\filldraw[square, fill=\CAE] (p111) -- (p112) -- (p122) -- (p121);
		\filldraw[square, fill=\CAF] (p112) -- (p113) -- (p123) -- (p122);
		\filldraw[square, fill=\CAG] (p120) -- (p121) -- (p131) -- (p130);
		\filldraw[square, fill=\CAH] (p121) -- (p122) -- (p132) -- (p131);
		\filldraw[square, fill=\CAI] (p122) -- (p123) -- (p133) -- (p132);
		
		\filldraw[square, fill=\CBA] (p200) -- (p201) -- (p211) -- (p210);
		\filldraw[square, fill=\CBB] (p201) -- (p202) -- (p212) -- (p211);
		\filldraw[square, fill=\CBC] (p202) -- (p203) -- (p213) -- (p212);
		\filldraw[square, fill=\CBD] (p210) -- (p211) -- (p221) -- (p220);
		\filldraw[square, fill=\CBE] (p211) -- (p212) -- (p222) -- (p221);
		\filldraw[square, fill=\CBF] (p212) -- (p213) -- (p223) -- (p222);
		\filldraw[square, fill=\CBG] (p220) -- (p221) -- (p231) -- (p230);
		\filldraw[square, fill=\CBH] (p221) -- (p222) -- (p232) -- (p231);
		\filldraw[square, fill=\CBI] (p222) -- (p223) -- (p233) -- (p232);
		
		\filldraw[square, fill=\CCA] (p300) -- (p301) -- (p311) -- (p310);
		\filldraw[square, fill=\CCB] (p301) -- (p302) -- (p312) -- (p311);
		\filldraw[square, fill=\CCC] (p302) -- (p303) -- (p313) -- (p312);
		\filldraw[square, fill=\CCD] (p310) -- (p311) -- (p321) -- (p320);
		\filldraw[square, fill=\CCE] (p311) -- (p312) -- (p322) -- (p321);
		\filldraw[square, fill=\CCF] (p312) -- (p313) -- (p323) -- (p322);
		\filldraw[square, fill=\CCG] (p320) -- (p321) -- (p331) -- (p330);
		\filldraw[square, fill=\CCH] (p321) -- (p322) -- (p332) -- (p331);
		\filldraw[square, fill=\CCI] (p322) -- (p323) -- (p333) -- (p332);
	
    	\draw[side] (p100) -- (p200) -- (p203) -- (p233) -- (p330) -- (p130) -- (p100) -- (p200);
    	\draw[side] (p130) -- (p133) -- (p233);
    	\draw[side] (p103) -- (p133);
    	
    	\draw[square] (p101) -- (p131) -- (p331);
    	\draw[square] (p102) -- (p132) -- (p332);
    	\draw[square] (p201) -- (p231) -- (p310);
    	\draw[square] (p202) -- (p232) -- (p320);
    	\draw[square] (p110) -- (p113) -- (p213);
    	\draw[square] (p120) -- (p123) -- (p223);
	\draw[doubled]($(p320)+0.5*(vh3)$) -- ($(p130)+0.5*(vh1)$) -- ($(p110)+0.5*(vh1)$);
	\end{tikzpicture}
	}    	
\end{subfigure}
\hspace{\S}
\begin{subfigure}[b]{\W\columnwidth}
\centering
	\def\CAA{cred}
	\def\CAB{cred}
	\def\CAC{cred}
	\def\CAD{cred}
	\def\CAE{cred}
	\def\CAF{cred}
	\def\CAG{cred}
	\def\CAH{cred}
	\def\CAI{cred}
	
	\def\CBA{cwhite}
	\def\CBB{cblue}
	\def\CBC{cblue}
	\def\CBD{cwhite}
	\def\CBE{cblue}
	\def\CBF{cblue}
	\def\CBG{cwhite}
	\def\CBH{cblue}
	\def\CBI{cblue}
	
	\def\CCA{cgreen}
	\def\CCB{cgreen}
	\def\CCC{cgreen}
	\def\CCD{cwhite}
	\def\CCE{cwhite}
	\def\CCF{cwhite}
	\def\CCG{cwhite}
	\def\CCH{cwhite}
	\def\CCI{cwhite}
	\caption*{\label{fig:mouvF} \LARGE{$F$}}
	\resizebox{\columnwidth}{!}
	{
		\def\OFFSET{1.4cm}
	\def\OFF{2.5pt}
	\def\RATIO{0.2}
    \def\HXA{2.5cm}
    \def\HYA{-0.5cm}
    \def\VXA{0.0cm}
    \def\VYA{2.55cm}
    \def\HXB{1.5cm}
    \def\HYB{0.85cm}
    \def\VXB{\VXA}
    \def\VYB{\VYA}
    \def\HXC{\HXA}
    \def\HYC{\HYA}
    \def\VXC{\HXB}
    \def\VYC{\HYB}
    \tikzstyle{square}=[line width=3pt, join=round, cap=round]
    \tikzstyle{side}=[line width=5pt, join=round, cap=round]
    \tikzstyle{arrow}=[line width=9pt, join=round, cap=round,->,rounded corners=3cm, cpurple]
    %\tikzstyle{double arrow}=[9pt colored by black and white]
    \tikzstyle{border}=[line width=2pt, join=round, cap=round]
    \begin{tikzpicture}[>=triangle 45]
		\coordinate (vh1) at (\HXA ,\HYA);
		\coordinate (vv1) at (\VXA, \VYA);
		\coordinate (vh2) at (\HXB, \HYB);
		\coordinate (vv2) at (\VXB, \VYB);
		\coordinate (vh3) at (\HXC, \HYC);
		\coordinate (vv3) at (\VXC, \VYC);
		    
    	\coordinate (p100) at (0,0);
    	\coordinate (p101) at ($(p100)+1*(vh1)$);
    	\coordinate (p102) at ($(p100)+2*(vh1)$);
    	\coordinate (p103) at ($(p100)+3*(vh1)$);
    	\coordinate (p110) at ($(p100)+1*(vv1)$);
    	\coordinate (p111) at ($(p110)+1*(vh1)$);
		\coordinate (p112) at ($(p110)+2*(vh1)$);
    	\coordinate (p113) at ($(p110)+3*(vh1)$);
    	\coordinate (p120) at ($(p100)+2*(vv1)$);
    	\coordinate (p121) at ($(p120)+1*(vh1)$);
    	\coordinate (p122) at ($(p120)+2*(vh1)$);
    	\coordinate (p123) at ($(p120)+3*(vh1)$);
    	\coordinate (p130) at ($(p100)+3*(vv1)$);
    	\coordinate (p131) at ($(p130)+1*(vh1)$);
    	\coordinate (p132) at ($(p130)+2*(vh1)$);
    	\coordinate (p133) at ($(p130)+3*(vh1)$);
    	
    	\coordinate (p200) at (p103);
    	\coordinate (p201) at ($(p200)+1*(vh2)$);
    	\coordinate (p202) at ($(p200)+2*(vh2)$);
    	\coordinate (p203) at ($(p200)+3*(vh2)$);
    	\coordinate (p210) at ($(p200)+1*(vv2)$);
    	\coordinate (p211) at ($(p210)+1*(vh2)$);
		\coordinate (p212) at ($(p210)+2*(vh2)$);
    	\coordinate (p213) at ($(p210)+3*(vh2)$);
    	\coordinate (p220) at ($(p200)+2*(vv2)$);
    	\coordinate (p221) at ($(p220)+1*(vh2)$);
    	\coordinate (p222) at ($(p220)+2*(vh2)$);
    	\coordinate (p223) at ($(p220)+3*(vh2)$);
    	\coordinate (p230) at ($(p200)+3*(vv2)$);
    	\coordinate (p231) at ($(p230)+1*(vh2)$);
    	\coordinate (p232) at ($(p230)+2*(vh2)$);
    	\coordinate (p233) at ($(p230)+3*(vh2)$);
    	
		\coordinate (p300) at (p130);
    	\coordinate (p301) at ($(p300)+1*(vh3)$);
    	\coordinate (p302) at ($(p300)+2*(vh3)$);
    	\coordinate (p303) at ($(p300)+3*(vh3)$);
    	\coordinate (p310) at ($(p300)+1*(vv3)$);
    	\coordinate (p311) at ($(p310)+1*(vh3)$);
		\coordinate (p312) at ($(p310)+2*(vh3)$);
    	\coordinate (p313) at ($(p310)+3*(vh3)$);
    	\coordinate (p320) at ($(p300)+2*(vv3)$);
    	\coordinate (p321) at ($(p320)+1*(vh3)$);
    	\coordinate (p322) at ($(p320)+2*(vh3)$);
    	\coordinate (p323) at ($(p320)+3*(vh3)$);
    	\coordinate (p330) at ($(p300)+3*(vv3)$);
    	\coordinate (p331) at ($(p330)+1*(vh3)$);
    	\coordinate (p332) at ($(p330)+2*(vh3)$);
    	\coordinate (p333) at ($(p330)+3*(vh3)$);  	  	
    	
		\filldraw[square, fill=\CAA] (p100) -- (p101) -- (p111) -- (p110);
		\filldraw[square, fill=\CAB] (p101) -- (p102) -- (p112) -- (p111);
		\filldraw[square, fill=\CAC] (p102) -- (p103) -- (p113) -- (p112);
		\filldraw[square, fill=\CAD] (p110) -- (p111) -- (p121) -- (p120);
		\filldraw[square, fill=\CAE] (p111) -- (p112) -- (p122) -- (p121);
		\filldraw[square, fill=\CAF] (p112) -- (p113) -- (p123) -- (p122);
		\filldraw[square, fill=\CAG] (p120) -- (p121) -- (p131) -- (p130);
		\filldraw[square, fill=\CAH] (p121) -- (p122) -- (p132) -- (p131);
		\filldraw[square, fill=\CAI] (p122) -- (p123) -- (p133) -- (p132);
		
		\filldraw[square, fill=\CBA] (p200) -- (p201) -- (p211) -- (p210);
		\filldraw[square, fill=\CBB] (p201) -- (p202) -- (p212) -- (p211);
		\filldraw[square, fill=\CBC] (p202) -- (p203) -- (p213) -- (p212);
		\filldraw[square, fill=\CBD] (p210) -- (p211) -- (p221) -- (p220);
		\filldraw[square, fill=\CBE] (p211) -- (p212) -- (p222) -- (p221);
		\filldraw[square, fill=\CBF] (p212) -- (p213) -- (p223) -- (p222);
		\filldraw[square, fill=\CBG] (p220) -- (p221) -- (p231) -- (p230);
		\filldraw[square, fill=\CBH] (p221) -- (p222) -- (p232) -- (p231);
		\filldraw[square, fill=\CBI] (p222) -- (p223) -- (p233) -- (p232);
		
		\filldraw[square, fill=\CCA] (p300) -- (p301) -- (p311) -- (p310);
		\filldraw[square, fill=\CCB] (p301) -- (p302) -- (p312) -- (p311);
		\filldraw[square, fill=\CCC] (p302) -- (p303) -- (p313) -- (p312);
		\filldraw[square, fill=\CCD] (p310) -- (p311) -- (p321) -- (p320);
		\filldraw[square, fill=\CCE] (p311) -- (p312) -- (p322) -- (p321);
		\filldraw[square, fill=\CCF] (p312) -- (p313) -- (p323) -- (p322);
		\filldraw[square, fill=\CCG] (p320) -- (p321) -- (p331) -- (p330);
		\filldraw[square, fill=\CCH] (p321) -- (p322) -- (p332) -- (p331);
		\filldraw[square, fill=\CCI] (p322) -- (p323) -- (p333) -- (p332);
	
    	\draw[side] (p100) -- (p200) -- (p203) -- (p233) -- (p330) -- (p130) -- (p100) -- (p200);
    	\draw[side] (p130) -- (p133) -- (p233);
    	\draw[side] (p103) -- (p133);
    	
    	\draw[square] (p101) -- (p131) -- (p331);
    	\draw[square] (p102) -- (p132) -- (p332);
    	\draw[square] (p201) -- (p231) -- (p310);
    	\draw[square] (p202) -- (p232) -- (p320);
    	\draw[square] (p110) -- (p113) -- (p213);
    	\draw[square] (p120) -- (p123) -- (p223);
	\draw[doubled]($(p301)+0.5*(vv3)$) -- ($(p303)+0.5*(vv3)$) -- ($(p210)+0.5*(vh2)$);
	\end{tikzpicture}
	}    
\end{subfigure}
\hspace{\S}
\begin{subfigure}[b]{\W\columnwidth}
\centering
	\def\CAA{cgreen}
	\def\CAB{cgreen}
	\def\CAC{cgreen}
	\def\CAD{cred}
	\def\CAE{cred}
	\def\CAF{cred}
	\def\CAG{cred}
	\def\CAH{cred}
	\def\CAI{cred}
	
	\def\CBA{cred}
	\def\CBB{cred}
	\def\CBC{cred}
	\def\CBD{cblue}
	\def\CBE{cblue}
	\def\CBF{cblue}
	\def\CBG{cblue}
	\def\CBH{cblue}
	\def\CBI{cblue}
	
	\def\CCA{cwhite}
	\def\CCB{cwhite}
	\def\CCC{cwhite}
	\def\CCD{cwhite}
	\def\CCE{cwhite}
	\def\CCF{cwhite}
	\def\CCG{cwhite}
	\def\CCH{cwhite}
	\def\CCI{cwhite}
	\caption*{\label{fig:mouvD} \LARGE{$D$}}
	\resizebox{\columnwidth}{!}
	{
		\def\OFFSET{1.4cm}
	\def\OFF{2.5pt}
	\def\RATIO{0.2}
    \def\HXA{2.5cm}
    \def\HYA{-0.5cm}
    \def\VXA{0.0cm}
    \def\VYA{2.55cm}
    \def\HXB{1.5cm}
    \def\HYB{0.85cm}
    \def\VXB{\VXA}
    \def\VYB{\VYA}
    \def\HXC{\HXA}
    \def\HYC{\HYA}
    \def\VXC{\HXB}
    \def\VYC{\HYB}
    \tikzstyle{square}=[line width=3pt, join=round, cap=round]
    \tikzstyle{side}=[line width=5pt, join=round, cap=round]
    \tikzstyle{arrow}=[line width=9pt, join=round, cap=round,->,rounded corners=3cm, cpurple]
    %\tikzstyle{double arrow}=[9pt colored by black and white]
    \tikzstyle{border}=[line width=2pt, join=round, cap=round]
    \begin{tikzpicture}[>=triangle 45]
		\coordinate (vh1) at (\HXA ,\HYA);
		\coordinate (vv1) at (\VXA, \VYA);
		\coordinate (vh2) at (\HXB, \HYB);
		\coordinate (vv2) at (\VXB, \VYB);
		\coordinate (vh3) at (\HXC, \HYC);
		\coordinate (vv3) at (\VXC, \VYC);
		    
    	\coordinate (p100) at (0,0);
    	\coordinate (p101) at ($(p100)+1*(vh1)$);
    	\coordinate (p102) at ($(p100)+2*(vh1)$);
    	\coordinate (p103) at ($(p100)+3*(vh1)$);
    	\coordinate (p110) at ($(p100)+1*(vv1)$);
    	\coordinate (p111) at ($(p110)+1*(vh1)$);
		\coordinate (p112) at ($(p110)+2*(vh1)$);
    	\coordinate (p113) at ($(p110)+3*(vh1)$);
    	\coordinate (p120) at ($(p100)+2*(vv1)$);
    	\coordinate (p121) at ($(p120)+1*(vh1)$);
    	\coordinate (p122) at ($(p120)+2*(vh1)$);
    	\coordinate (p123) at ($(p120)+3*(vh1)$);
    	\coordinate (p130) at ($(p100)+3*(vv1)$);
    	\coordinate (p131) at ($(p130)+1*(vh1)$);
    	\coordinate (p132) at ($(p130)+2*(vh1)$);
    	\coordinate (p133) at ($(p130)+3*(vh1)$);
    	
    	\coordinate (p200) at (p103);
    	\coordinate (p201) at ($(p200)+1*(vh2)$);
    	\coordinate (p202) at ($(p200)+2*(vh2)$);
    	\coordinate (p203) at ($(p200)+3*(vh2)$);
    	\coordinate (p210) at ($(p200)+1*(vv2)$);
    	\coordinate (p211) at ($(p210)+1*(vh2)$);
		\coordinate (p212) at ($(p210)+2*(vh2)$);
    	\coordinate (p213) at ($(p210)+3*(vh2)$);
    	\coordinate (p220) at ($(p200)+2*(vv2)$);
    	\coordinate (p221) at ($(p220)+1*(vh2)$);
    	\coordinate (p222) at ($(p220)+2*(vh2)$);
    	\coordinate (p223) at ($(p220)+3*(vh2)$);
    	\coordinate (p230) at ($(p200)+3*(vv2)$);
    	\coordinate (p231) at ($(p230)+1*(vh2)$);
    	\coordinate (p232) at ($(p230)+2*(vh2)$);
    	\coordinate (p233) at ($(p230)+3*(vh2)$);
    	
		\coordinate (p300) at (p130);
    	\coordinate (p301) at ($(p300)+1*(vh3)$);
    	\coordinate (p302) at ($(p300)+2*(vh3)$);
    	\coordinate (p303) at ($(p300)+3*(vh3)$);
    	\coordinate (p310) at ($(p300)+1*(vv3)$);
    	\coordinate (p311) at ($(p310)+1*(vh3)$);
		\coordinate (p312) at ($(p310)+2*(vh3)$);
    	\coordinate (p313) at ($(p310)+3*(vh3)$);
    	\coordinate (p320) at ($(p300)+2*(vv3)$);
    	\coordinate (p321) at ($(p320)+1*(vh3)$);
    	\coordinate (p322) at ($(p320)+2*(vh3)$);
    	\coordinate (p323) at ($(p320)+3*(vh3)$);
    	\coordinate (p330) at ($(p300)+3*(vv3)$);
    	\coordinate (p331) at ($(p330)+1*(vh3)$);
    	\coordinate (p332) at ($(p330)+2*(vh3)$);
    	\coordinate (p333) at ($(p330)+3*(vh3)$);  	  	
    	
		\filldraw[square, fill=\CAA] (p100) -- (p101) -- (p111) -- (p110);
		\filldraw[square, fill=\CAB] (p101) -- (p102) -- (p112) -- (p111);
		\filldraw[square, fill=\CAC] (p102) -- (p103) -- (p113) -- (p112);
		\filldraw[square, fill=\CAD] (p110) -- (p111) -- (p121) -- (p120);
		\filldraw[square, fill=\CAE] (p111) -- (p112) -- (p122) -- (p121);
		\filldraw[square, fill=\CAF] (p112) -- (p113) -- (p123) -- (p122);
		\filldraw[square, fill=\CAG] (p120) -- (p121) -- (p131) -- (p130);
		\filldraw[square, fill=\CAH] (p121) -- (p122) -- (p132) -- (p131);
		\filldraw[square, fill=\CAI] (p122) -- (p123) -- (p133) -- (p132);
		
		\filldraw[square, fill=\CBA] (p200) -- (p201) -- (p211) -- (p210);
		\filldraw[square, fill=\CBB] (p201) -- (p202) -- (p212) -- (p211);
		\filldraw[square, fill=\CBC] (p202) -- (p203) -- (p213) -- (p212);
		\filldraw[square, fill=\CBD] (p210) -- (p211) -- (p221) -- (p220);
		\filldraw[square, fill=\CBE] (p211) -- (p212) -- (p222) -- (p221);
		\filldraw[square, fill=\CBF] (p212) -- (p213) -- (p223) -- (p222);
		\filldraw[square, fill=\CBG] (p220) -- (p221) -- (p231) -- (p230);
		\filldraw[square, fill=\CBH] (p221) -- (p222) -- (p232) -- (p231);
		\filldraw[square, fill=\CBI] (p222) -- (p223) -- (p233) -- (p232);
		
		\filldraw[square, fill=\CCA] (p300) -- (p301) -- (p311) -- (p310);
		\filldraw[square, fill=\CCB] (p301) -- (p302) -- (p312) -- (p311);
		\filldraw[square, fill=\CCC] (p302) -- (p303) -- (p313) -- (p312);
		\filldraw[square, fill=\CCD] (p310) -- (p311) -- (p321) -- (p320);
		\filldraw[square, fill=\CCE] (p311) -- (p312) -- (p322) -- (p321);
		\filldraw[square, fill=\CCF] (p312) -- (p313) -- (p323) -- (p322);
		\filldraw[square, fill=\CCG] (p320) -- (p321) -- (p331) -- (p330);
		\filldraw[square, fill=\CCH] (p321) -- (p322) -- (p332) -- (p331);
		\filldraw[square, fill=\CCI] (p322) -- (p323) -- (p333) -- (p332);
	
    	\draw[side] (p100) -- (p200) -- (p203) -- (p233) -- (p330) -- (p130) -- (p100) -- (p200);
    	\draw[side] (p130) -- (p133) -- (p233);
    	\draw[side] (p103) -- (p133);
    	
    	\draw[square] (p101) -- (p131) -- (p331);
    	\draw[square] (p102) -- (p132) -- (p332);
    	\draw[square] (p201) -- (p231) -- (p310);
    	\draw[square] (p202) -- (p232) -- (p320);
    	\draw[square] (p110) -- (p113) -- (p213);
    	\draw[square] (p120) -- (p123) -- (p223);
	\draw[doubled]($(p101)+0.5*(vv1)$) -- ($(p103)+0.5*(vv1)$) -- ($(p202)+0.5*(vv2)$);
	\end{tikzpicture}
	}    
\end{subfigure}
\hspace{\S}
\begin{subfigure}[b]{\W\columnwidth}
\centering
	\def\CAA{cred}
	\def\CAB{cred}
	\def\CAC{cred}
	\def\CAD{cred}
	\def\CAE{cred}
	\def\CAF{cred}
	\def\CAG{cblue}
	\def\CAH{cblue}
	\def\CAI{cblue}
	
	\def\CBA{cblue}
	\def\CBB{cblue}
	\def\CBC{cblue}
	\def\CBD{cblue}
	\def\CBE{cblue}
	\def\CBF{cblue}
	\def\CBG{corange}
	\def\CBH{corange}
	\def\CBI{corange}
	
	\def\CCA{cwhite}
	\def\CCB{cwhite}
	\def\CCC{cwhite}
	\def\CCD{cwhite}
	\def\CCE{cwhite}
	\def\CCF{cwhite}
	\def\CCG{cwhite}
	\def\CCH{cwhite}
	\def\CCI{cwhite}
	\caption*{\label{fig:mouvU} \LARGE{$U$}}
	\resizebox{\columnwidth}{!}
	{
		\def\OFFSET{1.4cm}
	\def\OFF{2.5pt}
	\def\RATIO{0.2}
    \def\HXA{2.5cm}
    \def\HYA{-0.5cm}
    \def\VXA{0.0cm}
    \def\VYA{2.55cm}
    \def\HXB{1.5cm}
    \def\HYB{0.85cm}
    \def\VXB{\VXA}
    \def\VYB{\VYA}
    \def\HXC{\HXA}
    \def\HYC{\HYA}
    \def\VXC{\HXB}
    \def\VYC{\HYB}
    \tikzstyle{square}=[line width=3pt, join=round, cap=round]
    \tikzstyle{side}=[line width=5pt, join=round, cap=round]
    \tikzstyle{arrow}=[line width=9pt, join=round, cap=round,->,rounded corners=3cm, cpurple]
    %\tikzstyle{double arrow}=[9pt colored by black and white]
    \tikzstyle{border}=[line width=2pt, join=round, cap=round]
    \begin{tikzpicture}[>=triangle 45]
		\coordinate (vh1) at (\HXA ,\HYA);
		\coordinate (vv1) at (\VXA, \VYA);
		\coordinate (vh2) at (\HXB, \HYB);
		\coordinate (vv2) at (\VXB, \VYB);
		\coordinate (vh3) at (\HXC, \HYC);
		\coordinate (vv3) at (\VXC, \VYC);
		    
    	\coordinate (p100) at (0,0);
    	\coordinate (p101) at ($(p100)+1*(vh1)$);
    	\coordinate (p102) at ($(p100)+2*(vh1)$);
    	\coordinate (p103) at ($(p100)+3*(vh1)$);
    	\coordinate (p110) at ($(p100)+1*(vv1)$);
    	\coordinate (p111) at ($(p110)+1*(vh1)$);
		\coordinate (p112) at ($(p110)+2*(vh1)$);
    	\coordinate (p113) at ($(p110)+3*(vh1)$);
    	\coordinate (p120) at ($(p100)+2*(vv1)$);
    	\coordinate (p121) at ($(p120)+1*(vh1)$);
    	\coordinate (p122) at ($(p120)+2*(vh1)$);
    	\coordinate (p123) at ($(p120)+3*(vh1)$);
    	\coordinate (p130) at ($(p100)+3*(vv1)$);
    	\coordinate (p131) at ($(p130)+1*(vh1)$);
    	\coordinate (p132) at ($(p130)+2*(vh1)$);
    	\coordinate (p133) at ($(p130)+3*(vh1)$);
    	
    	\coordinate (p200) at (p103);
    	\coordinate (p201) at ($(p200)+1*(vh2)$);
    	\coordinate (p202) at ($(p200)+2*(vh2)$);
    	\coordinate (p203) at ($(p200)+3*(vh2)$);
    	\coordinate (p210) at ($(p200)+1*(vv2)$);
    	\coordinate (p211) at ($(p210)+1*(vh2)$);
		\coordinate (p212) at ($(p210)+2*(vh2)$);
    	\coordinate (p213) at ($(p210)+3*(vh2)$);
    	\coordinate (p220) at ($(p200)+2*(vv2)$);
    	\coordinate (p221) at ($(p220)+1*(vh2)$);
    	\coordinate (p222) at ($(p220)+2*(vh2)$);
    	\coordinate (p223) at ($(p220)+3*(vh2)$);
    	\coordinate (p230) at ($(p200)+3*(vv2)$);
    	\coordinate (p231) at ($(p230)+1*(vh2)$);
    	\coordinate (p232) at ($(p230)+2*(vh2)$);
    	\coordinate (p233) at ($(p230)+3*(vh2)$);
    	
		\coordinate (p300) at (p130);
    	\coordinate (p301) at ($(p300)+1*(vh3)$);
    	\coordinate (p302) at ($(p300)+2*(vh3)$);
    	\coordinate (p303) at ($(p300)+3*(vh3)$);
    	\coordinate (p310) at ($(p300)+1*(vv3)$);
    	\coordinate (p311) at ($(p310)+1*(vh3)$);
		\coordinate (p312) at ($(p310)+2*(vh3)$);
    	\coordinate (p313) at ($(p310)+3*(vh3)$);
    	\coordinate (p320) at ($(p300)+2*(vv3)$);
    	\coordinate (p321) at ($(p320)+1*(vh3)$);
    	\coordinate (p322) at ($(p320)+2*(vh3)$);
    	\coordinate (p323) at ($(p320)+3*(vh3)$);
    	\coordinate (p330) at ($(p300)+3*(vv3)$);
    	\coordinate (p331) at ($(p330)+1*(vh3)$);
    	\coordinate (p332) at ($(p330)+2*(vh3)$);
    	\coordinate (p333) at ($(p330)+3*(vh3)$);  	  	
    	
		\filldraw[square, fill=\CAA] (p100) -- (p101) -- (p111) -- (p110);
		\filldraw[square, fill=\CAB] (p101) -- (p102) -- (p112) -- (p111);
		\filldraw[square, fill=\CAC] (p102) -- (p103) -- (p113) -- (p112);
		\filldraw[square, fill=\CAD] (p110) -- (p111) -- (p121) -- (p120);
		\filldraw[square, fill=\CAE] (p111) -- (p112) -- (p122) -- (p121);
		\filldraw[square, fill=\CAF] (p112) -- (p113) -- (p123) -- (p122);
		\filldraw[square, fill=\CAG] (p120) -- (p121) -- (p131) -- (p130);
		\filldraw[square, fill=\CAH] (p121) -- (p122) -- (p132) -- (p131);
		\filldraw[square, fill=\CAI] (p122) -- (p123) -- (p133) -- (p132);
		
		\filldraw[square, fill=\CBA] (p200) -- (p201) -- (p211) -- (p210);
		\filldraw[square, fill=\CBB] (p201) -- (p202) -- (p212) -- (p211);
		\filldraw[square, fill=\CBC] (p202) -- (p203) -- (p213) -- (p212);
		\filldraw[square, fill=\CBD] (p210) -- (p211) -- (p221) -- (p220);
		\filldraw[square, fill=\CBE] (p211) -- (p212) -- (p222) -- (p221);
		\filldraw[square, fill=\CBF] (p212) -- (p213) -- (p223) -- (p222);
		\filldraw[square, fill=\CBG] (p220) -- (p221) -- (p231) -- (p230);
		\filldraw[square, fill=\CBH] (p221) -- (p222) -- (p232) -- (p231);
		\filldraw[square, fill=\CBI] (p222) -- (p223) -- (p233) -- (p232);
		
		\filldraw[square, fill=\CCA] (p300) -- (p301) -- (p311) -- (p310);
		\filldraw[square, fill=\CCB] (p301) -- (p302) -- (p312) -- (p311);
		\filldraw[square, fill=\CCC] (p302) -- (p303) -- (p313) -- (p312);
		\filldraw[square, fill=\CCD] (p310) -- (p311) -- (p321) -- (p320);
		\filldraw[square, fill=\CCE] (p311) -- (p312) -- (p322) -- (p321);
		\filldraw[square, fill=\CCF] (p312) -- (p313) -- (p323) -- (p322);
		\filldraw[square, fill=\CCG] (p320) -- (p321) -- (p331) -- (p330);
		\filldraw[square, fill=\CCH] (p321) -- (p322) -- (p332) -- (p331);
		\filldraw[square, fill=\CCI] (p322) -- (p323) -- (p333) -- (p332);
	
    	\draw[side] (p100) -- (p200) -- (p203) -- (p233) -- (p330) -- (p130) -- (p100) -- (p200);
    	\draw[side] (p130) -- (p133) -- (p233);
    	\draw[side] (p103) -- (p133);
    	
    	\draw[square] (p101) -- (p131) -- (p331);
    	\draw[square] (p102) -- (p132) -- (p332);
    	\draw[square] (p201) -- (p231) -- (p310);
    	\draw[square] (p202) -- (p232) -- (p320);
    	\draw[square] (p110) -- (p113) -- (p213);
    	\draw[square] (p120) -- (p123) -- (p223);
	\draw[doubled]($(p222)+0.5*(vv2)$) -- ($(p123)+0.5*(vv1)$) -- ($(p121)+0.5*(vv1)$);
	\end{tikzpicture}
	}    
\end{subfigure}
\hspace{\S}
\begin{subfigure}[b]{\W\columnwidth}
\centering
	\def\CAA{cred}
	\def\CAB{cred}
	\def\CAC{cred}
	\def\CAD{cred}
	\def\CAE{cred}
	\def\CAF{cred}
	\def\CAG{cred}
	\def\CAH{cred}
	\def\CAI{cred}
	
	\def\CBA{cblue}
	\def\CBB{cblue}
	\def\CBC{cyellow}
	\def\CBD{cblue}
	\def\CBE{cblue}
	\def\CBF{cyellow}
	\def\CBG{cblue}
	\def\CBH{cblue}
	\def\CBI{cyellow}
	
	\def\CCA{cwhite}
	\def\CCB{cwhite}
	\def\CCC{cwhite}
	\def\CCD{cwhite}
	\def\CCE{cwhite}
	\def\CCF{cwhite}
	\def\CCG{cblue}
	\def\CCH{cblue}
	\def\CCI{cblue}
	\caption*{\label{fig:mouvB} \LARGE{$B$}} 
	\resizebox{\columnwidth}{!}
	{
		\def\OFFSET{1.4cm}
	\def\OFF{2.5pt}
	\def\RATIO{0.2}
    \def\HXA{2.5cm}
    \def\HYA{-0.5cm}
    \def\VXA{0.0cm}
    \def\VYA{2.55cm}
    \def\HXB{1.5cm}
    \def\HYB{0.85cm}
    \def\VXB{\VXA}
    \def\VYB{\VYA}
    \def\HXC{\HXA}
    \def\HYC{\HYA}
    \def\VXC{\HXB}
    \def\VYC{\HYB}
    \tikzstyle{square}=[line width=3pt, join=round, cap=round]
    \tikzstyle{side}=[line width=5pt, join=round, cap=round]
    \tikzstyle{arrow}=[line width=9pt, join=round, cap=round,->,rounded corners=3cm, cpurple]
    %\tikzstyle{double arrow}=[9pt colored by black and white]
    \tikzstyle{border}=[line width=2pt, join=round, cap=round]
    \begin{tikzpicture}[>=triangle 45]
		\coordinate (vh1) at (\HXA ,\HYA);
		\coordinate (vv1) at (\VXA, \VYA);
		\coordinate (vh2) at (\HXB, \HYB);
		\coordinate (vv2) at (\VXB, \VYB);
		\coordinate (vh3) at (\HXC, \HYC);
		\coordinate (vv3) at (\VXC, \VYC);
		    
    	\coordinate (p100) at (0,0);
    	\coordinate (p101) at ($(p100)+1*(vh1)$);
    	\coordinate (p102) at ($(p100)+2*(vh1)$);
    	\coordinate (p103) at ($(p100)+3*(vh1)$);
    	\coordinate (p110) at ($(p100)+1*(vv1)$);
    	\coordinate (p111) at ($(p110)+1*(vh1)$);
		\coordinate (p112) at ($(p110)+2*(vh1)$);
    	\coordinate (p113) at ($(p110)+3*(vh1)$);
    	\coordinate (p120) at ($(p100)+2*(vv1)$);
    	\coordinate (p121) at ($(p120)+1*(vh1)$);
    	\coordinate (p122) at ($(p120)+2*(vh1)$);
    	\coordinate (p123) at ($(p120)+3*(vh1)$);
    	\coordinate (p130) at ($(p100)+3*(vv1)$);
    	\coordinate (p131) at ($(p130)+1*(vh1)$);
    	\coordinate (p132) at ($(p130)+2*(vh1)$);
    	\coordinate (p133) at ($(p130)+3*(vh1)$);
    	
    	\coordinate (p200) at (p103);
    	\coordinate (p201) at ($(p200)+1*(vh2)$);
    	\coordinate (p202) at ($(p200)+2*(vh2)$);
    	\coordinate (p203) at ($(p200)+3*(vh2)$);
    	\coordinate (p210) at ($(p200)+1*(vv2)$);
    	\coordinate (p211) at ($(p210)+1*(vh2)$);
		\coordinate (p212) at ($(p210)+2*(vh2)$);
    	\coordinate (p213) at ($(p210)+3*(vh2)$);
    	\coordinate (p220) at ($(p200)+2*(vv2)$);
    	\coordinate (p221) at ($(p220)+1*(vh2)$);
    	\coordinate (p222) at ($(p220)+2*(vh2)$);
    	\coordinate (p223) at ($(p220)+3*(vh2)$);
    	\coordinate (p230) at ($(p200)+3*(vv2)$);
    	\coordinate (p231) at ($(p230)+1*(vh2)$);
    	\coordinate (p232) at ($(p230)+2*(vh2)$);
    	\coordinate (p233) at ($(p230)+3*(vh2)$);
    	
		\coordinate (p300) at (p130);
    	\coordinate (p301) at ($(p300)+1*(vh3)$);
    	\coordinate (p302) at ($(p300)+2*(vh3)$);
    	\coordinate (p303) at ($(p300)+3*(vh3)$);
    	\coordinate (p310) at ($(p300)+1*(vv3)$);
    	\coordinate (p311) at ($(p310)+1*(vh3)$);
		\coordinate (p312) at ($(p310)+2*(vh3)$);
    	\coordinate (p313) at ($(p310)+3*(vh3)$);
    	\coordinate (p320) at ($(p300)+2*(vv3)$);
    	\coordinate (p321) at ($(p320)+1*(vh3)$);
    	\coordinate (p322) at ($(p320)+2*(vh3)$);
    	\coordinate (p323) at ($(p320)+3*(vh3)$);
    	\coordinate (p330) at ($(p300)+3*(vv3)$);
    	\coordinate (p331) at ($(p330)+1*(vh3)$);
    	\coordinate (p332) at ($(p330)+2*(vh3)$);
    	\coordinate (p333) at ($(p330)+3*(vh3)$);  	  	
    	
		\filldraw[square, fill=\CAA] (p100) -- (p101) -- (p111) -- (p110);
		\filldraw[square, fill=\CAB] (p101) -- (p102) -- (p112) -- (p111);
		\filldraw[square, fill=\CAC] (p102) -- (p103) -- (p113) -- (p112);
		\filldraw[square, fill=\CAD] (p110) -- (p111) -- (p121) -- (p120);
		\filldraw[square, fill=\CAE] (p111) -- (p112) -- (p122) -- (p121);
		\filldraw[square, fill=\CAF] (p112) -- (p113) -- (p123) -- (p122);
		\filldraw[square, fill=\CAG] (p120) -- (p121) -- (p131) -- (p130);
		\filldraw[square, fill=\CAH] (p121) -- (p122) -- (p132) -- (p131);
		\filldraw[square, fill=\CAI] (p122) -- (p123) -- (p133) -- (p132);
		
		\filldraw[square, fill=\CBA] (p200) -- (p201) -- (p211) -- (p210);
		\filldraw[square, fill=\CBB] (p201) -- (p202) -- (p212) -- (p211);
		\filldraw[square, fill=\CBC] (p202) -- (p203) -- (p213) -- (p212);
		\filldraw[square, fill=\CBD] (p210) -- (p211) -- (p221) -- (p220);
		\filldraw[square, fill=\CBE] (p211) -- (p212) -- (p222) -- (p221);
		\filldraw[square, fill=\CBF] (p212) -- (p213) -- (p223) -- (p222);
		\filldraw[square, fill=\CBG] (p220) -- (p221) -- (p231) -- (p230);
		\filldraw[square, fill=\CBH] (p221) -- (p222) -- (p232) -- (p231);
		\filldraw[square, fill=\CBI] (p222) -- (p223) -- (p233) -- (p232);
		
		\filldraw[square, fill=\CCA] (p300) -- (p301) -- (p311) -- (p310);
		\filldraw[square, fill=\CCB] (p301) -- (p302) -- (p312) -- (p311);
		\filldraw[square, fill=\CCC] (p302) -- (p303) -- (p313) -- (p312);
		\filldraw[square, fill=\CCD] (p310) -- (p311) -- (p321) -- (p320);
		\filldraw[square, fill=\CCE] (p311) -- (p312) -- (p322) -- (p321);
		\filldraw[square, fill=\CCF] (p312) -- (p313) -- (p323) -- (p322);
		\filldraw[square, fill=\CCG] (p320) -- (p321) -- (p331) -- (p330);
		\filldraw[square, fill=\CCH] (p321) -- (p322) -- (p332) -- (p331);
		\filldraw[square, fill=\CCI] (p322) -- (p323) -- (p333) -- (p332);
	
    	\draw[side] (p100) -- (p200) -- (p203) -- (p233) -- (p330) -- (p130) -- (p100) -- (p200);
    	\draw[side] (p130) -- (p133) -- (p233);
    	\draw[side] (p103) -- (p133);
    	
    	\draw[square] (p101) -- (p131) -- (p331);
    	\draw[square] (p102) -- (p132) -- (p332);
    	\draw[square] (p201) -- (p231) -- (p310);
    	\draw[square] (p202) -- (p232) -- (p320);
    	\draw[square] (p110) -- (p113) -- (p213);
    	\draw[square] (p120) -- (p123) -- (p223);
	\draw[doubled]($(p212)+0.5*(vh2)$) -- ($(p232)+0.5*(vh2)$) -- ($(p321)+0.5*(vv3)$);
	\end{tikzpicture}
	}   
\end{subfigure}
\\
\begin{subfigure}[b]{\W\columnwidth}
\centering
	\def\CAA{cred}
	\def\CAB{cred}
	\def\CAC{cwhite}
	\def\CAD{cred}
	\def\CAE{cred}
	\def\CAF{cwhite}
	\def\CAG{cred}
	\def\CAH{cred}
	\def\CAI{cwhite}
	
	\def\CBA{cblue}
	\def\CBB{cblue}
	\def\CBC{cblue}
	\def\CBD{cblue}
	\def\CBE{cblue}
	\def\CBF{cblue}
	\def\CBG{cblue}
	\def\CBH{cblue}
	\def\CBI{cblue}
	
	\def\CCA{cwhite}
	\def\CCB{cwhite}
	\def\CCC{corange}
	\def\CCD{cwhite}
	\def\CCE{cwhite}
	\def\CCF{corange}
	\def\CCG{cwhite}
	\def\CCH{cwhite}
	\def\CCI{corange}
	\resizebox{\columnwidth}{!}
	{
		\def\OFFSET{1.4cm}
	\def\OFF{2.5pt}
	\def\RATIO{0.2}
    \def\HXA{2.5cm}
    \def\HYA{-0.5cm}
    \def\VXA{0.0cm}
    \def\VYA{2.55cm}
    \def\HXB{1.5cm}
    \def\HYB{0.85cm}
    \def\VXB{\VXA}
    \def\VYB{\VYA}
    \def\HXC{\HXA}
    \def\HYC{\HYA}
    \def\VXC{\HXB}
    \def\VYC{\HYB}
    \tikzstyle{square}=[line width=3pt, join=round, cap=round]
    \tikzstyle{side}=[line width=5pt, join=round, cap=round]
    \tikzstyle{arrow}=[line width=9pt, join=round, cap=round,->,rounded corners=3cm, cpurple]
    %\tikzstyle{double arrow}=[9pt colored by black and white]
    \tikzstyle{border}=[line width=2pt, join=round, cap=round]
    \begin{tikzpicture}[>=triangle 45]
		\coordinate (vh1) at (\HXA ,\HYA);
		\coordinate (vv1) at (\VXA, \VYA);
		\coordinate (vh2) at (\HXB, \HYB);
		\coordinate (vv2) at (\VXB, \VYB);
		\coordinate (vh3) at (\HXC, \HYC);
		\coordinate (vv3) at (\VXC, \VYC);
		    
    	\coordinate (p100) at (0,0);
    	\coordinate (p101) at ($(p100)+1*(vh1)$);
    	\coordinate (p102) at ($(p100)+2*(vh1)$);
    	\coordinate (p103) at ($(p100)+3*(vh1)$);
    	\coordinate (p110) at ($(p100)+1*(vv1)$);
    	\coordinate (p111) at ($(p110)+1*(vh1)$);
		\coordinate (p112) at ($(p110)+2*(vh1)$);
    	\coordinate (p113) at ($(p110)+3*(vh1)$);
    	\coordinate (p120) at ($(p100)+2*(vv1)$);
    	\coordinate (p121) at ($(p120)+1*(vh1)$);
    	\coordinate (p122) at ($(p120)+2*(vh1)$);
    	\coordinate (p123) at ($(p120)+3*(vh1)$);
    	\coordinate (p130) at ($(p100)+3*(vv1)$);
    	\coordinate (p131) at ($(p130)+1*(vh1)$);
    	\coordinate (p132) at ($(p130)+2*(vh1)$);
    	\coordinate (p133) at ($(p130)+3*(vh1)$);
    	
    	\coordinate (p200) at (p103);
    	\coordinate (p201) at ($(p200)+1*(vh2)$);
    	\coordinate (p202) at ($(p200)+2*(vh2)$);
    	\coordinate (p203) at ($(p200)+3*(vh2)$);
    	\coordinate (p210) at ($(p200)+1*(vv2)$);
    	\coordinate (p211) at ($(p210)+1*(vh2)$);
		\coordinate (p212) at ($(p210)+2*(vh2)$);
    	\coordinate (p213) at ($(p210)+3*(vh2)$);
    	\coordinate (p220) at ($(p200)+2*(vv2)$);
    	\coordinate (p221) at ($(p220)+1*(vh2)$);
    	\coordinate (p222) at ($(p220)+2*(vh2)$);
    	\coordinate (p223) at ($(p220)+3*(vh2)$);
    	\coordinate (p230) at ($(p200)+3*(vv2)$);
    	\coordinate (p231) at ($(p230)+1*(vh2)$);
    	\coordinate (p232) at ($(p230)+2*(vh2)$);
    	\coordinate (p233) at ($(p230)+3*(vh2)$);
    	
		\coordinate (p300) at (p130);
    	\coordinate (p301) at ($(p300)+1*(vh3)$);
    	\coordinate (p302) at ($(p300)+2*(vh3)$);
    	\coordinate (p303) at ($(p300)+3*(vh3)$);
    	\coordinate (p310) at ($(p300)+1*(vv3)$);
    	\coordinate (p311) at ($(p310)+1*(vh3)$);
		\coordinate (p312) at ($(p310)+2*(vh3)$);
    	\coordinate (p313) at ($(p310)+3*(vh3)$);
    	\coordinate (p320) at ($(p300)+2*(vv3)$);
    	\coordinate (p321) at ($(p320)+1*(vh3)$);
    	\coordinate (p322) at ($(p320)+2*(vh3)$);
    	\coordinate (p323) at ($(p320)+3*(vh3)$);
    	\coordinate (p330) at ($(p300)+3*(vv3)$);
    	\coordinate (p331) at ($(p330)+1*(vh3)$);
    	\coordinate (p332) at ($(p330)+2*(vh3)$);
    	\coordinate (p333) at ($(p330)+3*(vh3)$);  	  	
    	
		\filldraw[square, fill=\CAA] (p100) -- (p101) -- (p111) -- (p110);
		\filldraw[square, fill=\CAB] (p101) -- (p102) -- (p112) -- (p111);
		\filldraw[square, fill=\CAC] (p102) -- (p103) -- (p113) -- (p112);
		\filldraw[square, fill=\CAD] (p110) -- (p111) -- (p121) -- (p120);
		\filldraw[square, fill=\CAE] (p111) -- (p112) -- (p122) -- (p121);
		\filldraw[square, fill=\CAF] (p112) -- (p113) -- (p123) -- (p122);
		\filldraw[square, fill=\CAG] (p120) -- (p121) -- (p131) -- (p130);
		\filldraw[square, fill=\CAH] (p121) -- (p122) -- (p132) -- (p131);
		\filldraw[square, fill=\CAI] (p122) -- (p123) -- (p133) -- (p132);
		
		\filldraw[square, fill=\CBA] (p200) -- (p201) -- (p211) -- (p210);
		\filldraw[square, fill=\CBB] (p201) -- (p202) -- (p212) -- (p211);
		\filldraw[square, fill=\CBC] (p202) -- (p203) -- (p213) -- (p212);
		\filldraw[square, fill=\CBD] (p210) -- (p211) -- (p221) -- (p220);
		\filldraw[square, fill=\CBE] (p211) -- (p212) -- (p222) -- (p221);
		\filldraw[square, fill=\CBF] (p212) -- (p213) -- (p223) -- (p222);
		\filldraw[square, fill=\CBG] (p220) -- (p221) -- (p231) -- (p230);
		\filldraw[square, fill=\CBH] (p221) -- (p222) -- (p232) -- (p231);
		\filldraw[square, fill=\CBI] (p222) -- (p223) -- (p233) -- (p232);
		
		\filldraw[square, fill=\CCA] (p300) -- (p301) -- (p311) -- (p310);
		\filldraw[square, fill=\CCB] (p301) -- (p302) -- (p312) -- (p311);
		\filldraw[square, fill=\CCC] (p302) -- (p303) -- (p313) -- (p312);
		\filldraw[square, fill=\CCD] (p310) -- (p311) -- (p321) -- (p320);
		\filldraw[square, fill=\CCE] (p311) -- (p312) -- (p322) -- (p321);
		\filldraw[square, fill=\CCF] (p312) -- (p313) -- (p323) -- (p322);
		\filldraw[square, fill=\CCG] (p320) -- (p321) -- (p331) -- (p330);
		\filldraw[square, fill=\CCH] (p321) -- (p322) -- (p332) -- (p331);
		\filldraw[square, fill=\CCI] (p322) -- (p323) -- (p333) -- (p332);
	
    	\draw[side] (p100) -- (p200) -- (p203) -- (p233) -- (p330) -- (p130) -- (p100) -- (p200);
    	\draw[side] (p130) -- (p133) -- (p233);
    	\draw[side] (p103) -- (p133);
    	
    	\draw[square] (p101) -- (p131) -- (p331);
    	\draw[square] (p102) -- (p132) -- (p332);
    	\draw[square] (p201) -- (p231) -- (p310);
    	\draw[square] (p202) -- (p232) -- (p320);
    	\draw[square] (p110) -- (p113) -- (p213);
    	\draw[square] (p120) -- (p123) -- (p223);
	\draw[doubled]($(p322)+0.5*(vh3)$) -- ($(p132)+0.5*(vh1)$) -- ($(p112)+0.5*(vh1)$);
	\end{tikzpicture}
	}   
	\caption*{\label{fig:mouvR'} \LARGE{$R'$}} 	
\end{subfigure}
\hspace{\S}
\begin{subfigure}[b]{\W\columnwidth}
\centering
	\def\CAA{cyellow}
	\def\CAB{cred}
	\def\CAC{cred}
	\def\CAD{cyellow}
	\def\CAE{cred}
	\def\CAF{cred}
	\def\CAG{cyellow}
	\def\CAH{cred}
	\def\CAI{cred}
	
	\def\CBA{cblue}
	\def\CBB{cblue}
	\def\CBC{cblue}
	\def\CBD{cblue}
	\def\CBE{cblue}
	\def\CBF{cblue}
	\def\CBG{cblue}
	\def\CBH{cblue}
	\def\CBI{cblue}
	
	\def\CCA{cred}
	\def\CCB{cwhite}
	\def\CCC{cwhite}
	\def\CCD{cred}
	\def\CCE{cwhite}
	\def\CCF{cwhite}
	\def\CCG{cred}
	\def\CCH{cwhite}
	\def\CCI{cwhite}
	\resizebox{\columnwidth}{!}
	{
		\def\OFFSET{1.4cm}
	\def\OFF{2.5pt}
	\def\RATIO{0.2}
    \def\HXA{2.5cm}
    \def\HYA{-0.5cm}
    \def\VXA{0.0cm}
    \def\VYA{2.55cm}
    \def\HXB{1.5cm}
    \def\HYB{0.85cm}
    \def\VXB{\VXA}
    \def\VYB{\VYA}
    \def\HXC{\HXA}
    \def\HYC{\HYA}
    \def\VXC{\HXB}
    \def\VYC{\HYB}
    \tikzstyle{square}=[line width=3pt, join=round, cap=round]
    \tikzstyle{side}=[line width=5pt, join=round, cap=round]
    \tikzstyle{arrow}=[line width=9pt, join=round, cap=round,->,rounded corners=3cm, cpurple]
    %\tikzstyle{double arrow}=[9pt colored by black and white]
    \tikzstyle{border}=[line width=2pt, join=round, cap=round]
    \begin{tikzpicture}[>=triangle 45]
		\coordinate (vh1) at (\HXA ,\HYA);
		\coordinate (vv1) at (\VXA, \VYA);
		\coordinate (vh2) at (\HXB, \HYB);
		\coordinate (vv2) at (\VXB, \VYB);
		\coordinate (vh3) at (\HXC, \HYC);
		\coordinate (vv3) at (\VXC, \VYC);
		    
    	\coordinate (p100) at (0,0);
    	\coordinate (p101) at ($(p100)+1*(vh1)$);
    	\coordinate (p102) at ($(p100)+2*(vh1)$);
    	\coordinate (p103) at ($(p100)+3*(vh1)$);
    	\coordinate (p110) at ($(p100)+1*(vv1)$);
    	\coordinate (p111) at ($(p110)+1*(vh1)$);
		\coordinate (p112) at ($(p110)+2*(vh1)$);
    	\coordinate (p113) at ($(p110)+3*(vh1)$);
    	\coordinate (p120) at ($(p100)+2*(vv1)$);
    	\coordinate (p121) at ($(p120)+1*(vh1)$);
    	\coordinate (p122) at ($(p120)+2*(vh1)$);
    	\coordinate (p123) at ($(p120)+3*(vh1)$);
    	\coordinate (p130) at ($(p100)+3*(vv1)$);
    	\coordinate (p131) at ($(p130)+1*(vh1)$);
    	\coordinate (p132) at ($(p130)+2*(vh1)$);
    	\coordinate (p133) at ($(p130)+3*(vh1)$);
    	
    	\coordinate (p200) at (p103);
    	\coordinate (p201) at ($(p200)+1*(vh2)$);
    	\coordinate (p202) at ($(p200)+2*(vh2)$);
    	\coordinate (p203) at ($(p200)+3*(vh2)$);
    	\coordinate (p210) at ($(p200)+1*(vv2)$);
    	\coordinate (p211) at ($(p210)+1*(vh2)$);
		\coordinate (p212) at ($(p210)+2*(vh2)$);
    	\coordinate (p213) at ($(p210)+3*(vh2)$);
    	\coordinate (p220) at ($(p200)+2*(vv2)$);
    	\coordinate (p221) at ($(p220)+1*(vh2)$);
    	\coordinate (p222) at ($(p220)+2*(vh2)$);
    	\coordinate (p223) at ($(p220)+3*(vh2)$);
    	\coordinate (p230) at ($(p200)+3*(vv2)$);
    	\coordinate (p231) at ($(p230)+1*(vh2)$);
    	\coordinate (p232) at ($(p230)+2*(vh2)$);
    	\coordinate (p233) at ($(p230)+3*(vh2)$);
    	
		\coordinate (p300) at (p130);
    	\coordinate (p301) at ($(p300)+1*(vh3)$);
    	\coordinate (p302) at ($(p300)+2*(vh3)$);
    	\coordinate (p303) at ($(p300)+3*(vh3)$);
    	\coordinate (p310) at ($(p300)+1*(vv3)$);
    	\coordinate (p311) at ($(p310)+1*(vh3)$);
		\coordinate (p312) at ($(p310)+2*(vh3)$);
    	\coordinate (p313) at ($(p310)+3*(vh3)$);
    	\coordinate (p320) at ($(p300)+2*(vv3)$);
    	\coordinate (p321) at ($(p320)+1*(vh3)$);
    	\coordinate (p322) at ($(p320)+2*(vh3)$);
    	\coordinate (p323) at ($(p320)+3*(vh3)$);
    	\coordinate (p330) at ($(p300)+3*(vv3)$);
    	\coordinate (p331) at ($(p330)+1*(vh3)$);
    	\coordinate (p332) at ($(p330)+2*(vh3)$);
    	\coordinate (p333) at ($(p330)+3*(vh3)$);  	  	
    	
		\filldraw[square, fill=\CAA] (p100) -- (p101) -- (p111) -- (p110);
		\filldraw[square, fill=\CAB] (p101) -- (p102) -- (p112) -- (p111);
		\filldraw[square, fill=\CAC] (p102) -- (p103) -- (p113) -- (p112);
		\filldraw[square, fill=\CAD] (p110) -- (p111) -- (p121) -- (p120);
		\filldraw[square, fill=\CAE] (p111) -- (p112) -- (p122) -- (p121);
		\filldraw[square, fill=\CAF] (p112) -- (p113) -- (p123) -- (p122);
		\filldraw[square, fill=\CAG] (p120) -- (p121) -- (p131) -- (p130);
		\filldraw[square, fill=\CAH] (p121) -- (p122) -- (p132) -- (p131);
		\filldraw[square, fill=\CAI] (p122) -- (p123) -- (p133) -- (p132);
		
		\filldraw[square, fill=\CBA] (p200) -- (p201) -- (p211) -- (p210);
		\filldraw[square, fill=\CBB] (p201) -- (p202) -- (p212) -- (p211);
		\filldraw[square, fill=\CBC] (p202) -- (p203) -- (p213) -- (p212);
		\filldraw[square, fill=\CBD] (p210) -- (p211) -- (p221) -- (p220);
		\filldraw[square, fill=\CBE] (p211) -- (p212) -- (p222) -- (p221);
		\filldraw[square, fill=\CBF] (p212) -- (p213) -- (p223) -- (p222);
		\filldraw[square, fill=\CBG] (p220) -- (p221) -- (p231) -- (p230);
		\filldraw[square, fill=\CBH] (p221) -- (p222) -- (p232) -- (p231);
		\filldraw[square, fill=\CBI] (p222) -- (p223) -- (p233) -- (p232);
		
		\filldraw[square, fill=\CCA] (p300) -- (p301) -- (p311) -- (p310);
		\filldraw[square, fill=\CCB] (p301) -- (p302) -- (p312) -- (p311);
		\filldraw[square, fill=\CCC] (p302) -- (p303) -- (p313) -- (p312);
		\filldraw[square, fill=\CCD] (p310) -- (p311) -- (p321) -- (p320);
		\filldraw[square, fill=\CCE] (p311) -- (p312) -- (p322) -- (p321);
		\filldraw[square, fill=\CCF] (p312) -- (p313) -- (p323) -- (p322);
		\filldraw[square, fill=\CCG] (p320) -- (p321) -- (p331) -- (p330);
		\filldraw[square, fill=\CCH] (p321) -- (p322) -- (p332) -- (p331);
		\filldraw[square, fill=\CCI] (p322) -- (p323) -- (p333) -- (p332);
	
    	\draw[side] (p100) -- (p200) -- (p203) -- (p233) -- (p330) -- (p130) -- (p100) -- (p200);
    	\draw[side] (p130) -- (p133) -- (p233);
    	\draw[side] (p103) -- (p133);
    	
    	\draw[square] (p101) -- (p131) -- (p331);
    	\draw[square] (p102) -- (p132) -- (p332);
    	\draw[square] (p201) -- (p231) -- (p310);
    	\draw[square] (p202) -- (p232) -- (p320);
    	\draw[square] (p110) -- (p113) -- (p213);
    	\draw[square] (p120) -- (p123) -- (p223);
	\draw[doubled]($(p110)+0.5*(vh1)$) -- ($(p130)+0.5*(vh1)$) -- ($(p320)+0.5*(vh3)$);
	\end{tikzpicture}
	}    
	\caption*{\label{fig:mouvL'} \LARGE{$L'$}}	
\end{subfigure}
\hspace{\S}
\begin{subfigure}[b]{\W\columnwidth}
\centering
	\def\CAA{cred}
	\def\CAB{cred}
	\def\CAC{cred}
	\def\CAD{cred}
	\def\CAE{cred}
	\def\CAF{cred}
	\def\CAG{cred}
	\def\CAH{cred}
	\def\CAI{cred}
	
	\def\CBA{cyellow}
	\def\CBB{cblue}
	\def\CBC{cblue}
	\def\CBD{cyellow}
	\def\CBE{cblue}
	\def\CBF{cblue}
	\def\CBG{cyellow}
	\def\CBH{cblue}
	\def\CBI{cblue}
	
	\def\CCA{cblue}
	\def\CCB{cblue}
	\def\CCC{cblue}
	\def\CCD{cwhite}
	\def\CCE{cwhite}
	\def\CCF{cwhite}
	\def\CCG{cwhite}
	\def\CCH{cwhite}
	\def\CCI{cwhite}
	\resizebox{\columnwidth}{!}
	{
		\def\OFFSET{1.4cm}
	\def\OFF{2.5pt}
	\def\RATIO{0.2}
    \def\HXA{2.5cm}
    \def\HYA{-0.5cm}
    \def\VXA{0.0cm}
    \def\VYA{2.55cm}
    \def\HXB{1.5cm}
    \def\HYB{0.85cm}
    \def\VXB{\VXA}
    \def\VYB{\VYA}
    \def\HXC{\HXA}
    \def\HYC{\HYA}
    \def\VXC{\HXB}
    \def\VYC{\HYB}
    \tikzstyle{square}=[line width=3pt, join=round, cap=round]
    \tikzstyle{side}=[line width=5pt, join=round, cap=round]
    \tikzstyle{arrow}=[line width=9pt, join=round, cap=round,->,rounded corners=3cm, cpurple]
    %\tikzstyle{double arrow}=[9pt colored by black and white]
    \tikzstyle{border}=[line width=2pt, join=round, cap=round]
    \begin{tikzpicture}[>=triangle 45]
		\coordinate (vh1) at (\HXA ,\HYA);
		\coordinate (vv1) at (\VXA, \VYA);
		\coordinate (vh2) at (\HXB, \HYB);
		\coordinate (vv2) at (\VXB, \VYB);
		\coordinate (vh3) at (\HXC, \HYC);
		\coordinate (vv3) at (\VXC, \VYC);
		    
    	\coordinate (p100) at (0,0);
    	\coordinate (p101) at ($(p100)+1*(vh1)$);
    	\coordinate (p102) at ($(p100)+2*(vh1)$);
    	\coordinate (p103) at ($(p100)+3*(vh1)$);
    	\coordinate (p110) at ($(p100)+1*(vv1)$);
    	\coordinate (p111) at ($(p110)+1*(vh1)$);
		\coordinate (p112) at ($(p110)+2*(vh1)$);
    	\coordinate (p113) at ($(p110)+3*(vh1)$);
    	\coordinate (p120) at ($(p100)+2*(vv1)$);
    	\coordinate (p121) at ($(p120)+1*(vh1)$);
    	\coordinate (p122) at ($(p120)+2*(vh1)$);
    	\coordinate (p123) at ($(p120)+3*(vh1)$);
    	\coordinate (p130) at ($(p100)+3*(vv1)$);
    	\coordinate (p131) at ($(p130)+1*(vh1)$);
    	\coordinate (p132) at ($(p130)+2*(vh1)$);
    	\coordinate (p133) at ($(p130)+3*(vh1)$);
    	
    	\coordinate (p200) at (p103);
    	\coordinate (p201) at ($(p200)+1*(vh2)$);
    	\coordinate (p202) at ($(p200)+2*(vh2)$);
    	\coordinate (p203) at ($(p200)+3*(vh2)$);
    	\coordinate (p210) at ($(p200)+1*(vv2)$);
    	\coordinate (p211) at ($(p210)+1*(vh2)$);
		\coordinate (p212) at ($(p210)+2*(vh2)$);
    	\coordinate (p213) at ($(p210)+3*(vh2)$);
    	\coordinate (p220) at ($(p200)+2*(vv2)$);
    	\coordinate (p221) at ($(p220)+1*(vh2)$);
    	\coordinate (p222) at ($(p220)+2*(vh2)$);
    	\coordinate (p223) at ($(p220)+3*(vh2)$);
    	\coordinate (p230) at ($(p200)+3*(vv2)$);
    	\coordinate (p231) at ($(p230)+1*(vh2)$);
    	\coordinate (p232) at ($(p230)+2*(vh2)$);
    	\coordinate (p233) at ($(p230)+3*(vh2)$);
    	
		\coordinate (p300) at (p130);
    	\coordinate (p301) at ($(p300)+1*(vh3)$);
    	\coordinate (p302) at ($(p300)+2*(vh3)$);
    	\coordinate (p303) at ($(p300)+3*(vh3)$);
    	\coordinate (p310) at ($(p300)+1*(vv3)$);
    	\coordinate (p311) at ($(p310)+1*(vh3)$);
		\coordinate (p312) at ($(p310)+2*(vh3)$);
    	\coordinate (p313) at ($(p310)+3*(vh3)$);
    	\coordinate (p320) at ($(p300)+2*(vv3)$);
    	\coordinate (p321) at ($(p320)+1*(vh3)$);
    	\coordinate (p322) at ($(p320)+2*(vh3)$);
    	\coordinate (p323) at ($(p320)+3*(vh3)$);
    	\coordinate (p330) at ($(p300)+3*(vv3)$);
    	\coordinate (p331) at ($(p330)+1*(vh3)$);
    	\coordinate (p332) at ($(p330)+2*(vh3)$);
    	\coordinate (p333) at ($(p330)+3*(vh3)$);  	  	
    	
		\filldraw[square, fill=\CAA] (p100) -- (p101) -- (p111) -- (p110);
		\filldraw[square, fill=\CAB] (p101) -- (p102) -- (p112) -- (p111);
		\filldraw[square, fill=\CAC] (p102) -- (p103) -- (p113) -- (p112);
		\filldraw[square, fill=\CAD] (p110) -- (p111) -- (p121) -- (p120);
		\filldraw[square, fill=\CAE] (p111) -- (p112) -- (p122) -- (p121);
		\filldraw[square, fill=\CAF] (p112) -- (p113) -- (p123) -- (p122);
		\filldraw[square, fill=\CAG] (p120) -- (p121) -- (p131) -- (p130);
		\filldraw[square, fill=\CAH] (p121) -- (p122) -- (p132) -- (p131);
		\filldraw[square, fill=\CAI] (p122) -- (p123) -- (p133) -- (p132);
		
		\filldraw[square, fill=\CBA] (p200) -- (p201) -- (p211) -- (p210);
		\filldraw[square, fill=\CBB] (p201) -- (p202) -- (p212) -- (p211);
		\filldraw[square, fill=\CBC] (p202) -- (p203) -- (p213) -- (p212);
		\filldraw[square, fill=\CBD] (p210) -- (p211) -- (p221) -- (p220);
		\filldraw[square, fill=\CBE] (p211) -- (p212) -- (p222) -- (p221);
		\filldraw[square, fill=\CBF] (p212) -- (p213) -- (p223) -- (p222);
		\filldraw[square, fill=\CBG] (p220) -- (p221) -- (p231) -- (p230);
		\filldraw[square, fill=\CBH] (p221) -- (p222) -- (p232) -- (p231);
		\filldraw[square, fill=\CBI] (p222) -- (p223) -- (p233) -- (p232);
		
		\filldraw[square, fill=\CCA] (p300) -- (p301) -- (p311) -- (p310);
		\filldraw[square, fill=\CCB] (p301) -- (p302) -- (p312) -- (p311);
		\filldraw[square, fill=\CCC] (p302) -- (p303) -- (p313) -- (p312);
		\filldraw[square, fill=\CCD] (p310) -- (p311) -- (p321) -- (p320);
		\filldraw[square, fill=\CCE] (p311) -- (p312) -- (p322) -- (p321);
		\filldraw[square, fill=\CCF] (p312) -- (p313) -- (p323) -- (p322);
		\filldraw[square, fill=\CCG] (p320) -- (p321) -- (p331) -- (p330);
		\filldraw[square, fill=\CCH] (p321) -- (p322) -- (p332) -- (p331);
		\filldraw[square, fill=\CCI] (p322) -- (p323) -- (p333) -- (p332);
	
    	\draw[side] (p100) -- (p200) -- (p203) -- (p233) -- (p330) -- (p130) -- (p100) -- (p200);
    	\draw[side] (p130) -- (p133) -- (p233);
    	\draw[side] (p103) -- (p133);
    	
    	\draw[square] (p101) -- (p131) -- (p331);
    	\draw[square] (p102) -- (p132) -- (p332);
    	\draw[square] (p201) -- (p231) -- (p310);
    	\draw[square] (p202) -- (p232) -- (p320);
    	\draw[square] (p110) -- (p113) -- (p213);
    	\draw[square] (p120) -- (p123) -- (p223);
	\draw[doubled]($(p210)+0.5*(vh2)$) -- ($(p303)+0.5*(vv3)$) -- ($(p301)+0.5*(vv3)$);
	\end{tikzpicture}
	}    
	\caption*{\label{fig:mouvF'} \LARGE{$F'$}}
\end{subfigure}
\hspace{\S}
\begin{subfigure}[b]{\W\columnwidth}
\centering
	\def\CAA{cblue}
	\def\CAB{cblue}
	\def\CAC{cblue}
	\def\CAD{cred}
	\def\CAE{cred}
	\def\CAF{cred}
	\def\CAG{cred}
	\def\CAH{cred}
	\def\CAI{cred}
	
	\def\CBA{corange}
	\def\CBB{corange}
	\def\CBC{corange}
	\def\CBD{cblue}
	\def\CBE{cblue}
	\def\CBF{cblue}
	\def\CBG{cblue}
	\def\CBH{cblue}
	\def\CBI{cblue}
	
	\def\CCA{cwhite}
	\def\CCB{cwhite}
	\def\CCC{cwhite}
	\def\CCD{cwhite}
	\def\CCE{cwhite}
	\def\CCF{cwhite}
	\def\CCG{cwhite}
	\def\CCH{cwhite}
	\def\CCI{cwhite}
	\resizebox{\columnwidth}{!}
	{
		\def\OFFSET{1.4cm}
	\def\OFF{2.5pt}
	\def\RATIO{0.2}
    \def\HXA{2.5cm}
    \def\HYA{-0.5cm}
    \def\VXA{0.0cm}
    \def\VYA{2.55cm}
    \def\HXB{1.5cm}
    \def\HYB{0.85cm}
    \def\VXB{\VXA}
    \def\VYB{\VYA}
    \def\HXC{\HXA}
    \def\HYC{\HYA}
    \def\VXC{\HXB}
    \def\VYC{\HYB}
    \tikzstyle{square}=[line width=3pt, join=round, cap=round]
    \tikzstyle{side}=[line width=5pt, join=round, cap=round]
    \tikzstyle{arrow}=[line width=9pt, join=round, cap=round,->,rounded corners=3cm, cpurple]
    %\tikzstyle{double arrow}=[9pt colored by black and white]
    \tikzstyle{border}=[line width=2pt, join=round, cap=round]
    \begin{tikzpicture}[>=triangle 45]
		\coordinate (vh1) at (\HXA ,\HYA);
		\coordinate (vv1) at (\VXA, \VYA);
		\coordinate (vh2) at (\HXB, \HYB);
		\coordinate (vv2) at (\VXB, \VYB);
		\coordinate (vh3) at (\HXC, \HYC);
		\coordinate (vv3) at (\VXC, \VYC);
		    
    	\coordinate (p100) at (0,0);
    	\coordinate (p101) at ($(p100)+1*(vh1)$);
    	\coordinate (p102) at ($(p100)+2*(vh1)$);
    	\coordinate (p103) at ($(p100)+3*(vh1)$);
    	\coordinate (p110) at ($(p100)+1*(vv1)$);
    	\coordinate (p111) at ($(p110)+1*(vh1)$);
		\coordinate (p112) at ($(p110)+2*(vh1)$);
    	\coordinate (p113) at ($(p110)+3*(vh1)$);
    	\coordinate (p120) at ($(p100)+2*(vv1)$);
    	\coordinate (p121) at ($(p120)+1*(vh1)$);
    	\coordinate (p122) at ($(p120)+2*(vh1)$);
    	\coordinate (p123) at ($(p120)+3*(vh1)$);
    	\coordinate (p130) at ($(p100)+3*(vv1)$);
    	\coordinate (p131) at ($(p130)+1*(vh1)$);
    	\coordinate (p132) at ($(p130)+2*(vh1)$);
    	\coordinate (p133) at ($(p130)+3*(vh1)$);
    	
    	\coordinate (p200) at (p103);
    	\coordinate (p201) at ($(p200)+1*(vh2)$);
    	\coordinate (p202) at ($(p200)+2*(vh2)$);
    	\coordinate (p203) at ($(p200)+3*(vh2)$);
    	\coordinate (p210) at ($(p200)+1*(vv2)$);
    	\coordinate (p211) at ($(p210)+1*(vh2)$);
		\coordinate (p212) at ($(p210)+2*(vh2)$);
    	\coordinate (p213) at ($(p210)+3*(vh2)$);
    	\coordinate (p220) at ($(p200)+2*(vv2)$);
    	\coordinate (p221) at ($(p220)+1*(vh2)$);
    	\coordinate (p222) at ($(p220)+2*(vh2)$);
    	\coordinate (p223) at ($(p220)+3*(vh2)$);
    	\coordinate (p230) at ($(p200)+3*(vv2)$);
    	\coordinate (p231) at ($(p230)+1*(vh2)$);
    	\coordinate (p232) at ($(p230)+2*(vh2)$);
    	\coordinate (p233) at ($(p230)+3*(vh2)$);
    	
		\coordinate (p300) at (p130);
    	\coordinate (p301) at ($(p300)+1*(vh3)$);
    	\coordinate (p302) at ($(p300)+2*(vh3)$);
    	\coordinate (p303) at ($(p300)+3*(vh3)$);
    	\coordinate (p310) at ($(p300)+1*(vv3)$);
    	\coordinate (p311) at ($(p310)+1*(vh3)$);
		\coordinate (p312) at ($(p310)+2*(vh3)$);
    	\coordinate (p313) at ($(p310)+3*(vh3)$);
    	\coordinate (p320) at ($(p300)+2*(vv3)$);
    	\coordinate (p321) at ($(p320)+1*(vh3)$);
    	\coordinate (p322) at ($(p320)+2*(vh3)$);
    	\coordinate (p323) at ($(p320)+3*(vh3)$);
    	\coordinate (p330) at ($(p300)+3*(vv3)$);
    	\coordinate (p331) at ($(p330)+1*(vh3)$);
    	\coordinate (p332) at ($(p330)+2*(vh3)$);
    	\coordinate (p333) at ($(p330)+3*(vh3)$);  	  	
    	
		\filldraw[square, fill=\CAA] (p100) -- (p101) -- (p111) -- (p110);
		\filldraw[square, fill=\CAB] (p101) -- (p102) -- (p112) -- (p111);
		\filldraw[square, fill=\CAC] (p102) -- (p103) -- (p113) -- (p112);
		\filldraw[square, fill=\CAD] (p110) -- (p111) -- (p121) -- (p120);
		\filldraw[square, fill=\CAE] (p111) -- (p112) -- (p122) -- (p121);
		\filldraw[square, fill=\CAF] (p112) -- (p113) -- (p123) -- (p122);
		\filldraw[square, fill=\CAG] (p120) -- (p121) -- (p131) -- (p130);
		\filldraw[square, fill=\CAH] (p121) -- (p122) -- (p132) -- (p131);
		\filldraw[square, fill=\CAI] (p122) -- (p123) -- (p133) -- (p132);
		
		\filldraw[square, fill=\CBA] (p200) -- (p201) -- (p211) -- (p210);
		\filldraw[square, fill=\CBB] (p201) -- (p202) -- (p212) -- (p211);
		\filldraw[square, fill=\CBC] (p202) -- (p203) -- (p213) -- (p212);
		\filldraw[square, fill=\CBD] (p210) -- (p211) -- (p221) -- (p220);
		\filldraw[square, fill=\CBE] (p211) -- (p212) -- (p222) -- (p221);
		\filldraw[square, fill=\CBF] (p212) -- (p213) -- (p223) -- (p222);
		\filldraw[square, fill=\CBG] (p220) -- (p221) -- (p231) -- (p230);
		\filldraw[square, fill=\CBH] (p221) -- (p222) -- (p232) -- (p231);
		\filldraw[square, fill=\CBI] (p222) -- (p223) -- (p233) -- (p232);
		
		\filldraw[square, fill=\CCA] (p300) -- (p301) -- (p311) -- (p310);
		\filldraw[square, fill=\CCB] (p301) -- (p302) -- (p312) -- (p311);
		\filldraw[square, fill=\CCC] (p302) -- (p303) -- (p313) -- (p312);
		\filldraw[square, fill=\CCD] (p310) -- (p311) -- (p321) -- (p320);
		\filldraw[square, fill=\CCE] (p311) -- (p312) -- (p322) -- (p321);
		\filldraw[square, fill=\CCF] (p312) -- (p313) -- (p323) -- (p322);
		\filldraw[square, fill=\CCG] (p320) -- (p321) -- (p331) -- (p330);
		\filldraw[square, fill=\CCH] (p321) -- (p322) -- (p332) -- (p331);
		\filldraw[square, fill=\CCI] (p322) -- (p323) -- (p333) -- (p332);
	
    	\draw[side] (p100) -- (p200) -- (p203) -- (p233) -- (p330) -- (p130) -- (p100) -- (p200);
    	\draw[side] (p130) -- (p133) -- (p233);
    	\draw[side] (p103) -- (p133);
    	
    	\draw[square] (p101) -- (p131) -- (p331);
    	\draw[square] (p102) -- (p132) -- (p332);
    	\draw[square] (p201) -- (p231) -- (p310);
    	\draw[square] (p202) -- (p232) -- (p320);
    	\draw[square] (p110) -- (p113) -- (p213);
    	\draw[square] (p120) -- (p123) -- (p223);
	\draw[doubled]($(p202)+0.5*(vv2)$) -- ($(p103)+0.5*(vv1)$) -- ($(p101)+0.5*(vv1)$);
	\end{tikzpicture}
	}    
	\caption*{\label{fig:mouvD'} \LARGE{$D'$}}
\end{subfigure}
\hspace{\S}
\begin{subfigure}[b]{\W\columnwidth}
\centering
	\def\CAA{cred}
	\def\CAB{cred}
	\def\CAC{cred}
	\def\CAD{cred}
	\def\CAE{cred}
	\def\CAF{cred}
	\def\CAG{cgreen}
	\def\CAH{cgreen}
	\def\CAI{cgreen}
	
	\def\CBA{cblue}
	\def\CBB{cblue}
	\def\CBC{cblue}
	\def\CBD{cblue}
	\def\CBE{cblue}
	\def\CBF{cblue}
	\def\CBG{cred}
	\def\CBH{cred}
	\def\CBI{cred}
	
	\def\CCA{cwhite}
	\def\CCB{cwhite}
	\def\CCC{cwhite}
	\def\CCD{cwhite}
	\def\CCE{cwhite}
	\def\CCF{cwhite}
	\def\CCG{cwhite}
	\def\CCH{cwhite}
	\def\CCI{cwhite}
	\resizebox{\columnwidth}{!}
	{
		\def\OFFSET{1.4cm}
	\def\OFF{2.5pt}
	\def\RATIO{0.2}
    \def\HXA{2.5cm}
    \def\HYA{-0.5cm}
    \def\VXA{0.0cm}
    \def\VYA{2.55cm}
    \def\HXB{1.5cm}
    \def\HYB{0.85cm}
    \def\VXB{\VXA}
    \def\VYB{\VYA}
    \def\HXC{\HXA}
    \def\HYC{\HYA}
    \def\VXC{\HXB}
    \def\VYC{\HYB}
    \tikzstyle{square}=[line width=3pt, join=round, cap=round]
    \tikzstyle{side}=[line width=5pt, join=round, cap=round]
    \tikzstyle{arrow}=[line width=9pt, join=round, cap=round,->,rounded corners=3cm, cpurple]
    %\tikzstyle{double arrow}=[9pt colored by black and white]
    \tikzstyle{border}=[line width=2pt, join=round, cap=round]
    \begin{tikzpicture}[>=triangle 45]
		\coordinate (vh1) at (\HXA ,\HYA);
		\coordinate (vv1) at (\VXA, \VYA);
		\coordinate (vh2) at (\HXB, \HYB);
		\coordinate (vv2) at (\VXB, \VYB);
		\coordinate (vh3) at (\HXC, \HYC);
		\coordinate (vv3) at (\VXC, \VYC);
		    
    	\coordinate (p100) at (0,0);
    	\coordinate (p101) at ($(p100)+1*(vh1)$);
    	\coordinate (p102) at ($(p100)+2*(vh1)$);
    	\coordinate (p103) at ($(p100)+3*(vh1)$);
    	\coordinate (p110) at ($(p100)+1*(vv1)$);
    	\coordinate (p111) at ($(p110)+1*(vh1)$);
		\coordinate (p112) at ($(p110)+2*(vh1)$);
    	\coordinate (p113) at ($(p110)+3*(vh1)$);
    	\coordinate (p120) at ($(p100)+2*(vv1)$);
    	\coordinate (p121) at ($(p120)+1*(vh1)$);
    	\coordinate (p122) at ($(p120)+2*(vh1)$);
    	\coordinate (p123) at ($(p120)+3*(vh1)$);
    	\coordinate (p130) at ($(p100)+3*(vv1)$);
    	\coordinate (p131) at ($(p130)+1*(vh1)$);
    	\coordinate (p132) at ($(p130)+2*(vh1)$);
    	\coordinate (p133) at ($(p130)+3*(vh1)$);
    	
    	\coordinate (p200) at (p103);
    	\coordinate (p201) at ($(p200)+1*(vh2)$);
    	\coordinate (p202) at ($(p200)+2*(vh2)$);
    	\coordinate (p203) at ($(p200)+3*(vh2)$);
    	\coordinate (p210) at ($(p200)+1*(vv2)$);
    	\coordinate (p211) at ($(p210)+1*(vh2)$);
		\coordinate (p212) at ($(p210)+2*(vh2)$);
    	\coordinate (p213) at ($(p210)+3*(vh2)$);
    	\coordinate (p220) at ($(p200)+2*(vv2)$);
    	\coordinate (p221) at ($(p220)+1*(vh2)$);
    	\coordinate (p222) at ($(p220)+2*(vh2)$);
    	\coordinate (p223) at ($(p220)+3*(vh2)$);
    	\coordinate (p230) at ($(p200)+3*(vv2)$);
    	\coordinate (p231) at ($(p230)+1*(vh2)$);
    	\coordinate (p232) at ($(p230)+2*(vh2)$);
    	\coordinate (p233) at ($(p230)+3*(vh2)$);
    	
		\coordinate (p300) at (p130);
    	\coordinate (p301) at ($(p300)+1*(vh3)$);
    	\coordinate (p302) at ($(p300)+2*(vh3)$);
    	\coordinate (p303) at ($(p300)+3*(vh3)$);
    	\coordinate (p310) at ($(p300)+1*(vv3)$);
    	\coordinate (p311) at ($(p310)+1*(vh3)$);
		\coordinate (p312) at ($(p310)+2*(vh3)$);
    	\coordinate (p313) at ($(p310)+3*(vh3)$);
    	\coordinate (p320) at ($(p300)+2*(vv3)$);
    	\coordinate (p321) at ($(p320)+1*(vh3)$);
    	\coordinate (p322) at ($(p320)+2*(vh3)$);
    	\coordinate (p323) at ($(p320)+3*(vh3)$);
    	\coordinate (p330) at ($(p300)+3*(vv3)$);
    	\coordinate (p331) at ($(p330)+1*(vh3)$);
    	\coordinate (p332) at ($(p330)+2*(vh3)$);
    	\coordinate (p333) at ($(p330)+3*(vh3)$);  	  	
    	
		\filldraw[square, fill=\CAA] (p100) -- (p101) -- (p111) -- (p110);
		\filldraw[square, fill=\CAB] (p101) -- (p102) -- (p112) -- (p111);
		\filldraw[square, fill=\CAC] (p102) -- (p103) -- (p113) -- (p112);
		\filldraw[square, fill=\CAD] (p110) -- (p111) -- (p121) -- (p120);
		\filldraw[square, fill=\CAE] (p111) -- (p112) -- (p122) -- (p121);
		\filldraw[square, fill=\CAF] (p112) -- (p113) -- (p123) -- (p122);
		\filldraw[square, fill=\CAG] (p120) -- (p121) -- (p131) -- (p130);
		\filldraw[square, fill=\CAH] (p121) -- (p122) -- (p132) -- (p131);
		\filldraw[square, fill=\CAI] (p122) -- (p123) -- (p133) -- (p132);
		
		\filldraw[square, fill=\CBA] (p200) -- (p201) -- (p211) -- (p210);
		\filldraw[square, fill=\CBB] (p201) -- (p202) -- (p212) -- (p211);
		\filldraw[square, fill=\CBC] (p202) -- (p203) -- (p213) -- (p212);
		\filldraw[square, fill=\CBD] (p210) -- (p211) -- (p221) -- (p220);
		\filldraw[square, fill=\CBE] (p211) -- (p212) -- (p222) -- (p221);
		\filldraw[square, fill=\CBF] (p212) -- (p213) -- (p223) -- (p222);
		\filldraw[square, fill=\CBG] (p220) -- (p221) -- (p231) -- (p230);
		\filldraw[square, fill=\CBH] (p221) -- (p222) -- (p232) -- (p231);
		\filldraw[square, fill=\CBI] (p222) -- (p223) -- (p233) -- (p232);
		
		\filldraw[square, fill=\CCA] (p300) -- (p301) -- (p311) -- (p310);
		\filldraw[square, fill=\CCB] (p301) -- (p302) -- (p312) -- (p311);
		\filldraw[square, fill=\CCC] (p302) -- (p303) -- (p313) -- (p312);
		\filldraw[square, fill=\CCD] (p310) -- (p311) -- (p321) -- (p320);
		\filldraw[square, fill=\CCE] (p311) -- (p312) -- (p322) -- (p321);
		\filldraw[square, fill=\CCF] (p312) -- (p313) -- (p323) -- (p322);
		\filldraw[square, fill=\CCG] (p320) -- (p321) -- (p331) -- (p330);
		\filldraw[square, fill=\CCH] (p321) -- (p322) -- (p332) -- (p331);
		\filldraw[square, fill=\CCI] (p322) -- (p323) -- (p333) -- (p332);
	
    	\draw[side] (p100) -- (p200) -- (p203) -- (p233) -- (p330) -- (p130) -- (p100) -- (p200);
    	\draw[side] (p130) -- (p133) -- (p233);
    	\draw[side] (p103) -- (p133);
    	
    	\draw[square] (p101) -- (p131) -- (p331);
    	\draw[square] (p102) -- (p132) -- (p332);
    	\draw[square] (p201) -- (p231) -- (p310);
    	\draw[square] (p202) -- (p232) -- (p320);
    	\draw[square] (p110) -- (p113) -- (p213);
    	\draw[square] (p120) -- (p123) -- (p223);
	\draw[doubled]($(p121)+0.5*(vv1)$) -- ($(p123)+0.5*(vv1)$) -- ($(p222)+0.5*(vv2)$);
	\end{tikzpicture}
	}    
	\caption*{\label{fig:mouvU'} \LARGE{$U'$}}
\end{subfigure}
\hspace{\S}
\begin{subfigure}[b]{\W\columnwidth}
\centering
	\def\CAA{cred}
	\def\CAB{cred}
	\def\CAC{cred}
	\def\CAD{cred}
	\def\CAE{cred}
	\def\CAF{cred}
	\def\CAG{cred}
	\def\CAH{cred}
	\def\CAI{cred}
	
	\def\CBA{cblue}
	\def\CBB{cblue}
	\def\CBC{cwhite}
	\def\CBD{cblue}
	\def\CBE{cblue}
	\def\CBF{cwhite}
	\def\CBG{cblue}
	\def\CBH{cblue}
	\def\CBI{cwhite}
	
	\def\CCA{cwhite}
	\def\CCB{cwhite}
	\def\CCC{cwhite}
	\def\CCD{cwhite}
	\def\CCE{cwhite}
	\def\CCF{cwhite}
	\def\CCG{cgreen}
	\def\CCH{cgreen}
	\def\CCI{cgreen} 
	\resizebox{\columnwidth}{!}
	{
		\def\OFFSET{1.4cm}
	\def\OFF{2.5pt}
	\def\RATIO{0.2}
    \def\HXA{2.5cm}
    \def\HYA{-0.5cm}
    \def\VXA{0.0cm}
    \def\VYA{2.55cm}
    \def\HXB{1.5cm}
    \def\HYB{0.85cm}
    \def\VXB{\VXA}
    \def\VYB{\VYA}
    \def\HXC{\HXA}
    \def\HYC{\HYA}
    \def\VXC{\HXB}
    \def\VYC{\HYB}
    \tikzstyle{square}=[line width=3pt, join=round, cap=round]
    \tikzstyle{side}=[line width=5pt, join=round, cap=round]
    \tikzstyle{arrow}=[line width=9pt, join=round, cap=round,->,rounded corners=3cm, cpurple]
    %\tikzstyle{double arrow}=[9pt colored by black and white]
    \tikzstyle{border}=[line width=2pt, join=round, cap=round]
    \begin{tikzpicture}[>=triangle 45]
		\coordinate (vh1) at (\HXA ,\HYA);
		\coordinate (vv1) at (\VXA, \VYA);
		\coordinate (vh2) at (\HXB, \HYB);
		\coordinate (vv2) at (\VXB, \VYB);
		\coordinate (vh3) at (\HXC, \HYC);
		\coordinate (vv3) at (\VXC, \VYC);
		    
    	\coordinate (p100) at (0,0);
    	\coordinate (p101) at ($(p100)+1*(vh1)$);
    	\coordinate (p102) at ($(p100)+2*(vh1)$);
    	\coordinate (p103) at ($(p100)+3*(vh1)$);
    	\coordinate (p110) at ($(p100)+1*(vv1)$);
    	\coordinate (p111) at ($(p110)+1*(vh1)$);
		\coordinate (p112) at ($(p110)+2*(vh1)$);
    	\coordinate (p113) at ($(p110)+3*(vh1)$);
    	\coordinate (p120) at ($(p100)+2*(vv1)$);
    	\coordinate (p121) at ($(p120)+1*(vh1)$);
    	\coordinate (p122) at ($(p120)+2*(vh1)$);
    	\coordinate (p123) at ($(p120)+3*(vh1)$);
    	\coordinate (p130) at ($(p100)+3*(vv1)$);
    	\coordinate (p131) at ($(p130)+1*(vh1)$);
    	\coordinate (p132) at ($(p130)+2*(vh1)$);
    	\coordinate (p133) at ($(p130)+3*(vh1)$);
    	
    	\coordinate (p200) at (p103);
    	\coordinate (p201) at ($(p200)+1*(vh2)$);
    	\coordinate (p202) at ($(p200)+2*(vh2)$);
    	\coordinate (p203) at ($(p200)+3*(vh2)$);
    	\coordinate (p210) at ($(p200)+1*(vv2)$);
    	\coordinate (p211) at ($(p210)+1*(vh2)$);
		\coordinate (p212) at ($(p210)+2*(vh2)$);
    	\coordinate (p213) at ($(p210)+3*(vh2)$);
    	\coordinate (p220) at ($(p200)+2*(vv2)$);
    	\coordinate (p221) at ($(p220)+1*(vh2)$);
    	\coordinate (p222) at ($(p220)+2*(vh2)$);
    	\coordinate (p223) at ($(p220)+3*(vh2)$);
    	\coordinate (p230) at ($(p200)+3*(vv2)$);
    	\coordinate (p231) at ($(p230)+1*(vh2)$);
    	\coordinate (p232) at ($(p230)+2*(vh2)$);
    	\coordinate (p233) at ($(p230)+3*(vh2)$);
    	
		\coordinate (p300) at (p130);
    	\coordinate (p301) at ($(p300)+1*(vh3)$);
    	\coordinate (p302) at ($(p300)+2*(vh3)$);
    	\coordinate (p303) at ($(p300)+3*(vh3)$);
    	\coordinate (p310) at ($(p300)+1*(vv3)$);
    	\coordinate (p311) at ($(p310)+1*(vh3)$);
		\coordinate (p312) at ($(p310)+2*(vh3)$);
    	\coordinate (p313) at ($(p310)+3*(vh3)$);
    	\coordinate (p320) at ($(p300)+2*(vv3)$);
    	\coordinate (p321) at ($(p320)+1*(vh3)$);
    	\coordinate (p322) at ($(p320)+2*(vh3)$);
    	\coordinate (p323) at ($(p320)+3*(vh3)$);
    	\coordinate (p330) at ($(p300)+3*(vv3)$);
    	\coordinate (p331) at ($(p330)+1*(vh3)$);
    	\coordinate (p332) at ($(p330)+2*(vh3)$);
    	\coordinate (p333) at ($(p330)+3*(vh3)$);  	  	
    	
		\filldraw[square, fill=\CAA] (p100) -- (p101) -- (p111) -- (p110);
		\filldraw[square, fill=\CAB] (p101) -- (p102) -- (p112) -- (p111);
		\filldraw[square, fill=\CAC] (p102) -- (p103) -- (p113) -- (p112);
		\filldraw[square, fill=\CAD] (p110) -- (p111) -- (p121) -- (p120);
		\filldraw[square, fill=\CAE] (p111) -- (p112) -- (p122) -- (p121);
		\filldraw[square, fill=\CAF] (p112) -- (p113) -- (p123) -- (p122);
		\filldraw[square, fill=\CAG] (p120) -- (p121) -- (p131) -- (p130);
		\filldraw[square, fill=\CAH] (p121) -- (p122) -- (p132) -- (p131);
		\filldraw[square, fill=\CAI] (p122) -- (p123) -- (p133) -- (p132);
		
		\filldraw[square, fill=\CBA] (p200) -- (p201) -- (p211) -- (p210);
		\filldraw[square, fill=\CBB] (p201) -- (p202) -- (p212) -- (p211);
		\filldraw[square, fill=\CBC] (p202) -- (p203) -- (p213) -- (p212);
		\filldraw[square, fill=\CBD] (p210) -- (p211) -- (p221) -- (p220);
		\filldraw[square, fill=\CBE] (p211) -- (p212) -- (p222) -- (p221);
		\filldraw[square, fill=\CBF] (p212) -- (p213) -- (p223) -- (p222);
		\filldraw[square, fill=\CBG] (p220) -- (p221) -- (p231) -- (p230);
		\filldraw[square, fill=\CBH] (p221) -- (p222) -- (p232) -- (p231);
		\filldraw[square, fill=\CBI] (p222) -- (p223) -- (p233) -- (p232);
		
		\filldraw[square, fill=\CCA] (p300) -- (p301) -- (p311) -- (p310);
		\filldraw[square, fill=\CCB] (p301) -- (p302) -- (p312) -- (p311);
		\filldraw[square, fill=\CCC] (p302) -- (p303) -- (p313) -- (p312);
		\filldraw[square, fill=\CCD] (p310) -- (p311) -- (p321) -- (p320);
		\filldraw[square, fill=\CCE] (p311) -- (p312) -- (p322) -- (p321);
		\filldraw[square, fill=\CCF] (p312) -- (p313) -- (p323) -- (p322);
		\filldraw[square, fill=\CCG] (p320) -- (p321) -- (p331) -- (p330);
		\filldraw[square, fill=\CCH] (p321) -- (p322) -- (p332) -- (p331);
		\filldraw[square, fill=\CCI] (p322) -- (p323) -- (p333) -- (p332);
	
    	\draw[side] (p100) -- (p200) -- (p203) -- (p233) -- (p330) -- (p130) -- (p100) -- (p200);
    	\draw[side] (p130) -- (p133) -- (p233);
    	\draw[side] (p103) -- (p133);
    	
    	\draw[square] (p101) -- (p131) -- (p331);
    	\draw[square] (p102) -- (p132) -- (p332);
    	\draw[square] (p201) -- (p231) -- (p310);
    	\draw[square] (p202) -- (p232) -- (p320);
    	\draw[square] (p110) -- (p113) -- (p213);
    	\draw[square] (p120) -- (p123) -- (p223);
	\draw[doubled]($(p321)+0.5*(vv3)$) -- ($(p232)+0.5*(vh2)$) -- ($(p212)+0.5*(vh2)$);
	\end{tikzpicture}
	}   
	\caption*{\label{fig:mouvB'} \LARGE{$B'$}}
\end{subfigure}
\showto{french}{\caption{\label{fig:mouv}Les différents mouvements standard sont montrés.}}
\showto{english}{\caption{\label{fig:mouv}Here are the different movements}}
\end{figure}

\begin{shownto}{french}
Il y a en plus quelques mouvements spéciaux. Un mouvement définis par une lettre minuscule suit le même principes qu'avec une lettre majuscule, sauf qu’il faut faire tourner deux faces, comme montrer (fig.\ref{fig:specialMouv}) avec la face frontale. Les doubles rotations sont aussi utile pour simplifier les notations, il s'agit simplement de rotation de 180° de la face, cela est indiqué par le symbole mathématique carré ''$\phantom{i}^2$''.
\end{shownto}

\begin{shownto}{english}
There are also some special movements. A movement defined by a lowercase letter follows the same principles as with a capital letter, except that two faces must be rotated, as shown (fig. \ref{fig:specialMouv}) with the front face. Double rotations are also useful to simplify the notations, it is simply a 180° rotation of the face, this is indicated by the square mathematical symbol ''$\phantom{i}^2$ ''.
\end{shownto}

\begin{figure}[H]
\def\W{0.25}
\def\S{5mm}
\centering
\begin{subfigure}[b]{\W\columnwidth}
\centering
	\def\CAA{cred}
	\def\CAB{cred}
	\def\CAC{cred}
	\def\CAD{cred}
	\def\CAE{cred}
	\def\CAF{cred}
	\def\CAG{cred}
	\def\CAH{cred}
	\def\CAI{cred}
	
	\def\CBA{cwhite}
	\def\CBB{cwhite}
	\def\CBC{cblue}
	\def\CBD{cwhite}
	\def\CBE{cwhite}
	\def\CBF{cblue}
	\def\CBG{cwhite}
	\def\CBH{cwhite}
	\def\CBI{cblue}
	
	\def\CCA{cgreen}
	\def\CCB{cgreen}
	\def\CCC{cgreen}
	\def\CCD{cgreen}
	\def\CCE{cgreen}
	\def\CCF{cgreen}
	\def\CCG{cwhite}
	\def\CCH{cwhite}
	\def\CCI{cwhite}
	\resizebox{\columnwidth}{!}
	{
		\def\OFFSET{1.4cm}
	\def\OFF{2.5pt}
	\def\RATIO{0.2}
    \def\HXA{2.5cm}
    \def\HYA{-0.5cm}
    \def\VXA{0.0cm}
    \def\VYA{2.55cm}
    \def\HXB{1.5cm}
    \def\HYB{0.85cm}
    \def\VXB{\VXA}
    \def\VYB{\VYA}
    \def\HXC{\HXA}
    \def\HYC{\HYA}
    \def\VXC{\HXB}
    \def\VYC{\HYB}
    \tikzstyle{square}=[line width=3pt, join=round, cap=round]
    \tikzstyle{side}=[line width=5pt, join=round, cap=round]
    \tikzstyle{arrow}=[line width=9pt, join=round, cap=round,->,rounded corners=3cm, cpurple]
    %\tikzstyle{double arrow}=[9pt colored by black and white]
    \tikzstyle{border}=[line width=2pt, join=round, cap=round]
    \begin{tikzpicture}[>=triangle 45]
		\coordinate (vh1) at (\HXA ,\HYA);
		\coordinate (vv1) at (\VXA, \VYA);
		\coordinate (vh2) at (\HXB, \HYB);
		\coordinate (vv2) at (\VXB, \VYB);
		\coordinate (vh3) at (\HXC, \HYC);
		\coordinate (vv3) at (\VXC, \VYC);
		    
    	\coordinate (p100) at (0,0);
    	\coordinate (p101) at ($(p100)+1*(vh1)$);
    	\coordinate (p102) at ($(p100)+2*(vh1)$);
    	\coordinate (p103) at ($(p100)+3*(vh1)$);
    	\coordinate (p110) at ($(p100)+1*(vv1)$);
    	\coordinate (p111) at ($(p110)+1*(vh1)$);
		\coordinate (p112) at ($(p110)+2*(vh1)$);
    	\coordinate (p113) at ($(p110)+3*(vh1)$);
    	\coordinate (p120) at ($(p100)+2*(vv1)$);
    	\coordinate (p121) at ($(p120)+1*(vh1)$);
    	\coordinate (p122) at ($(p120)+2*(vh1)$);
    	\coordinate (p123) at ($(p120)+3*(vh1)$);
    	\coordinate (p130) at ($(p100)+3*(vv1)$);
    	\coordinate (p131) at ($(p130)+1*(vh1)$);
    	\coordinate (p132) at ($(p130)+2*(vh1)$);
    	\coordinate (p133) at ($(p130)+3*(vh1)$);
    	
    	\coordinate (p200) at (p103);
    	\coordinate (p201) at ($(p200)+1*(vh2)$);
    	\coordinate (p202) at ($(p200)+2*(vh2)$);
    	\coordinate (p203) at ($(p200)+3*(vh2)$);
    	\coordinate (p210) at ($(p200)+1*(vv2)$);
    	\coordinate (p211) at ($(p210)+1*(vh2)$);
		\coordinate (p212) at ($(p210)+2*(vh2)$);
    	\coordinate (p213) at ($(p210)+3*(vh2)$);
    	\coordinate (p220) at ($(p200)+2*(vv2)$);
    	\coordinate (p221) at ($(p220)+1*(vh2)$);
    	\coordinate (p222) at ($(p220)+2*(vh2)$);
    	\coordinate (p223) at ($(p220)+3*(vh2)$);
    	\coordinate (p230) at ($(p200)+3*(vv2)$);
    	\coordinate (p231) at ($(p230)+1*(vh2)$);
    	\coordinate (p232) at ($(p230)+2*(vh2)$);
    	\coordinate (p233) at ($(p230)+3*(vh2)$);
    	
		\coordinate (p300) at (p130);
    	\coordinate (p301) at ($(p300)+1*(vh3)$);
    	\coordinate (p302) at ($(p300)+2*(vh3)$);
    	\coordinate (p303) at ($(p300)+3*(vh3)$);
    	\coordinate (p310) at ($(p300)+1*(vv3)$);
    	\coordinate (p311) at ($(p310)+1*(vh3)$);
		\coordinate (p312) at ($(p310)+2*(vh3)$);
    	\coordinate (p313) at ($(p310)+3*(vh3)$);
    	\coordinate (p320) at ($(p300)+2*(vv3)$);
    	\coordinate (p321) at ($(p320)+1*(vh3)$);
    	\coordinate (p322) at ($(p320)+2*(vh3)$);
    	\coordinate (p323) at ($(p320)+3*(vh3)$);
    	\coordinate (p330) at ($(p300)+3*(vv3)$);
    	\coordinate (p331) at ($(p330)+1*(vh3)$);
    	\coordinate (p332) at ($(p330)+2*(vh3)$);
    	\coordinate (p333) at ($(p330)+3*(vh3)$);  	  	
    	
		\filldraw[square, fill=\CAA] (p100) -- (p101) -- (p111) -- (p110);
		\filldraw[square, fill=\CAB] (p101) -- (p102) -- (p112) -- (p111);
		\filldraw[square, fill=\CAC] (p102) -- (p103) -- (p113) -- (p112);
		\filldraw[square, fill=\CAD] (p110) -- (p111) -- (p121) -- (p120);
		\filldraw[square, fill=\CAE] (p111) -- (p112) -- (p122) -- (p121);
		\filldraw[square, fill=\CAF] (p112) -- (p113) -- (p123) -- (p122);
		\filldraw[square, fill=\CAG] (p120) -- (p121) -- (p131) -- (p130);
		\filldraw[square, fill=\CAH] (p121) -- (p122) -- (p132) -- (p131);
		\filldraw[square, fill=\CAI] (p122) -- (p123) -- (p133) -- (p132);
		
		\filldraw[square, fill=\CBA] (p200) -- (p201) -- (p211) -- (p210);
		\filldraw[square, fill=\CBB] (p201) -- (p202) -- (p212) -- (p211);
		\filldraw[square, fill=\CBC] (p202) -- (p203) -- (p213) -- (p212);
		\filldraw[square, fill=\CBD] (p210) -- (p211) -- (p221) -- (p220);
		\filldraw[square, fill=\CBE] (p211) -- (p212) -- (p222) -- (p221);
		\filldraw[square, fill=\CBF] (p212) -- (p213) -- (p223) -- (p222);
		\filldraw[square, fill=\CBG] (p220) -- (p221) -- (p231) -- (p230);
		\filldraw[square, fill=\CBH] (p221) -- (p222) -- (p232) -- (p231);
		\filldraw[square, fill=\CBI] (p222) -- (p223) -- (p233) -- (p232);
		
		\filldraw[square, fill=\CCA] (p300) -- (p301) -- (p311) -- (p310);
		\filldraw[square, fill=\CCB] (p301) -- (p302) -- (p312) -- (p311);
		\filldraw[square, fill=\CCC] (p302) -- (p303) -- (p313) -- (p312);
		\filldraw[square, fill=\CCD] (p310) -- (p311) -- (p321) -- (p320);
		\filldraw[square, fill=\CCE] (p311) -- (p312) -- (p322) -- (p321);
		\filldraw[square, fill=\CCF] (p312) -- (p313) -- (p323) -- (p322);
		\filldraw[square, fill=\CCG] (p320) -- (p321) -- (p331) -- (p330);
		\filldraw[square, fill=\CCH] (p321) -- (p322) -- (p332) -- (p331);
		\filldraw[square, fill=\CCI] (p322) -- (p323) -- (p333) -- (p332);
	
    	\draw[side] (p100) -- (p200) -- (p203) -- (p233) -- (p330) -- (p130) -- (p100) -- (p200);
    	\draw[side] (p130) -- (p133) -- (p233);
    	\draw[side] (p103) -- (p133);
    	
    	\draw[square] (p101) -- (p131) -- (p331);
    	\draw[square] (p102) -- (p132) -- (p332);
    	\draw[square] (p201) -- (p231) -- (p310);
    	\draw[square] (p202) -- (p232) -- (p320);
    	\draw[square] (p110) -- (p113) -- (p213);
    	\draw[square] (p120) -- (p123) -- (p223);
	\draw[doubled]($(p301)+0.5*(vv3)$) -- ($(p303)+0.5*(vv3)$) -- ($(p210)+0.5*(vh2)$);
	\draw[doubled]($(p311)+0.5*(vv3)$) -- ($(p313)+0.5*(vv3)$) -- ($(p211)+0.5*(vh2)$);
	\end{tikzpicture}
	}
	\caption*{\label{fig:mouvf} \LARGE{$f$}}    	
\end{subfigure}
\hspace{\S}
\begin{subfigure}[b]{\W\columnwidth}
\centering
	\def\CAA{cred}
	\def\CAB{cred}
	\def\CAC{cred}
	\def\CAD{cred}
	\def\CAE{cred}
	\def\CAF{cred}
	\def\CAG{cred}
	\def\CAH{cred}
	\def\CAI{cred}
	
	\def\CBA{cyellow}
	\def\CBB{cyellow}
	\def\CBC{cblue}
	\def\CBD{cyellow}
	\def\CBE{cyellow}
	\def\CBF{cblue}
	\def\CBG{cyellow}
	\def\CBH{cyellow}
	\def\CBI{cblue}
	
	\def\CCA{cblue}
	\def\CCB{cblue}
	\def\CCC{cblue}
	\def\CCD{cblue}
	\def\CCE{cblue}
	\def\CCF{cblue}
	\def\CCG{cwhite}
	\def\CCH{cwhite}
	\def\CCI{cwhite}	
	\resizebox{\columnwidth}{!}
	{
		\def\OFFSET{1.4cm}
	\def\OFF{2.5pt}
	\def\RATIO{0.2}
    \def\HXA{2.5cm}
    \def\HYA{-0.5cm}
    \def\VXA{0.0cm}
    \def\VYA{2.55cm}
    \def\HXB{1.5cm}
    \def\HYB{0.85cm}
    \def\VXB{\VXA}
    \def\VYB{\VYA}
    \def\HXC{\HXA}
    \def\HYC{\HYA}
    \def\VXC{\HXB}
    \def\VYC{\HYB}
    \tikzstyle{square}=[line width=3pt, join=round, cap=round]
    \tikzstyle{side}=[line width=5pt, join=round, cap=round]
    \tikzstyle{arrow}=[line width=9pt, join=round, cap=round,->,rounded corners=3cm, cpurple]
    %\tikzstyle{double arrow}=[9pt colored by black and white]
    \tikzstyle{border}=[line width=2pt, join=round, cap=round]
    \begin{tikzpicture}[>=triangle 45]
		\coordinate (vh1) at (\HXA ,\HYA);
		\coordinate (vv1) at (\VXA, \VYA);
		\coordinate (vh2) at (\HXB, \HYB);
		\coordinate (vv2) at (\VXB, \VYB);
		\coordinate (vh3) at (\HXC, \HYC);
		\coordinate (vv3) at (\VXC, \VYC);
		    
    	\coordinate (p100) at (0,0);
    	\coordinate (p101) at ($(p100)+1*(vh1)$);
    	\coordinate (p102) at ($(p100)+2*(vh1)$);
    	\coordinate (p103) at ($(p100)+3*(vh1)$);
    	\coordinate (p110) at ($(p100)+1*(vv1)$);
    	\coordinate (p111) at ($(p110)+1*(vh1)$);
		\coordinate (p112) at ($(p110)+2*(vh1)$);
    	\coordinate (p113) at ($(p110)+3*(vh1)$);
    	\coordinate (p120) at ($(p100)+2*(vv1)$);
    	\coordinate (p121) at ($(p120)+1*(vh1)$);
    	\coordinate (p122) at ($(p120)+2*(vh1)$);
    	\coordinate (p123) at ($(p120)+3*(vh1)$);
    	\coordinate (p130) at ($(p100)+3*(vv1)$);
    	\coordinate (p131) at ($(p130)+1*(vh1)$);
    	\coordinate (p132) at ($(p130)+2*(vh1)$);
    	\coordinate (p133) at ($(p130)+3*(vh1)$);
    	
    	\coordinate (p200) at (p103);
    	\coordinate (p201) at ($(p200)+1*(vh2)$);
    	\coordinate (p202) at ($(p200)+2*(vh2)$);
    	\coordinate (p203) at ($(p200)+3*(vh2)$);
    	\coordinate (p210) at ($(p200)+1*(vv2)$);
    	\coordinate (p211) at ($(p210)+1*(vh2)$);
		\coordinate (p212) at ($(p210)+2*(vh2)$);
    	\coordinate (p213) at ($(p210)+3*(vh2)$);
    	\coordinate (p220) at ($(p200)+2*(vv2)$);
    	\coordinate (p221) at ($(p220)+1*(vh2)$);
    	\coordinate (p222) at ($(p220)+2*(vh2)$);
    	\coordinate (p223) at ($(p220)+3*(vh2)$);
    	\coordinate (p230) at ($(p200)+3*(vv2)$);
    	\coordinate (p231) at ($(p230)+1*(vh2)$);
    	\coordinate (p232) at ($(p230)+2*(vh2)$);
    	\coordinate (p233) at ($(p230)+3*(vh2)$);
    	
		\coordinate (p300) at (p130);
    	\coordinate (p301) at ($(p300)+1*(vh3)$);
    	\coordinate (p302) at ($(p300)+2*(vh3)$);
    	\coordinate (p303) at ($(p300)+3*(vh3)$);
    	\coordinate (p310) at ($(p300)+1*(vv3)$);
    	\coordinate (p311) at ($(p310)+1*(vh3)$);
		\coordinate (p312) at ($(p310)+2*(vh3)$);
    	\coordinate (p313) at ($(p310)+3*(vh3)$);
    	\coordinate (p320) at ($(p300)+2*(vv3)$);
    	\coordinate (p321) at ($(p320)+1*(vh3)$);
    	\coordinate (p322) at ($(p320)+2*(vh3)$);
    	\coordinate (p323) at ($(p320)+3*(vh3)$);
    	\coordinate (p330) at ($(p300)+3*(vv3)$);
    	\coordinate (p331) at ($(p330)+1*(vh3)$);
    	\coordinate (p332) at ($(p330)+2*(vh3)$);
    	\coordinate (p333) at ($(p330)+3*(vh3)$);  	  	
    	
		\filldraw[square, fill=\CAA] (p100) -- (p101) -- (p111) -- (p110);
		\filldraw[square, fill=\CAB] (p101) -- (p102) -- (p112) -- (p111);
		\filldraw[square, fill=\CAC] (p102) -- (p103) -- (p113) -- (p112);
		\filldraw[square, fill=\CAD] (p110) -- (p111) -- (p121) -- (p120);
		\filldraw[square, fill=\CAE] (p111) -- (p112) -- (p122) -- (p121);
		\filldraw[square, fill=\CAF] (p112) -- (p113) -- (p123) -- (p122);
		\filldraw[square, fill=\CAG] (p120) -- (p121) -- (p131) -- (p130);
		\filldraw[square, fill=\CAH] (p121) -- (p122) -- (p132) -- (p131);
		\filldraw[square, fill=\CAI] (p122) -- (p123) -- (p133) -- (p132);
		
		\filldraw[square, fill=\CBA] (p200) -- (p201) -- (p211) -- (p210);
		\filldraw[square, fill=\CBB] (p201) -- (p202) -- (p212) -- (p211);
		\filldraw[square, fill=\CBC] (p202) -- (p203) -- (p213) -- (p212);
		\filldraw[square, fill=\CBD] (p210) -- (p211) -- (p221) -- (p220);
		\filldraw[square, fill=\CBE] (p211) -- (p212) -- (p222) -- (p221);
		\filldraw[square, fill=\CBF] (p212) -- (p213) -- (p223) -- (p222);
		\filldraw[square, fill=\CBG] (p220) -- (p221) -- (p231) -- (p230);
		\filldraw[square, fill=\CBH] (p221) -- (p222) -- (p232) -- (p231);
		\filldraw[square, fill=\CBI] (p222) -- (p223) -- (p233) -- (p232);
		
		\filldraw[square, fill=\CCA] (p300) -- (p301) -- (p311) -- (p310);
		\filldraw[square, fill=\CCB] (p301) -- (p302) -- (p312) -- (p311);
		\filldraw[square, fill=\CCC] (p302) -- (p303) -- (p313) -- (p312);
		\filldraw[square, fill=\CCD] (p310) -- (p311) -- (p321) -- (p320);
		\filldraw[square, fill=\CCE] (p311) -- (p312) -- (p322) -- (p321);
		\filldraw[square, fill=\CCF] (p312) -- (p313) -- (p323) -- (p322);
		\filldraw[square, fill=\CCG] (p320) -- (p321) -- (p331) -- (p330);
		\filldraw[square, fill=\CCH] (p321) -- (p322) -- (p332) -- (p331);
		\filldraw[square, fill=\CCI] (p322) -- (p323) -- (p333) -- (p332);
	
    	\draw[side] (p100) -- (p200) -- (p203) -- (p233) -- (p330) -- (p130) -- (p100) -- (p200);
    	\draw[side] (p130) -- (p133) -- (p233);
    	\draw[side] (p103) -- (p133);
    	
    	\draw[square] (p101) -- (p131) -- (p331);
    	\draw[square] (p102) -- (p132) -- (p332);
    	\draw[square] (p201) -- (p231) -- (p310);
    	\draw[square] (p202) -- (p232) -- (p320);
    	\draw[square] (p110) -- (p113) -- (p213);
    	\draw[square] (p120) -- (p123) -- (p223);
	\draw[doubled]($(p210)+0.5*(vh2)$) -- ($(p303)+0.5*(vv3)$) -- ($(p301)+0.5*(vv3)$);
	\draw[doubled]($(p211)+0.5*(vh2)$) -- ($(p313)+0.5*(vv3)$) -- ($(p311)+0.5*(vv3)$);
	\end{tikzpicture}
	}
	\caption*{\label{fig:mouvf'} \LARGE{$f'$}}    	
\end{subfigure}
\hspace{\S}
\begin{subfigure}[b]{\W\columnwidth}
\centering
	\def\CAA{cred}
	\def\CAB{cred}
	\def\CAC{cred}
	\def\CAD{cred}
	\def\CAE{cred}
	\def\CAF{cred}
	\def\CAG{corange}
	\def\CAH{corange}
	\def\CAI{corange}
	
	\def\CBA{cblue}
	\def\CBB{cblue}
	\def\CBC{cblue}
	\def\CBD{cblue}
	\def\CBE{cblue}
	\def\CBF{cblue}
	\def\CBG{cgreen}
	\def\CBH{cgreen}
	\def\CBI{cgreen}
	
	\def\CCA{cwhite}
	\def\CCB{cwhite}
	\def\CCC{cwhite}
	\def\CCD{cwhite}
	\def\CCE{cwhite}
	\def\CCF{cwhite}
	\def\CCG{cwhite}
	\def\CCH{cwhite}
	\def\CCI{cwhite}
	\resizebox{\columnwidth}{!}
	{
		\def\OFFSET{1.4cm}
	\def\OFF{2.5pt}
	\def\RATIO{0.2}
    \def\HXA{2.5cm}
    \def\HYA{-0.5cm}
    \def\VXA{0.0cm}
    \def\VYA{2.55cm}
    \def\HXB{1.5cm}
    \def\HYB{0.85cm}
    \def\VXB{\VXA}
    \def\VYB{\VYA}
    \def\HXC{\HXA}
    \def\HYC{\HYA}
    \def\VXC{\HXB}
    \def\VYC{\HYB}
    \tikzstyle{square}=[line width=3pt, join=round, cap=round]
    \tikzstyle{side}=[line width=5pt, join=round, cap=round]
    \tikzstyle{arrow}=[line width=9pt, join=round, cap=round,->,rounded corners=3cm, cpurple]
    %\tikzstyle{double arrow}=[9pt colored by black and white]
    \tikzstyle{border}=[line width=2pt, join=round, cap=round]
    \begin{tikzpicture}[>=triangle 45]
		\coordinate (vh1) at (\HXA ,\HYA);
		\coordinate (vv1) at (\VXA, \VYA);
		\coordinate (vh2) at (\HXB, \HYB);
		\coordinate (vv2) at (\VXB, \VYB);
		\coordinate (vh3) at (\HXC, \HYC);
		\coordinate (vv3) at (\VXC, \VYC);
		    
    	\coordinate (p100) at (0,0);
    	\coordinate (p101) at ($(p100)+1*(vh1)$);
    	\coordinate (p102) at ($(p100)+2*(vh1)$);
    	\coordinate (p103) at ($(p100)+3*(vh1)$);
    	\coordinate (p110) at ($(p100)+1*(vv1)$);
    	\coordinate (p111) at ($(p110)+1*(vh1)$);
		\coordinate (p112) at ($(p110)+2*(vh1)$);
    	\coordinate (p113) at ($(p110)+3*(vh1)$);
    	\coordinate (p120) at ($(p100)+2*(vv1)$);
    	\coordinate (p121) at ($(p120)+1*(vh1)$);
    	\coordinate (p122) at ($(p120)+2*(vh1)$);
    	\coordinate (p123) at ($(p120)+3*(vh1)$);
    	\coordinate (p130) at ($(p100)+3*(vv1)$);
    	\coordinate (p131) at ($(p130)+1*(vh1)$);
    	\coordinate (p132) at ($(p130)+2*(vh1)$);
    	\coordinate (p133) at ($(p130)+3*(vh1)$);
    	
    	\coordinate (p200) at (p103);
    	\coordinate (p201) at ($(p200)+1*(vh2)$);
    	\coordinate (p202) at ($(p200)+2*(vh2)$);
    	\coordinate (p203) at ($(p200)+3*(vh2)$);
    	\coordinate (p210) at ($(p200)+1*(vv2)$);
    	\coordinate (p211) at ($(p210)+1*(vh2)$);
		\coordinate (p212) at ($(p210)+2*(vh2)$);
    	\coordinate (p213) at ($(p210)+3*(vh2)$);
    	\coordinate (p220) at ($(p200)+2*(vv2)$);
    	\coordinate (p221) at ($(p220)+1*(vh2)$);
    	\coordinate (p222) at ($(p220)+2*(vh2)$);
    	\coordinate (p223) at ($(p220)+3*(vh2)$);
    	\coordinate (p230) at ($(p200)+3*(vv2)$);
    	\coordinate (p231) at ($(p230)+1*(vh2)$);
    	\coordinate (p232) at ($(p230)+2*(vh2)$);
    	\coordinate (p233) at ($(p230)+3*(vh2)$);
    	
		\coordinate (p300) at (p130);
    	\coordinate (p301) at ($(p300)+1*(vh3)$);
    	\coordinate (p302) at ($(p300)+2*(vh3)$);
    	\coordinate (p303) at ($(p300)+3*(vh3)$);
    	\coordinate (p310) at ($(p300)+1*(vv3)$);
    	\coordinate (p311) at ($(p310)+1*(vh3)$);
		\coordinate (p312) at ($(p310)+2*(vh3)$);
    	\coordinate (p313) at ($(p310)+3*(vh3)$);
    	\coordinate (p320) at ($(p300)+2*(vv3)$);
    	\coordinate (p321) at ($(p320)+1*(vh3)$);
    	\coordinate (p322) at ($(p320)+2*(vh3)$);
    	\coordinate (p323) at ($(p320)+3*(vh3)$);
    	\coordinate (p330) at ($(p300)+3*(vv3)$);
    	\coordinate (p331) at ($(p330)+1*(vh3)$);
    	\coordinate (p332) at ($(p330)+2*(vh3)$);
    	\coordinate (p333) at ($(p330)+3*(vh3)$);  	  	
    	
		\filldraw[square, fill=\CAA] (p100) -- (p101) -- (p111) -- (p110);
		\filldraw[square, fill=\CAB] (p101) -- (p102) -- (p112) -- (p111);
		\filldraw[square, fill=\CAC] (p102) -- (p103) -- (p113) -- (p112);
		\filldraw[square, fill=\CAD] (p110) -- (p111) -- (p121) -- (p120);
		\filldraw[square, fill=\CAE] (p111) -- (p112) -- (p122) -- (p121);
		\filldraw[square, fill=\CAF] (p112) -- (p113) -- (p123) -- (p122);
		\filldraw[square, fill=\CAG] (p120) -- (p121) -- (p131) -- (p130);
		\filldraw[square, fill=\CAH] (p121) -- (p122) -- (p132) -- (p131);
		\filldraw[square, fill=\CAI] (p122) -- (p123) -- (p133) -- (p132);
		
		\filldraw[square, fill=\CBA] (p200) -- (p201) -- (p211) -- (p210);
		\filldraw[square, fill=\CBB] (p201) -- (p202) -- (p212) -- (p211);
		\filldraw[square, fill=\CBC] (p202) -- (p203) -- (p213) -- (p212);
		\filldraw[square, fill=\CBD] (p210) -- (p211) -- (p221) -- (p220);
		\filldraw[square, fill=\CBE] (p211) -- (p212) -- (p222) -- (p221);
		\filldraw[square, fill=\CBF] (p212) -- (p213) -- (p223) -- (p222);
		\filldraw[square, fill=\CBG] (p220) -- (p221) -- (p231) -- (p230);
		\filldraw[square, fill=\CBH] (p221) -- (p222) -- (p232) -- (p231);
		\filldraw[square, fill=\CBI] (p222) -- (p223) -- (p233) -- (p232);
		
		\filldraw[square, fill=\CCA] (p300) -- (p301) -- (p311) -- (p310);
		\filldraw[square, fill=\CCB] (p301) -- (p302) -- (p312) -- (p311);
		\filldraw[square, fill=\CCC] (p302) -- (p303) -- (p313) -- (p312);
		\filldraw[square, fill=\CCD] (p310) -- (p311) -- (p321) -- (p320);
		\filldraw[square, fill=\CCE] (p311) -- (p312) -- (p322) -- (p321);
		\filldraw[square, fill=\CCF] (p312) -- (p313) -- (p323) -- (p322);
		\filldraw[square, fill=\CCG] (p320) -- (p321) -- (p331) -- (p330);
		\filldraw[square, fill=\CCH] (p321) -- (p322) -- (p332) -- (p331);
		\filldraw[square, fill=\CCI] (p322) -- (p323) -- (p333) -- (p332);
	
    	\draw[side] (p100) -- (p200) -- (p203) -- (p233) -- (p330) -- (p130) -- (p100) -- (p200);
    	\draw[side] (p130) -- (p133) -- (p233);
    	\draw[side] (p103) -- (p133);
    	
    	\draw[square] (p101) -- (p131) -- (p331);
    	\draw[square] (p102) -- (p132) -- (p332);
    	\draw[square] (p201) -- (p231) -- (p310);
    	\draw[square] (p202) -- (p232) -- (p320);
    	\draw[square] (p110) -- (p113) -- (p213);
    	\draw[square] (p120) -- (p123) -- (p223);
	\draw[doubled]($(p331)+0.5*(vv3)$) -- ($(p223)+0.5*(vv2)$) -- ($(p220)+0.5*(vv2)$) -- ($(p121)+0.5*(vv1)$);
	\end{tikzpicture}
	}
	\caption*{\label{fig:mouvU2} \LARGE{$U^2$}}    
\end{subfigure}
\showto{french}{\caption{\label{fig:specialMouv}Les mouvements spéciaux sont présentés ici.}}
\showto{english}{\caption{\label{fig:specialMouv}Special movements are shown here.}}
\end{figure}

\end{document}
